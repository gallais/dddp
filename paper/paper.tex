%% For double-blind review submission, w/o CCS and ACM Reference (max submission space)
\documentclass[sigplan]{acmart}%\settopmatter{printfolios=true,printccs=false,printacmref=false}
%% For double-blind review submission, w/ CCS and ACM Reference
%\documentclass[sigplan,review,anonymous]{acmart}\settopmatter{printfolios=true}
%% For single-blind review submission, w/o CCS and ACM Reference (max submission space)
%\documentclass[sigplan,review]{acmart}\settopmatter{printfolios=true,printccs=false,printacmref=false}
%% For single-blind review submission, w/ CCS and ACM Reference
%\documentclass[sigplan,review]{acmart}\settopmatter{printfolios=true}
%% For final camera-ready submission, w/ required CCS and ACM Reference
%\documentclass[sigplan]{acmart}\settopmatter{}
\usepackage{flushend}
\usepackage{soul}
%% Conference information
%% Supplied to authors by publisher for camera-ready submission;
%% use defaults for review submission.

%%% The following is specific to TyDe '19 and the paper
%%% 'Deferring the Details and Deriving Programs'
%%% by Liam O'Connor.
%%%
\setcopyright{acmlicensed}
\acmPrice{15.00}
\acmDOI{10.1145/3331554.3342605}
\acmYear{2019}
\copyrightyear{2019}
\acmISBN{978-1-4503-6815-5/19/08}
\acmConference[TyDe '19]{Proceedings of the 4th ACM SIGPLAN International Workshop on Type-Driven Development}{August 18, 2019}{Berlin, Germany}
\acmBooktitle{Proceedings of the 4th ACM SIGPLAN International Workshop on Type-Driven Development (TyDe '19), August 18, 2019, Berlin, Germany}
%\startPage{1}

%% Copyright information
%% Supplied to authors (based on authors' rights management selection;
%% see authors.acm.org) by publisher for camera-ready submission;
%% use 'none' for review submission.
%\setcopyright{none}
%\setcopyright{acmcopyright}
%\setcopyright{acmlicensed}
%\setcopyright{rightsretained}
%\copyrightyear{2018}           %% If different from \acmYear

%% Bibliography style
\bibliographystyle{ACM-Reference-Format}
%% Citation style
%\citestyle{acmauthoryear}  %% For author/year citations
%\citestyle{acmnumeric}     %% For numeric citations
%\setcitestyle{nosort}      %% With 'acmnumeric', to disable automatic
                            %% sorting of references within a single citation;
                            %% e.g., \cite{Smith99,Carpenter05,Baker12}
                            %% rendered as [14,5,2] rather than [2,5,14].
%\setcitesyle{nocompress}   %% With 'acmnumeric', to disable automatic
                            %% compression of sequential references within a
                            %% single citation;
                            %% e.g., \cite{Baker12,Baker14,Baker16}
                            %% rendered as [2,3,4] rather than [2-4].


%%%%%%%%%%%%%%%%%%%%%%%%%%%%%%%%%%%%%%%%%%%%%%%%%%%%%%%%%%%%%%%%%%%%%%
%% Note: Authors migrating a paper from traditional SIGPLAN
%% proceedings format to PACMPL format must update the
%% '\documentclass' and topmatter commands above; see
%% 'acmart-pacmpl-template.tex'.
%%%%%%%%%%%%%%%%%%%%%%%%%%%%%%%%%%%%%%%%%%%%%%%%%%%%%%%%%%%%%%%%%%%%%%


%% Some recommended packages.
\usepackage{booktabs}   %% For formal tables:
                        %% http://ctan.org/pkg/booktabs
\usepackage{subcaption} %% For complex figures with subfigures/subcaptions
                        %% http://ctan.org/pkg/subcaption

\usepackage{agdagremlins}
\usepackage{mathpartir}
\usepackage{tikz}
\begin{document}

%% Title information
\title{Deferring the Details and Deriving Programs} %% [Short Title] is optional;
                                        %% when present, will be used in
                                        %% header instead of Full Title.
%\titlenote{with title note}             %% \titlenote is optional;
                                        %% can be repeated if necessary;
                                        %% contents suppressed with 'anonymous'
%\subtitle{Subtitle}                     %% \subtitle is optional
%\subtitlenote{with subtitle note}       %% \subtitlenote is optional;
                                        %% can be repeated if necessary;
                                        %% contents suppressed with 'anonymous'


%% Author information
%% Contents and number of authors suppressed with 'anonymous'.
%% Each author should be introduced by \author, followed by
%% \authornote (optional), \orcid (optional), \affiliation, and
%% \email.
%% An author may have multiple affiliations and/or emails; repeat the
%% appropriate command.
%% Many elements are not rendered, but should be provided for metadata
%% extraction tools.

%% Author with single affiliation.
\author{Liam O'Connor}
%\authornote{with author1 note}          %% \authornote is optional;
                                        %% can be repeated if necessary
%\orcid{nnnn-nnnn-nnnn-nnnn}             %% \orcid is optional
\affiliation{
%  \position{Position1}
%  \department{}              %% \department is recommended
  \institution{UNSW Australia}            %% \institution is required
%  \streetaddress{}
  \city{Sydney}
  \state{NSW}
%  \postcode{Post-Code1}
  \country{Australia}                    %% \country is recommended
}
\email{liamoc@cse.unsw.edu.au}          %% \email is recommended

%% Author with two affiliations and emails.
%\author{First2 Last2}
%\authornote{with author2 note}          %% \authornote is optional;
%                                        %% can be repeated if necessary
%\orcid{nnnn-nnnn-nnnn-nnnn}             %% \orcid is optional
%\affiliation{
%  \position{Position2a}
%  \department{Department2a}             %% \department is recommended
%  \institution{Institution2a}           %% \institution is required
%  \streetaddress{Street2a Address2a}
%  \city{City2a}
%  \state{State2a}
%  \postcode{Post-Code2a}
%  \country{Country2a}                   %% \country is recommended
%}
%\email{first2.last2@inst2a.com}         %% \email is recommended
%\affiliation{
%  \position{Position2b}
%  \department{Department2b}             %% \department is recommended
%  \institution{Institution2b}           %% \institution is required
%  \streetaddress{Street3b Address2b}
%  \city{City2b}
%  \state{State2b}
%  \postcode{Post-Code2b}
%  \country{Country2b}                   %% \country is recommended
%}
%\email{first2.last2@inst2b.org}         %% \email is recommended
%

%% Abstract
%% Note: \begin{abstract}...\end{abstract} environment must come
%% before \maketitle command
\def\figureautorefname{Figure}
\def\sectionautorefname{Section}
\def\subsectionautorefname{Section}
\newcommand{\TODO}[1]{\textbf{TODO} \hl{#1}}
  \newcommand{\trip}[3]{ \{ #1 \}\ #2\ \{ #3\} }
  \newcommand{\pee}{\mathcal{P}}
  \newcommand{\quu}{\mathcal{Q}}
\begin{abstract}
A commonly-used technique in dependently-typed programming is to encode 
invariants about a data structure into its type, thus ensuring that 
the data structure is correct by construction.
Unfortunately, this often necessitates the embedding of explicit
proof terms within the data structure, which are not part of the structure 
conceptually, but merely supplied to ensure that the data invariants are maintained.
As the complexity of the specifications in the types increases,
these additional terms tend to clutter definitions, reducing readability.
We introduce a technique where these proof terms can be supplied later, by
constructing the data structure within a \emph{proof delay} applicative functor. 
We apply this technique to \textsc{Trip}, our new language for Hoare-logic verification
of imperative programs embedded in Agda,
where our applicative functor is used as the basis for a verification condition 
generator, turning the typed holes of Agda into a method for stepwise derivation 
of a  program from its specification in the form of a Hoare triple. 
\end{abstract}
\begin{CCSXML}
<ccs2012>
<concept>
<concept_id>10003752.10003790.10002990</concept_id>
<concept_desc>Theory of computation~Logic and verification</concept_desc>
<concept_significance>500</concept_significance>
</concept>
<concept>
<concept_id>10003752.10003790.10011741</concept_id>
<concept_desc>Theory of computation~Hoare logic</concept_desc>
<concept_significance>500</concept_significance>
</concept>
<concept>
<concept_id>10003752.10010124.10010125.10010130</concept_id>
<concept_desc>Theory of computation~Type structures</concept_desc>
<concept_significance>500</concept_significance>
</concept>
<concept>
<concept_id>10003752.10010124.10010138</concept_id>
<concept_desc>Theory of computation~Program reasoning</concept_desc>
<concept_significance>500</concept_significance>
</concept>
<concept>
<concept_id>10003752.10003790.10011740</concept_id>
<concept_desc>Theory of computation~Type theory</concept_desc>
<concept_significance>300</concept_significance>
</concept>
<concept>
<concept_id>10011007.10010940.10010992.10010998.10010999</concept_id>
<concept_desc>Software and its engineering~Software verification</concept_desc>
<concept_significance>500</concept_significance>
</concept>
<concept>
<concept_id>10011007.10011006.10011008.10011009.10011010</concept_id>
<concept_desc>Software and its engineering~Imperative languages</concept_desc>
<concept_significance>300</concept_significance>
</concept>
</ccs2012>
\end{CCSXML}

\ccsdesc[500]{Theory of computation~Logic and verification}
\ccsdesc[500]{Theory of computation~Hoare logic}
\ccsdesc[500]{Theory of computation~Type structures}
\ccsdesc[500]{Theory of computation~Program reasoning}
\ccsdesc[300]{Theory of computation~Type theory}
\ccsdesc[500]{Software and its engineering~Software verification}
\ccsdesc[300]{Software and its engineering~Imperative languages}


%% End of generated code


%% Keywords
%% comma separated list
\keywords{hoare logic, agda, verification conditions, applicative functor, types, imperative programming}  %% \keywords are mandatory in final camera-ready submission


%% \maketitle
%% Note: \maketitle command must come after title commands, author
%% commands, abstract environment, Computing Classification System
%% environment and commands, and keywords command.
\maketitle


\section{Introduction}\label{sec:intro}

In traditional proof assistants such as those in the LCF tradition, the types
used to model data structures are usually very simple inductive types. Data 
invariants are described as separate predicates, and operations are shown 
to preserve these invariants in separately-proven lemmas.

When we set about programming with dependent types, however, we have nearly unlimited 
specification power in the types themselves. Rather than 
specify a simple inductive type and prove that data invariants are maintained
by every operation, we can bake the data invariants directly into the type, thus 
ensuring that the data invariants are maintained by construction.
For example, this ordered list type from \citet{hasochism}
is parameterised by its lower and upper bounds,
requiring $n + 1$ proofs of element ordering for a list of length $n$:
\begin{code}
\>[2]\AgdaKeyword{data}\AgdaSpace{}%
\AgdaDatatype{OList}\AgdaSpace{}%
\AgdaSymbol{(}\AgdaBound{m}\AgdaSpace{}%
\AgdaBound{n}\AgdaSpace{}%
\AgdaSymbol{:}\AgdaSpace{}%
\AgdaDatatype{ℕ}\AgdaSymbol{)}\AgdaSpace{}%
\AgdaSymbol{:}\AgdaSpace{}%
\AgdaPrimitiveType{Set}\AgdaSpace{}%
\AgdaKeyword{where}\<%
\\
\>[2][@{}l@{\AgdaIndent{0}}]%
\>[4]\AgdaInductiveConstructor{Nil}%
\>[9]\AgdaSymbol{:}\AgdaSpace{}%
\AgdaSymbol{(}\AgdaBound{m}\AgdaSpace{}%
\AgdaOperator{\AgdaDatatype{≤}}\AgdaSpace{}%
\AgdaBound{n}\AgdaSymbol{)}\AgdaSpace{}%
\AgdaSymbol{→}\AgdaSpace{}%
\AgdaDatatype{OList}\AgdaSpace{}%
\AgdaBound{m}\AgdaSpace{}%
\AgdaBound{n}\<%
\\
%
\>[4]\AgdaInductiveConstructor{Cons}\AgdaSpace{}%
\AgdaSymbol{:}\AgdaSpace{}%
\AgdaSymbol{(}\AgdaBound{x}\AgdaSpace{}%
\AgdaSymbol{:}\AgdaSpace{}%
\AgdaDatatype{ℕ}\AgdaSymbol{)}\AgdaSpace{}%
\AgdaSymbol{→}\AgdaSpace{}%
\AgdaSymbol{(}\AgdaBound{m}\AgdaSpace{}%
\AgdaOperator{\AgdaDatatype{≤}}\AgdaSpace{}%
\AgdaBound{x}\AgdaSymbol{)}\AgdaSpace{}%
\AgdaSymbol{→}\AgdaSpace{}%
\AgdaDatatype{OList}\AgdaSpace{}%
\AgdaBound{x}\AgdaSpace{}%
\AgdaBound{n}\AgdaSpace{}%
\AgdaSymbol{→}\AgdaSpace{}%
\AgdaDatatype{OList}\AgdaSpace{}%
\AgdaBound{m}\AgdaSpace{}%
\AgdaBound{n}\<%
%
\end{code}
Another elegant example is the binary search tree data type 
of \citet{neighbors}, which 
carries ordering evidence in its leaves:
\begin{code}
\>[2]\AgdaKeyword{data}\AgdaSpace{}%
\AgdaDatatype{BST}\AgdaSpace{}%
\AgdaSymbol{(}\AgdaBound{m}\AgdaSpace{}%
\AgdaBound{n}\AgdaSpace{}%
\AgdaSymbol{:}\AgdaSpace{}%
\AgdaDatatype{ℕ}\AgdaSymbol{)}\AgdaSpace{}%
\AgdaSymbol{:}\AgdaSpace{}%
\AgdaPrimitiveType{Set}\AgdaSpace{}%
\AgdaKeyword{where}\<%
\\
\>[2][@{}l@{\AgdaIndent{0}}]%
\>[4]\AgdaInductiveConstructor{Leaf}\AgdaSpace{}%
\AgdaSymbol{:}\AgdaSpace{}%
\AgdaSymbol{(}\AgdaBound{m}\AgdaSpace{}%
\AgdaOperator{\AgdaDatatype{≤}}\AgdaSpace{}%
\AgdaBound{n}\AgdaSymbol{)}\AgdaSpace{}%
\AgdaSymbol{→}\AgdaSpace{}%
\AgdaDatatype{BST}\AgdaSpace{}%
\AgdaBound{m}\AgdaSpace{}%
\AgdaBound{n}\<%
\\
%
\>[4]\AgdaInductiveConstructor{Branch}%
\>[12]\AgdaSymbol{:}\AgdaSpace{}%
\AgdaSymbol{(}\AgdaBound{x}\AgdaSpace{}%
\AgdaSymbol{:}\AgdaSpace{}%
\AgdaDatatype{ℕ}\AgdaSymbol{)}\AgdaSpace{}%
\AgdaSymbol{→}\AgdaSpace{}%
\AgdaDatatype{BST}\AgdaSpace{}%
\AgdaBound{m}\AgdaSpace{}%
\AgdaBound{x}\AgdaSpace{}%
\AgdaSymbol{→}\AgdaSpace{}%
\AgdaDatatype{BST}\AgdaSpace{}%
\AgdaBound{x}\AgdaSpace{}%
\AgdaBound{n}\AgdaSpace{}%
\AgdaSymbol{→}\AgdaSpace{}%
\AgdaDatatype{BST}\AgdaSpace{}%
\AgdaBound{m}\AgdaSpace{}%
\AgdaBound{n}\<%
%
\end{code}
While these definitions seem quite appealing, attempting to 
construct even a simple two-element $\AgdaDatatype{BST}$ 
value leads to some rather unsightly results:
\begin{code}
\>[2]\AgdaFunction{tree}\AgdaSpace{}%
\AgdaSymbol{:}\AgdaSpace{}%
\AgdaDatatype{BST}\AgdaSpace{}%
\AgdaNumber{2}\AgdaSpace{}%
\AgdaNumber{3}\<%
\\
%
\>[2]\AgdaFunction{tree}\AgdaSpace{}%
\AgdaSymbol{=}%
\>[847I]\AgdaInductiveConstructor{Branch}\AgdaSpace{}%
\AgdaNumber{3}\<%
\\
\>[847I][@{}l@{\AgdaIndent{0}}]%
\>[10]\AgdaSymbol{(}\AgdaInductiveConstructor{Branch}\AgdaSpace{}%
\AgdaNumber{2}\<%
\\
\>[10][@{}l@{\AgdaIndent{0}}]%
\>[13]\AgdaSymbol{(}\AgdaInductiveConstructor{Leaf}\AgdaSpace{}%
\AgdaSymbol{(}\AgdaInductiveConstructor{s≤s}\AgdaSpace{}%
\AgdaSymbol{(}\AgdaInductiveConstructor{s≤s}\AgdaSpace{}%
\AgdaInductiveConstructor{z≤n}\AgdaSymbol{)))}\<%
\\
%
\>[13]\AgdaSymbol{(}\AgdaInductiveConstructor{Leaf}\AgdaSpace{}%
\AgdaSymbol{(}\AgdaInductiveConstructor{s≤s}\AgdaSpace{}%
\AgdaSymbol{(}\AgdaInductiveConstructor{s≤s}\AgdaSpace{}%
\AgdaInductiveConstructor{z≤n}\AgdaSymbol{))))}\<%
\\
%
\>[10]\AgdaSymbol{(}\AgdaInductiveConstructor{Leaf}\AgdaSpace{}%
\AgdaSymbol{(}\AgdaInductiveConstructor{s≤s}\AgdaSpace{}%
\AgdaSymbol{(}\AgdaInductiveConstructor{s≤s}\AgdaSpace{}%
\AgdaSymbol{(}\AgdaInductiveConstructor{s≤s}\AgdaSpace{}%
\AgdaInductiveConstructor{z≤n}\AgdaSymbol{))))}\<%
\end{code}
The proofs embedded in the tree are not interesting and trivial to automate, but nonetheless
clutter the definition and make the overall structure harder to 
discern. This problem only grows worse as our definitions become more complex.
In \autoref{sec:core}, we introduce an imperative language embedded within Agda called \textsc{Trip},
where terms are typed by their correctness specifications. The proofs we must embed inside these 
terms grow to the hundreds of lines of proof for simple programs of approximately ten lines
of code. If these proofs were to be nested directly within the code, the 
overall program would become nearly unreadable.

We resolve this readability problem by introducing a \emph{proof delay}
applicative functor. This allows us to first sketch the big picture, 
and only fill in details afterwards. The applicative functor
collects any outstanding proof obligations as we sketch, and requires us 
to provide proofs of these goals before ultimately producing the actual data structure.


When applied to \textsc{Trip} programs, this framework becomes a 
\emph{verification condition generator}, allowing programs to be written 
without any proofs, all the while gathering the implications that must be discharged
in order to show correctness. 

In \autoref{sec:proofdelay}, we introduce the proof delay applicative functor in Agda, and provide 
some small examples of its use. In \autoref{sec:core}, we introduce the core of the \textsc{Trip}
language, give it semantics in terms of state-relations, and show soundness of the specifications
carried on the types of \textsc{Trip} terms. In \autoref{sec:surface}, we apply the proof delay 
applicative to \textsc{Trip} and create a readable surface syntax for the language using 
Agda's new macro system, and demonstrate our verification framework on a
number of examples. We discuss the design of both \textsc{Trip} and our proof delay 
applicative in \autoref{sec:discuss}, as well as examine related and future work. 

\section{The Proof Delay Applicative}\label{sec:proofdelay}
A computation of a type $X$ which may delay some obligations until later is written 
as $\AgdaDatatype{Delay}\ X$. The $\AgdaDatatype{Delay}$ type is defined as follows:
\begin{code}
\>[2]\AgdaKeyword{record}\AgdaSpace{}%
\AgdaRecord{Delay}\AgdaSpace{}%
\AgdaSymbol{(}\AgdaBound{X}\AgdaSpace{}%
\AgdaSymbol{:}\AgdaSpace{}%
\AgdaPrimitiveType{Set}\AgdaSpace{}%
\AgdaGeneralizable{ℓ}\AgdaSymbol{)}\AgdaSpace{}%
\AgdaSymbol{:}\AgdaSpace{}%
\AgdaPrimitiveType{Set}\AgdaSpace{}%
\AgdaSymbol{(}\AgdaPrimitive{Level.suc}\AgdaSpace{}%
\AgdaBound{ℓ}\AgdaSymbol{)}\AgdaSpace{}%
\AgdaKeyword{where}\<%
\\
\>[2][@{}l@{\AgdaIndent{0}}]%
\>[5]\AgdaKeyword{constructor}\AgdaSpace{}%
\AgdaInductiveConstructor{Prf}\<%
\\
%
\>[5]\AgdaKeyword{field}\<%
\\
\>[5][@{}l@{\AgdaIndent{0}}]%
\>[7]\AgdaField{goals}\AgdaSpace{}%
\AgdaSymbol{:}\AgdaSpace{}%
\AgdaDatatype{List}\AgdaSpace{}%
\AgdaPrimitiveType{Set}\<%
\\
%
\>[7]\AgdaField{prove}\AgdaSpace{}%
\AgdaSymbol{:}\AgdaSpace{}%
\AgdaDatatype{HList}\AgdaSpace{}%
\AgdaField{goals}\AgdaSpace{}%
\AgdaSymbol{→}\AgdaSpace{}%
\AgdaBound{X}\<%
\end{code}
The field $\AgdaField{goals}$ is a list of types, containing 
each of the propositions that must be proven in order to produce our 
result of type $X$. The $\AgdaField{prove}$ field is the actual computation
of the $X$ value that requires proof of each of the propositions
in $\AgdaField{goals}$.

The type $\AgdaDatatype{HList}$ is a heterogenous list~\citep{hlist}, indexed by a list 
of types corresponding to the type for each element:
\begin{code}
\>[2]\AgdaKeyword{data}\AgdaSpace{}%
\AgdaDatatype{HList}\AgdaSpace{}%
\AgdaSymbol{:}\AgdaSpace{}%
\AgdaDatatype{List}\AgdaSpace{}%
\AgdaPrimitiveType{Set}\AgdaSpace{}%
\AgdaSymbol{→}\AgdaSpace{}%
\AgdaPrimitiveType{Set}\AgdaSpace{}%
\AgdaKeyword{where}\<%
\\
\>[2][@{}l@{\AgdaIndent{0}}]%
\>[4]\AgdaInductiveConstructor{[]}\AgdaSpace{}%
\AgdaSymbol{:}\AgdaSpace{}%
\AgdaDatatype{HList}\AgdaSpace{}%
\AgdaInductiveConstructor{[]}\<%
\\
%
\>[4]\AgdaOperator{\AgdaInductiveConstructor{\AgdaUnderscore{}∷\AgdaUnderscore{}}}\AgdaSpace{}%
\AgdaSymbol{:}\AgdaSpace{}%
\AgdaSymbol{∀\{}\AgdaBound{S}\AgdaSymbol{\}\{}\AgdaBound{SS}\AgdaSymbol{\}}\AgdaSpace{}%
\AgdaSymbol{→}\AgdaSpace{}%
\AgdaBound{S}\AgdaSpace{}%
\AgdaSymbol{→}\AgdaSpace{}%
\AgdaDatatype{HList}\AgdaSpace{}%
\AgdaBound{SS}\AgdaSpace{}%
\AgdaSymbol{→}\AgdaSpace{}%
\AgdaDatatype{HList}\AgdaSpace{}%
\AgdaSymbol{(}\AgdaBound{S}\AgdaSpace{}%
\AgdaOperator{\AgdaInductiveConstructor{∷}}\AgdaSpace{}%
\AgdaBound{SS}\AgdaSymbol{)}\<%
\end{code}
\subsection{Construction}
There are two operations to construct a basic $\AgdaDatatype{Delay}$ computation.
The first is $\AgdaFunction{pure}$, which given a value, produces a $\AgdaDatatype{Delay}$
computation that returns that value without deferring any proofs until later:
\begin{code}
\>[2]\AgdaFunction{pure}\AgdaSpace{}%
\AgdaSymbol{:}\AgdaSpace{}%
\AgdaGeneralizable{X}\AgdaSpace{}%
\AgdaSymbol{→}\AgdaSpace{}%
\AgdaRecord{Delay}\AgdaSpace{}%
\AgdaGeneralizable{X}\<%
\\
%
\>[2]\AgdaFunction{pure}\AgdaSpace{}%
\AgdaBound{x}\AgdaSpace{}%
\AgdaSymbol{=}\AgdaSpace{}%
\AgdaInductiveConstructor{Prf}\AgdaSpace{}%
\AgdaInductiveConstructor{[]}\AgdaSpace{}%
\AgdaSymbol{(}\AgdaFunction{const}\AgdaSpace{}%
\AgdaBound{x}\AgdaSymbol{)}\<%
\end{code}
The second operation is $\AgdaFunction{later}$, which, as the name suggests, 
constructs a $\AgdaDatatype{Delay}\ X$ by requiring a value of type $X$ to be 
provided later: 
\begin{code}
\>[2]\AgdaFunction{later}\AgdaSpace{}%
\AgdaSymbol{:}\AgdaSpace{}%
\AgdaSymbol{∀\{}\AgdaBound{X}\AgdaSymbol{\}}\AgdaSpace{}%
\AgdaSymbol{→}\AgdaSpace{}%
\AgdaRecord{Delay}\AgdaSpace{}%
\AgdaBound{X}\<%
\\
%
\>[2]\AgdaFunction{later}\AgdaSpace{}%
\AgdaSymbol{\{}\AgdaBound{X}\AgdaSymbol{\}}\AgdaSpace{}%
\AgdaSymbol{=}\AgdaSpace{}%
\AgdaInductiveConstructor{Prf}\AgdaSpace{}%
\AgdaSymbol{(}\AgdaBound{X}\AgdaSpace{}%
\AgdaOperator{\AgdaInductiveConstructor{∷}}\AgdaSpace{}%
\AgdaInductiveConstructor{[]}\AgdaSymbol{)}\AgdaSpace{}%
\AgdaSymbol{λ}\AgdaSpace{}%
\AgdaSymbol{\{}\AgdaSpace{}%
\AgdaSymbol{(}\AgdaBound{x}\AgdaSpace{}%
\AgdaOperator{\AgdaInductiveConstructor{∷}}\AgdaSpace{}%
\AgdaInductiveConstructor{[]]}\AgdaSymbol{)}\AgdaSpace{}%
\AgdaSymbol{→}\AgdaSpace{}%
\AgdaBound{x}\AgdaSpace{}%
\AgdaSymbol{\}}\<%
\end{code}
\subsection{Composition}
We compose $\AgdaDatatype{Delay}$ computations using the application operator 
$\AgdaFunction{\ensuremath{\circledast}}$ which applies a $\AgdaDatatype{Delay}$
computation of a function to a $\AgdaDatatype{Delay}$ computation of its argument
by requiring the delayed obligations of \emph{both} computations to be satisfied before computing 
the function application:
\begin{code}
\>[2]\AgdaOperator{\AgdaFunction{\AgdaUnderscore{}\ensuremath{\circledast}\AgdaUnderscore{}}}\AgdaSpace{}%
\AgdaSymbol{:}%
\>[11]\AgdaRecord{Delay}\AgdaSpace{}%
\AgdaSymbol{(}\AgdaGeneralizable{A}\AgdaSpace{}%
\AgdaSymbol{→}\AgdaSpace{}%
\AgdaGeneralizable{B}\AgdaSymbol{)}\AgdaSpace{}%
\AgdaSymbol{→}\AgdaSpace{}%
\AgdaRecord{Delay}\AgdaSpace{}%
\AgdaGeneralizable{A}\AgdaSpace{}%
\AgdaSymbol{→}\AgdaSpace{}%
\AgdaRecord{Delay}\AgdaSpace{}%
\AgdaGeneralizable{B}\<%
\\
%
\>[2]\AgdaInductiveConstructor{Prf}\AgdaSpace{}%
\AgdaBound{goals₁}\AgdaSpace{}%
\AgdaBound{prove₁}\AgdaSpace{}%
\AgdaOperator{\AgdaFunction{\ensuremath{\circledast}}}\AgdaSpace{}%
\AgdaInductiveConstructor{Prf}\AgdaSpace{}%
\AgdaBound{goals₂}\AgdaSpace{}%
\AgdaBound{prove₂}\<%
\\
\>[2][@{}l@{\AgdaIndent{0}}]%
\>[4]\AgdaSymbol{=}\AgdaSpace{}%
\AgdaInductiveConstructor{Prf}%
\>[138I]\AgdaSymbol{(}\AgdaBound{goals₁}\AgdaSpace{}%
\AgdaOperator{\AgdaFunction{++}}\AgdaSpace{}%
\AgdaBound{goals₂}%
\AgdaSymbol{)}\<%
\\
\>[.][@{}l@{}]\<[138I]%
\>[10]\AgdaSymbol{λ}\AgdaSpace{}%
\AgdaBound{hl}\AgdaSpace{}%
\AgdaSymbol{→}\AgdaSpace{}%
\AgdaBound{prove₁}\AgdaSpace{}%
\AgdaSymbol{(}\AgdaFunction{takeH}\AgdaSpace{}%
\AgdaBound{hl}\AgdaSymbol{)}\AgdaSpace{}%
\AgdaSymbol{(}\AgdaBound{prove₂}\AgdaSpace{}%
\AgdaSymbol{(}\AgdaFunction{dropH}\AgdaSpace{}%
\AgdaBound{hl}\AgdaSymbol{))}\<%
\end{code}
Because we concatenate (using $\AgdaOperator{++}$) the goals of the function computation 
with the goals of the argument computation, when we actually discharge these obligations 
the incoming $\AgdaDatatype{HList}$ will be similarly concatenated. We use the functions 
$\AgdaFunction{takeH}$ and $\AgdaFunction{dropH}$ to break apart this $\AgdaDatatype{HList}$ in order to 
discharge the obligations of each sub-computation. While analogous to the 
conventional $\AgdaFunction{take}$ and $\AgdaFunction{drop}$ functions on
homogenous lists, the $\AgdaDatatype{HList}$ versions of these functions 
have only one explicit parameter, inferring the size of the two sublists from the
provided type indices:\nopagebreak
\begin{code}
\>[2]\AgdaFunction{takeH}\AgdaSpace{}%
\AgdaSymbol{:}\AgdaSpace{}%
\AgdaSymbol{∀\{}\AgdaBound{SS}\AgdaSpace{}%
\AgdaBound{TS}\AgdaSymbol{\}}\AgdaSpace{}%
\AgdaSymbol{→}\AgdaSpace{}%
\AgdaDatatype{HList}\AgdaSpace{}%
\AgdaSymbol{(}\AgdaBound{SS}\AgdaSpace{}%
\AgdaOperator{\AgdaFunction{++}}\AgdaSpace{}%
\AgdaBound{TS}\AgdaSymbol{)}\AgdaSpace{}%
\AgdaSymbol{→}\AgdaSpace{}%
\AgdaDatatype{HList}\AgdaSpace{}%
\AgdaBound{SS}\<%
\\
%
\>[2]\AgdaFunction{takeH}\AgdaSpace{}%
\AgdaSymbol{\{}\AgdaInductiveConstructor{[]}\AgdaSymbol{\}}\AgdaSpace{}%
\AgdaBound{ys}\AgdaSpace{}%
\AgdaSymbol{=}\AgdaSpace{}%
\AgdaInductiveConstructor{[]}\<%
\\
%
\>[2]\AgdaFunction{takeH}\AgdaSpace{}%
\AgdaSymbol{\{}\AgdaBound{S}\AgdaSpace{}%
\AgdaOperator{\AgdaInductiveConstructor{∷}}\AgdaSpace{}%
\AgdaBound{SS}\AgdaSymbol{\}}\AgdaSpace{}%
\AgdaSymbol{(}\AgdaBound{x}\AgdaSpace{}%
\AgdaOperator{\AgdaInductiveConstructor{∷}}\AgdaSpace{}%
\AgdaBound{xs}\AgdaSymbol{)}\AgdaSpace{}%
\AgdaSymbol{=}\AgdaSpace{}%
\AgdaBound{x}\AgdaSpace{}%
\AgdaOperator{\AgdaInductiveConstructor{∷}}\AgdaSpace{}%
\AgdaFunction{takeH}\AgdaSpace{}%
\AgdaSymbol{\{}\AgdaBound{SS}\AgdaSymbol{\}}\AgdaSpace{}%
\AgdaBound{xs}\<%
\\
%
\\[\AgdaEmptyExtraSkip]%
%
\>[2]\AgdaFunction{dropH}\AgdaSpace{}%
\AgdaSymbol{:}\AgdaSpace{}%
\AgdaSymbol{∀\{}\AgdaBound{SS}\AgdaSpace{}%
\AgdaBound{TS}\AgdaSymbol{\}}\AgdaSpace{}%
\AgdaSymbol{→}\AgdaSpace{}%
\AgdaDatatype{HList}\AgdaSpace{}%
\AgdaSymbol{(}\AgdaBound{SS}\AgdaSpace{}%
\AgdaOperator{\AgdaFunction{++}}\AgdaSpace{}%
\AgdaBound{TS}\AgdaSymbol{)}\AgdaSpace{}%
\AgdaSymbol{→}\AgdaSpace{}%
\AgdaDatatype{HList}\AgdaSpace{}%
\AgdaBound{TS}\<%
\\
%
\>[2]\AgdaFunction{dropH}\AgdaSpace{}%
\AgdaSymbol{\{}\AgdaInductiveConstructor{[]}\AgdaSymbol{\}}\AgdaSpace{}%
\AgdaBound{ys}\AgdaSpace{}%
\AgdaSymbol{=}\AgdaSpace{}%
\AgdaBound{ys}\<%
\\
%
\>[2]\AgdaFunction{dropH}\AgdaSpace{}%
\AgdaSymbol{\{}\AgdaBound{S}\AgdaSpace{}%
\AgdaOperator{\AgdaInductiveConstructor{∷}}\AgdaSpace{}%
\AgdaBound{SS}\AgdaSymbol{\}}\AgdaSpace{}%
\AgdaSymbol{(}\AgdaBound{x}\AgdaSpace{}%
\AgdaOperator{\AgdaInductiveConstructor{∷}}\AgdaSpace{}%
\AgdaBound{xs}\AgdaSymbol{)}\AgdaSpace{}%
\AgdaSymbol{=}\AgdaSpace{}%
\AgdaFunction{dropH}\AgdaSpace{}%
\AgdaSymbol{\{}\AgdaBound{SS}\AgdaSymbol{\}}\AgdaSpace{}%
\AgdaBound{xs}\<%
\end{code}
\subsection{Syntactic niceties}
The two operations $\AgdaOperator{\ensuremath{\circledast}}$ and $\AgdaFunction{pure}$ 
make the type $\AgdaDatatype{Delay}$ into an \emph{applicative functor}~\citep{applicatives}.
This allows us to use the \emph{idiom brackets} notation of \citet{applicatives}, as implemented in Agda 
version 2.6.0, to express $\AgdaDatatype{Delay}$ computations. These brackets denote nested $\AgdaDatatype{Delay}$ applications, for example $\AgdaFunction{pure}\ f\ \AgdaFunction{\ensuremath\circledast}\ a\ \AgdaFunction{\ensuremath\circledast}\ b\ \AgdaFunction{\ensuremath\circledast}\ c$, 
as pure function applications within banana brackets, i.e.\ $(\!\vert f\ a\ b\ c \vert\!)$.

We also define a synonym for the $\AgdaField{prove}$ field selector using Agda's mix-fix syntax, 
providing a clear syntactic delineation between overall structure and detailed proof:
\begin{code}
\>[2]\AgdaOperator{\AgdaFunction{structure:\AgdaUnderscore{}proofs:\AgdaUnderscore{}done}}\AgdaSpace{}%
\AgdaSymbol{=}\AgdaSpace{}%
\AgdaField{Delay.prove}\<%
\end{code}
\subsection{A non-monadic applicative functor}
Unlike many of the applicative functors in common use in functional programming, 
our $\AgdaDatatype{Delay}$ functor is \emph{not} a monad. This becomes apparent 
when we examine the type of a hypothetical monadic bind operation for \AgdaDatatype{Delay}:
$$
\AgdaFunction{\AgdaUnderscore{}\ensuremath{>\!\!>\!=}\AgdaUnderscore{}} : \AgdaDatatype{Delay}\ A \rightarrow (A \rightarrow \AgdaDatatype{Delay}\ B) \rightarrow \AgdaDatatype{Delay}\ B
$$
Like $\AgdaFunction{\ensuremath{\circledast}}$, the $\AgdaFunction{\ensuremath{>\!\!>\!=}}$ operator gives us a way to compose 
two $\AgdaDatatype{Delay}$ computations, but 
unlike $\AgdaFunction{\ensuremath{\circledast}}$, the second computation is
 \emph{dependent} on the results of the first. 
 This means that if our $\AgdaDatatype{Delay}$ type were a monad, we could not
 determine all the deferred obligations of a computation at once, 
 as some of these obligations would only be deferred \emph{after} proofs for other 
 goals had been provided.  

 As our \AgdaDatatype{HList} type is an $n$-ary generalisation of a simple \emph{non-dependent} 
 product type, we have no way to express such a dependency, and thus \AgdaDatatype{Delay}
 is not a monad. We can remedy this by defining our \AgdaDatatype{Delay} type more generally, using a single type for its deferred obligations, rather 
 than a list of types:
\begin{code}
\>[2]\AgdaKeyword{record}\AgdaSpace{}%
\AgdaRecord{MDelay}\AgdaSpace{}%
\AgdaSymbol{(}\AgdaBound{X}\AgdaSpace{}%
\AgdaSymbol{:}\AgdaSpace{}%
\AgdaPrimitiveType{Set}\AgdaSpace{}%
\AgdaGeneralizable{ℓ}\AgdaSymbol{)}\AgdaSpace{}%
\AgdaSymbol{:}\AgdaSpace{}%
\AgdaPrimitiveType{Set}\AgdaSpace{}%
\AgdaSymbol{(}\AgdaPrimitive{Level.suc}\AgdaSpace{}%
\AgdaBound{ℓ}\AgdaSymbol{)}\AgdaSpace{}%
\AgdaKeyword{where}\<%
\\
\>[2][@{}l@{\AgdaIndent{0}}]%
\>[5]\AgdaKeyword{constructor}\AgdaSpace{}%
\AgdaInductiveConstructor{MPrf}\<%
\\
%
\>[5]\AgdaKeyword{field}\<%
\\
\>[5][@{}l@{\AgdaIndent{0}}]%
\>[7]\AgdaField{goals}\AgdaSpace{}%
\AgdaSymbol{:}\AgdaSpace{}%
\AgdaPrimitiveType{Set}\<%
\\
%
\>[7]\AgdaField{prove}\AgdaSpace{}%
\AgdaSymbol{:}\AgdaSpace{}%
\AgdaField{goals}\AgdaSpace{}%
\AgdaSymbol{→}\AgdaSpace{}%
\AgdaBound{X}\<%
\end{code}
The application operation now combines the goals of the two subcomputations with a simple product type:

\begin{code}
\>[2]\AgdaOperator{\AgdaFunction{\AgdaUnderscore{}\ensuremath{\circledast}\AgdaUnderscore{}}}\AgdaSpace{}%
\AgdaSymbol{:}%
\>[11]\AgdaRecord{MDelay}\AgdaSpace{}%
\AgdaSymbol{(}\AgdaGeneralizable{A}\AgdaSpace{}%
\AgdaSymbol{→}\AgdaSpace{}%
\AgdaGeneralizable{B}\AgdaSymbol{)}\AgdaSpace{}%
\AgdaSymbol{→}\AgdaSpace{}%
\AgdaRecord{MDelay}\AgdaSpace{}%
\AgdaGeneralizable{A}\AgdaSpace{}%
\AgdaSymbol{→}\AgdaSpace{}%
\AgdaRecord{MDelay}\AgdaSpace{}%
\AgdaGeneralizable{B}\<%
\\
%
\>[2]\AgdaBound{d₁}\AgdaSpace{}%
\AgdaOperator{\AgdaFunction{\ensuremath{\circledast}}}\AgdaSpace{}%
\AgdaBound{d₂}\AgdaSpace{}%
\AgdaSymbol{=}%
\>[51I]\AgdaInductiveConstructor{MPrf}\AgdaSpace{}%
\AgdaSymbol{(}\AgdaField{goals}\AgdaSpace{}%
\AgdaBound{d₁}\AgdaSpace{}%
\AgdaOperator{\AgdaFunction{×}}\AgdaSpace{}%
\AgdaField{goals}\AgdaSpace{}%
\AgdaBound{d₂}\AgdaSymbol{)}\<%
\\
\>[51I][@{}l@{\AgdaIndent{0}}]%
\>[16]\AgdaSymbol{λ}\AgdaSpace{}%
\AgdaSymbol{\{}\AgdaSpace{}%
\AgdaSymbol{(}\AgdaBound{p₁}\AgdaSpace{}%
\AgdaOperator{\AgdaInductiveConstructor{,}}\AgdaSpace{}%
\AgdaBound{p₂}\AgdaSpace{}%
\AgdaSymbol{)}\AgdaSpace{}%
\AgdaSymbol{→}\AgdaSpace{}%
\AgdaField{prove}\AgdaSpace{}%
\AgdaBound{d₁}\AgdaSpace{}%
\AgdaBound{p₁}\AgdaSpace{}%
\AgdaSymbol{(}\AgdaField{prove}\AgdaSpace{}%
\AgdaBound{d₂}\AgdaSpace{}%
\AgdaBound{p₂}\AgdaSymbol{)}\AgdaSpace{}%
\AgdaSymbol{\}}\<%
\end{code}
 Furthermore, the monadic \AgdaDatatype{join} operator can now be defined using a $\Sigma$-type to express the telescopic dependency between goals:

 \begin{code}
  \>[2]\AgdaFunction{join}\AgdaSpace{}%
  \AgdaSymbol{:}\AgdaSpace{}%
  \AgdaRecord{MDelay}\AgdaSpace{}%
  \AgdaSymbol{(}\AgdaRecord{MDelay}\AgdaSpace{}%
  \AgdaGeneralizable{A}\AgdaSymbol{)}\AgdaSpace{}%
  \AgdaSymbol{→}\AgdaSpace{}%
  \AgdaRecord{MDelay}\AgdaSpace{}%
  \AgdaGeneralizable{A}\<%
  \\
  %
  \>[2]\AgdaFunction{join}\AgdaSpace{}%
  \AgdaBound{d}\AgdaSpace{}%
  \AgdaSymbol{=}%
  \>[79I]\AgdaInductiveConstructor{MPrf}\AgdaSpace{}%
  \AgdaSymbol{(}\AgdaRecord{Σ}\AgdaSpace{}%
  \AgdaSymbol{(}\AgdaField{goals}\AgdaSpace{}%
  \AgdaBound{d}\AgdaSymbol{)}\AgdaSpace{}%
  \AgdaSymbol{(}\AgdaField{goals}\AgdaSpace{}%
  \AgdaOperator{\AgdaFunction{∘}}\AgdaSpace{}%
  \AgdaField{prove}\AgdaSpace{}%
  \AgdaBound{d}\AgdaSymbol{))}\<%
  \\
  \>[79I][@{}l@{\AgdaIndent{0}}]%
  \>[12]\AgdaSymbol{λ}\AgdaSpace{}%
  \AgdaSymbol{\{}\AgdaSpace{}%
  \AgdaSymbol{(}\AgdaBound{g}\AgdaSpace{}%
  \AgdaOperator{\AgdaInductiveConstructor{,}}\AgdaSpace{}%
  \AgdaBound{g$'$}\AgdaSymbol{)}\AgdaSpace{}%
  \AgdaSymbol{→}%
  \>[28]\AgdaField{prove}\AgdaSpace{}%
  \AgdaSymbol{(}\AgdaField{prove}\AgdaSpace{}%
  \AgdaBound{d}\AgdaSpace{}%
  \AgdaBound{g}\AgdaSymbol{)}\AgdaSpace{}%
  \AgdaBound{g$'$}\AgdaSpace{}%
  \AgdaSymbol{\}}\<%
  \end{code}
This definition is slightly simpler than that of \AgdaDatatype{Delay}, but it results in the type of deferred obligations 
forming a tree structure that resembles the structure of the outline we have already written. We have to describe 
our structure twice: once in the outline, and once in the proofs.  Furthermore, we are repeating ourselves for little to 
no apparent benefit, as our \AgdaDatatype{Delay} type is sufficient for even the most sophisticated examples in this paper.

The greater static knowledge afforded by applicative functors has lead to them being preferred 
over monads in many domains for efficiency reasons~\citep{tyde16,haxl,haxl2,applicativedo}. In our case, however,
we prefer applicative functors to maintain a clean separation of concerns 
between structure and proof.

\subsection{Small examples}
Recalling the ordered list data type introduced in \autoref{sec:intro}, we now wrap each constructor 
in the $\AgdaDatatype{Delay}$ applicative, deferring any ordering proofs until later:
\begin{code}
\>[2]\AgdaFunction{nil}\AgdaSpace{}%
\AgdaSymbol{:}\AgdaSpace{}%
\AgdaRecord{Delay}\AgdaSpace{}%
\AgdaSymbol{(}\AgdaDatatype{OList}\AgdaSpace{}%
\AgdaGeneralizable{m}\AgdaSpace{}%
\AgdaGeneralizable{n}\AgdaSymbol{)}\<%
\\
%
\>[2]\AgdaFunction{nil}\AgdaSpace{}%
\AgdaSymbol{=}\AgdaSpace{}%
\AgdaSymbol{⦇}\AgdaSpace{}%
\AgdaInductiveConstructor{Nil}\AgdaSpace{}%
\AgdaFunction{later}\AgdaSpace{}%
\AgdaSymbol{⦈}\<%
%
\end{code}
\begin{code}
\>[2]\AgdaOperator{\AgdaFunction{\AgdaUnderscore{}cons\AgdaUnderscore{}}}\AgdaSpace{}%
\AgdaSymbol{:}\AgdaSpace{}%
\AgdaSymbol{(}\AgdaBound{x}\AgdaSpace{}%
\AgdaSymbol{:}\AgdaSpace{}%
\AgdaDatatype{ℕ}\AgdaSymbol{)}\AgdaSpace{}%
\AgdaSymbol{→}\AgdaSpace{}%
\AgdaRecord{Delay}\AgdaSpace{}%
\AgdaSymbol{(}\AgdaDatatype{OList}\AgdaSpace{}%
\AgdaBound{x}\AgdaSpace{}%
\AgdaGeneralizable{n}\AgdaSymbol{)}\AgdaSpace{}%
\AgdaSymbol{→}\AgdaSpace{}%
\AgdaRecord{Delay}\AgdaSpace{}%
\AgdaSymbol{(}\AgdaDatatype{OList}\AgdaSpace{}%
\AgdaGeneralizable{m}\AgdaSpace{}%
\AgdaGeneralizable{n}\AgdaSymbol{)}\<%
\\
%
\>[2]\AgdaBound{x}\AgdaSpace{}%
\AgdaOperator{\AgdaFunction{cons}}\AgdaSpace{}%
\AgdaBound{xs}\AgdaSpace{}%
\AgdaSymbol{=}\AgdaSpace{}%
\AgdaSymbol{⦇}\AgdaSpace{}%
\AgdaSymbol{(}\AgdaInductiveConstructor{Cons}\AgdaSpace{}%
\AgdaBound{x}\AgdaSymbol{)}\AgdaSpace{}%
\AgdaFunction{later}\AgdaSpace{}%
\AgdaBound{xs}\AgdaSpace{}%
\AgdaSymbol{⦈}\<%
\end{code}
These constructors allow ordered lists to be constructed just as any other list, albeit 
within the proof delay applicative. To extract the final $\AgdaDatatype{OList}$ value, we must 
provide a $\AgdaDatatype{HList}$ of ordering proofs:
\begin{code}
\>[2]\AgdaFunction{example}\AgdaSpace{}%
\AgdaSymbol{:}\AgdaSpace{}%
\AgdaDatatype{OList}\AgdaSpace{}%
\AgdaNumber{1}\AgdaSpace{}%
\AgdaNumber{5}\<%
\\
%
\>[2]\AgdaFunction{example}\AgdaSpace{}%
\AgdaSymbol{=}%
\>[984I]\AgdaOperator{\AgdaFunction{structure:}}\AgdaSpace{}%
\AgdaNumber{1}\AgdaSpace{}%
\AgdaOperator{\AgdaFunction{cons}}\AgdaSpace{}%
\AgdaNumber{2}\AgdaSpace{}%
\AgdaOperator{\AgdaFunction{cons}}\AgdaSpace{}%
\AgdaNumber{3}\AgdaSpace{}%
\AgdaOperator{\AgdaFunction{cons}}\AgdaSpace{}%
\AgdaNumber{4}\AgdaSpace{}%
\AgdaOperator{\AgdaFunction{cons}}\AgdaSpace{}%
\AgdaNumber{5}\AgdaSpace{}%
\AgdaOperator{\AgdaFunction{cons}}\AgdaSpace{}%
\AgdaFunction{nil}\<%
\\
\>[.][@{}l@{}]\<[984I]%
\>[13]\AgdaOperator{\AgdaFunction{proofs:}}\<%
\\
\>[13][@{}l@{\AgdaIndent{0}}]%
\>[15]\AgdaInductiveConstructor{s≤s}\AgdaSpace{}%
\AgdaInductiveConstructor{z≤n}\<%
\\
%
\>[13]\AgdaOperator{\AgdaInductiveConstructor{∷}}\AgdaSpace{}%
\AgdaInductiveConstructor{s≤s}\AgdaSpace{}%
\AgdaInductiveConstructor{z≤n}\<%
\\
%
\>[13]\AgdaOperator{\AgdaInductiveConstructor{∷}}\AgdaSpace{}%
\AgdaInductiveConstructor{s≤s}\AgdaSpace{}%
\AgdaSymbol{(}\AgdaInductiveConstructor{s≤s}\AgdaSpace{}%
\AgdaInductiveConstructor{z≤n}\AgdaSymbol{)}\<%
\\
%
\>[13]\AgdaOperator{\AgdaInductiveConstructor{∷}}\AgdaSpace{}%
\AgdaInductiveConstructor{s≤s}\AgdaSpace{}%
\AgdaSymbol{(}\AgdaInductiveConstructor{s≤s}\AgdaSpace{}%
\AgdaSymbol{(}\AgdaInductiveConstructor{s≤s}\AgdaSpace{}%
\AgdaInductiveConstructor{z≤n}\AgdaSymbol{))}\<%
\\
%
\>[13]\AgdaOperator{\AgdaInductiveConstructor{∷}}\AgdaSpace{}%
\AgdaInductiveConstructor{s≤s}\AgdaSpace{}%
\AgdaSymbol{(}\AgdaInductiveConstructor{s≤s}\AgdaSpace{}%
\AgdaSymbol{(}\AgdaInductiveConstructor{s≤s}\AgdaSpace{}%
\AgdaSymbol{(}\AgdaInductiveConstructor{s≤s}\AgdaSpace{}%
\AgdaInductiveConstructor{z≤n}\AgdaSymbol{)))}\<%
\\
%
\>[13]\AgdaOperator{\AgdaInductiveConstructor{∷}}\AgdaSpace{}%
\AgdaInductiveConstructor{s≤s}\AgdaSpace{}%
\AgdaSymbol{(}\AgdaInductiveConstructor{s≤s}\AgdaSpace{}%
\AgdaSymbol{(}\AgdaInductiveConstructor{s≤s}\AgdaSpace{}%
\AgdaSymbol{(}\AgdaInductiveConstructor{s≤s}\AgdaSpace{}%
\AgdaSymbol{(}\AgdaInductiveConstructor{s≤s}\AgdaSpace{}%
\AgdaInductiveConstructor{z≤n}\AgdaSymbol{))))}\<%
\\
%
\>[13]\AgdaOperator{\AgdaInductiveConstructor{∷}}\AgdaSpace{}%
\AgdaInductiveConstructor{[]}\<%
\\
%
\>[13]\AgdaOperator{\AgdaFunction{done}}\<%
%
\end{code}
Our binary search tree example is similar,
wrapping the constructors in the $\AgdaDatatype{Delay}$ applicative
and deferring the ordering proofs 
on $\AgdaInductiveConstructor{Leaf}$ nodes until $\AgdaFunction{later}$:
\begin{code}
\>[2]\AgdaFunction{leaf}\AgdaSpace{}%
\AgdaSymbol{:}\AgdaSpace{}%
\AgdaRecord{Delay}\AgdaSpace{}%
\AgdaSymbol{(}\AgdaDatatype{BST}\AgdaSpace{}%
\AgdaGeneralizable{m}\AgdaSpace{}%
\AgdaGeneralizable{n}\AgdaSymbol{)}\<%
\\
%
\>[2]\AgdaFunction{leaf}\AgdaSpace{}%
\AgdaSymbol{=}\AgdaSpace{}%
\AgdaSymbol{⦇}\AgdaSpace{}%
\AgdaInductiveConstructor{Leaf}\AgdaSpace{}%
\AgdaFunction{later}\AgdaSpace{}%
\AgdaSymbol{⦈}\<%
\\
\\[\AgdaEmptyExtraSkip]%
%
\>[2]\AgdaFunction{branch}%
\>[800I]\AgdaSymbol{:}\AgdaSpace{}%
\AgdaSymbol{(}\AgdaBound{x}\AgdaSpace{}%
\AgdaSymbol{:}\AgdaSpace{}%
\AgdaDatatype{ℕ}\AgdaSymbol{)}\AgdaSpace{}%
\AgdaSymbol{→}\AgdaSpace{}%
\AgdaRecord{Delay}\AgdaSpace{}%
\AgdaSymbol{(}\AgdaDatatype{BST}\AgdaSpace{}%
\AgdaGeneralizable{m}\AgdaSpace{}%
\AgdaBound{x}\AgdaSymbol{)}\AgdaSpace{}%
\AgdaSymbol{→}\AgdaSpace{}%
\AgdaRecord{Delay}\AgdaSpace{}%
\AgdaSymbol{(}\AgdaDatatype{BST}\AgdaSpace{}%
\AgdaBound{x}\AgdaSpace{}%
\AgdaGeneralizable{n}\AgdaSymbol{)}\<%
\\
\>[.][@{}l@{}]\<[800I]%
\>[9]\AgdaSymbol{→}\AgdaSpace{}%
\AgdaRecord{Delay}\AgdaSpace{}%
\AgdaSymbol{(}\AgdaDatatype{BST}\AgdaSpace{}%
\AgdaGeneralizable{m}\AgdaSpace{}%
\AgdaGeneralizable{n}\AgdaSymbol{)}\<%
\\
%
\>[2]\AgdaFunction{branch}\AgdaSpace{}%
\AgdaBound{x}\AgdaSpace{}%
\AgdaBound{l}\AgdaSpace{}%
\AgdaBound{r}\AgdaSpace{}%
\AgdaSymbol{=}\AgdaSpace{}%
\AgdaSymbol{⦇}\AgdaSpace{}%
\AgdaSymbol{(}\AgdaInductiveConstructor{Branch}\AgdaSpace{}%
\AgdaBound{x}\AgdaSymbol{)}\AgdaSpace{}%
\AgdaBound{l}\AgdaSpace{}%
\AgdaBound{r}\AgdaSpace{}%
\AgdaSymbol{⦈}\<%
%
\end{code}
With this example we can begin to glimpse the usefulness of the 
\AgdaDatatype{Delay} applicative for describing proof-carrying data structures.
The tree structure of the data is visually apparent, and the uninteresting proofs are 
relegated to a separate section of code, where they can be conveniently omitted from 
this paper in the interest of brevity:
\begin{code}
\>[2]\AgdaFunction{example₂}\AgdaSpace{}%
\AgdaSymbol{:}\AgdaSpace{}%
\AgdaDatatype{BST}\AgdaSpace{}%
\AgdaNumber{2}\AgdaSpace{}%
\AgdaNumber{10}\<%
\\
%
\>[2]\AgdaFunction{example₂}\AgdaSpace{}%
\AgdaSymbol{=}%
\>[861I]\AgdaOperator{\AgdaFunction{structure:}}%
\>[862I]\AgdaFunction{branch}\AgdaSpace{}%
\AgdaNumber{3}\<%
\\
\>[862I][@{}l@{\AgdaIndent{0}}]%
\>[25]\AgdaSymbol{(}\AgdaFunction{branch}\AgdaSpace{}%
\AgdaNumber{2}\AgdaSpace{}%
\AgdaFunction{leaf}\AgdaSpace{}%
\AgdaFunction{leaf}\AgdaSymbol{)}\<%
\\
%
\>[25]\AgdaSymbol{(}\AgdaFunction{branch}\AgdaSpace{}%
\AgdaNumber{5}\<%
\\
\>[25][@{}l@{\AgdaIndent{0}}]%
\>[28]\AgdaSymbol{(}\AgdaFunction{branch}\AgdaSpace{}%
\AgdaNumber{4}\AgdaSpace{}%
\AgdaFunction{leaf}\AgdaSpace{}%
\AgdaFunction{leaf}\AgdaSymbol{)}\<%
\\
%
\>[28]\AgdaSymbol{(}\AgdaFunction{branch}\AgdaSpace{}%
\AgdaNumber{10}\AgdaSpace{}%
\AgdaFunction{leaf}\AgdaSpace{}%
\AgdaFunction{leaf}\AgdaSymbol{))}\<%
\\
\>[.][@{}l@{}]\<[861I]%
\>[13]\AgdaOperator{\AgdaFunction{proofs:}} \ensuremath{\langle \textit{omitted for brevity} \rangle}\<%
\\
%
\>[13]\AgdaOperator{\AgdaFunction{done}}\<%
\end{code}
Ordering proofs are easily decidable 
and therefore not difficult to automate, either using 
Agda's proof search features~\citep{agsy} or by
exploiting computation within the Agda type checking process~\citep{tyde16}. 
Initial experiments show that these techniques are highly compatible 
with our approach, discharging all deferred ordering obligations entirely automatically.

\section{The Core of \textsc{Trip}}\label{sec:core}
\begin{figure}
  $$
  \begin{array}{llclr}
   \text{programs}& \mathcal{P},\mathcal{Q} & ::= & \mathcal{P} ; \mathcal{Q} & \text{(seq. composition)} \\ 
                  &          &  |  & \mathcal{P} + \mathcal{Q} & \text{(nondet. choice)} \\
                  &          &  |  & \mathcal{P}^\star & \text{(Kleene star)} \\
                  &          &  |  & g & \text{(guard)} \\
                  &          &  |  & \mathcal{U} & \text{(state update)} \\
    \text{updates} & \mathcal{U} & \in & \Sigma \rightarrow \Sigma \\ 
    \text{assertions} & \varphi,\psi, \alpha, g  & \in & \Sigma \rightarrow \mathbb{B} \\ 
    \text{states} & \Sigma \\
    \text{booleans} & \mathbb{B}
  \end{array}
  $$
  \caption{The language of Regular Imperative Programs}
  \label{fig:rip}
\end{figure}
A very similar distinction between structure and proof can be found in software verification, 
particularly when verifying imperative programs using program logics such as those of 
\citet{floyd} or \citet{hoare}. Typically, we would write out our program in full, 
interspersed with local \emph{assertions} about the state. By using the axioms of the program logic, 
we then derive a set of \emph{verification conditions}, i.e.\ logical formulae that together imply that 
our assertions hold for all executions of our program. The verification is completed 
by discharging each of the verification conditions by proof.

\subsection{Regular Imperative Programs}\label{subsec:rip}
\begin{figure*}
  \begin{tabular}{p{5cm}p{5cm}p{3cm}}
    \centering\inferrule[Seq]{\trip{\varphi}{\pee}{\alpha}\\ \trip{\alpha}{\quu}{\psi} }{\trip{\varphi}{\pee ; \quu}{\psi} }
 & 
    \centering\inferrule[Choice]{\trip{\varphi}{\pee}{\psi}\\ \trip{\varphi}{\quu}{\psi} }{\trip{\varphi}{\pee + \quu}{\psi} }\qquad
 & 
    \centering\inferrule[Star]{\trip{\varphi}{\pee}{\varphi}}{\trip{\varphi}{\pee^\star}{\varphi} }
  \end{tabular}
  \begin{tabular}{p{3cm}p{3cm}p{6cm}}
  \inferrule[Guard]{ }{ \trip{ g \rightarrow \varphi }{ g }{ \varphi } }
  &
  \inferrule[Update]{ }{ \trip{ \varphi \circ \mathcal{U}}{ \mathcal{U} }{\mathcal{\varphi}} }
  &
  \inferrule[Consequence]{ \varphi \rightarrow \varphi' \\ \trip{\varphi'}{ \pee}{\psi'} \\ \psi' \rightarrow \psi }{ \trip{\varphi}{\pee}{\psi}}
  \end{tabular}
  \caption{The Hoare logic rules for regular imperative programs.}
  \label{fig:hoare}
\end{figure*}
Before verifying imperative programs we must first define an imperative language in which 
to write them. We will base our definitions on the language of regular imperative programs,
so named for its resemblance to regular expressions, presented in Figure~\ref{fig:rip}. 
Expressing imperative programs as a Kleene algebra in this way is not new, but the exact origins 
of this language are unclear, emerging from folklore in the Netherlands and the US
during the mid-1970s~\cite{rip1,rip2,rip3,rip4}. The semantics of this language are given 
in terms of binary \emph{relations} on states. For a given program $\mathcal{P}$, 
we say $(\sigma_1, \sigma_2) \in \llbracket \mathcal{P} \rrbracket$ iff the state $\sigma_2$ 
could result from executing $\mathcal{P}$ in state $\sigma_1$. The semantics are given
as relations, rather than functions, because we allow our programs to be non-deterministic, with 
the choice operator being written as $\mathcal{P} + \mathcal{Q}$. This non-determinism can 
be constrained by using guard statements ($g$), which only execute when the guard $g$ is satisfied. 
For example, traditional $\textbf{if}$ statements can be recovered using this translation:
$$
\begin{array}{lcl}
\textbf{if}\ g\ \textbf{then}\ \mathcal{P}\ \textbf{else}\ \mathcal{Q}\ \textbf{fi} & \simeq & (g;\mathcal{P}) + (\neg g; \mathcal{Q}) 
\end{array}
$$
Similarly, standard loop constructs such as \textbf{while} can be encoded using the Kleene star $\mathcal{P}^\star$, which runs the program $\mathcal{P}$ a non-deterministic
number of times:
$$
\begin{array}{lcl}
\textbf{while}\ g\ \textbf{do}\ \mathcal{P}\ \textbf{od} & \simeq & (g; \mathcal{P})^\star; \neg g
\end{array}
$$
The language is parametric in the definition of states ($\Sigma$). For the verification of small programs or
specific algorithms, states are usually defined to be the mapping of all variable names to their values, but 
for large-scale software verification projects, this definition could be expanded to include models of hardware 
or heap memory~\citep{cmem}.

\subsection{Typed Regular Imperative Programs}\label{subsec:trip}

The core of our language \textsc{Trip} is an Agda encoding of the language in Figure~\ref{fig:rip}, where
program terms are typed by their Hoare logic specifications.

Our development is similarly parameterised by the type used to represent states. We define an
\AgdaDatatype{Assertion} to be a state-dependent proposition:
\begin{code}
\>[2]\AgdaFunction{Assertion}\AgdaSpace{}%
\AgdaSymbol{=}\AgdaSpace{}%
\AgdaSymbol{⦃}\AgdaSpace{}%
\AgdaBound{σ}\AgdaSpace{}%
\AgdaSymbol{:}\AgdaSpace{}%
\AgdaBound{State}\AgdaSpace{}%
\AgdaSymbol{⦄}\AgdaSpace{}%
\AgdaSymbol{→}\AgdaSpace{}%
\AgdaPrimitiveType{Set}\<%
\end{code}
We use Agda's \emph{instance arguments}~\citep{instanceargs} for the state parameter
so that assertions can be written as simple logical formulae such as $\AgdaField{n} > 0$ rather than 
less readable lambda abstractions like $(\lambda\ \sigma \rightarrow \AgdaField{n}\ \sigma > 0)$.

The type of \textsc{Trip} programs consists of a pair of \AgdaDatatype{Assertion}s, 
denoting its the pre- and post-conditions respectively. If a term $P$ has type $\AgdaFunction{[}\ \varphi\ \AgdaFunction{,}\ \psi\ \AgdaFunction{]}$,
that means that all executions of $P$ from a state where $\varphi$ holds will have 
a final state that satisfies $\psi$. This means that a typing judgement $P :\AgdaFunction{[}\ \varphi\ \AgdaFunction{,}\ \psi\ \AgdaFunction{]}$ is 
equivalent to a Hoare triple $\trip{\varphi}{P}{\psi}$. We prove this theorem with respect to the 
relational semantics of \textsc{Trip} in \autoref{subsec:soundness}.
\begin{code}
\>[2]\AgdaKeyword{data}\AgdaSpace{}%
\AgdaOperator{\AgdaDatatype{[\AgdaUnderscore{},\AgdaUnderscore{}]}}\AgdaSpace{}%
\AgdaSymbol{:}\AgdaSpace{}%
\AgdaFunction{Assertion}\AgdaSpace{}%
\AgdaSymbol{→}\AgdaSpace{}%
\AgdaFunction{Assertion}\AgdaSpace{}%
\AgdaSymbol{→}\AgdaSpace{}%
\AgdaPrimitiveType{Set₁}\AgdaSpace{}%
\AgdaKeyword{where}\<%
\end{code}
The types of our terms are based on the Hoare logic rules for regular imperative programs,
given in \autoref{fig:hoare}. For sequential composition, we have an intermediate assertion $\alpha$ which 
is the post-condition of the first program and the pre-condition of the second:
\begin{code}
\>[2][@{}l@{\AgdaIndent{0}}]%
\>[4]\AgdaInductiveConstructor{SEQ}%
\>[10]\AgdaSymbol{:}\AgdaSpace{}%
\AgdaOperator{\AgdaDatatype{[}}\AgdaSpace{}%
\AgdaGeneralizable{ϕ}\AgdaSpace{}%
\AgdaOperator{\AgdaDatatype{,}}\AgdaSpace{}%
\AgdaGeneralizable{α}\AgdaSpace{}%
\AgdaOperator{\AgdaDatatype{]}}\AgdaSpace{}%
\AgdaSymbol{→}\AgdaSpace{}%
\AgdaOperator{\AgdaDatatype{[}}\AgdaSpace{}%
\AgdaGeneralizable{α}\AgdaSpace{}%
\AgdaOperator{\AgdaDatatype{,}}\AgdaSpace{}%
\AgdaGeneralizable{ψ}\AgdaSpace{}%
\AgdaOperator{\AgdaDatatype{]}}\AgdaSpace{}%
\AgdaSymbol{→}\AgdaSpace{}%
\AgdaOperator{\AgdaDatatype{[}}\AgdaSpace{}%
\AgdaGeneralizable{ϕ}\AgdaSpace{}%
\AgdaOperator{\AgdaDatatype{,}}\AgdaSpace{}%
\AgdaGeneralizable{ψ}\AgdaSpace{}%
\AgdaOperator{\AgdaDatatype{]}}\<%
\end{code}
Non-deterministic choice requires the two operands to have the same specification:
\begin{code}
\>[2][@{}l@{\AgdaIndent{0}}]%
\>[4]\AgdaInductiveConstructor{CHO}%
\>[10]\AgdaSymbol{:}\AgdaSpace{}%
\AgdaOperator{\AgdaDatatype{[}}\AgdaSpace{}%
\AgdaGeneralizable{ϕ}\AgdaSpace{}%
\AgdaOperator{\AgdaDatatype{,}}\AgdaSpace{}%
\AgdaGeneralizable{ψ}\AgdaSpace{}%
\AgdaOperator{\AgdaDatatype{]}}\AgdaSpace{}%
\AgdaSymbol{→}\AgdaSpace{}%
\AgdaOperator{\AgdaDatatype{[}}\AgdaSpace{}%
\AgdaGeneralizable{ϕ}\AgdaSpace{}%
\AgdaOperator{\AgdaDatatype{,}}\AgdaSpace{}%
\AgdaGeneralizable{ψ}\AgdaSpace{}%
\AgdaOperator{\AgdaDatatype{]}}\AgdaSpace{}%
\AgdaSymbol{→}\AgdaSpace{}%
\AgdaOperator{\AgdaDatatype{[}}\AgdaSpace{}%
\AgdaGeneralizable{ϕ}\AgdaSpace{}%
\AgdaOperator{\AgdaDatatype{,}}\AgdaSpace{}%
\AgdaGeneralizable{ψ}\AgdaSpace{}%
\AgdaOperator{\AgdaDatatype{]}}\<%
\end{code}
Because the Kleene star can run the given program any number of times,
including zero, it must maintain the same assertion (the \emph{loop invariant}) 
throughout:
\begin{code}
\>[2][@{}l@{\AgdaIndent{0}}]%
\>[4]\AgdaInductiveConstructor{STAR}%
\>[10]\AgdaSymbol{:}\AgdaSpace{}%
\AgdaOperator{\AgdaDatatype{[}}\AgdaSpace{}%
\AgdaGeneralizable{ϕ}\AgdaSpace{}%
\AgdaOperator{\AgdaDatatype{,}}\AgdaSpace{}%
\AgdaGeneralizable{ϕ}\AgdaSpace{}%
\AgdaOperator{\AgdaDatatype{]}}\AgdaSpace{}%
\AgdaSymbol{→}\AgdaSpace{}%
\AgdaOperator{\AgdaDatatype{[}}\AgdaSpace{}%
\AgdaGeneralizable{ϕ}\AgdaSpace{}%
\AgdaOperator{\AgdaDatatype{,}}\AgdaSpace{}%
\AgdaGeneralizable{ϕ}\AgdaSpace{}%
\AgdaOperator{\AgdaDatatype{]}}\<%
\end{code}
Guard statements only successfully execute when the given guard holds. Therefore, 
any assertion that follows from the guard in the pre-condition is 
a valid post-condition.
\begin{code}
\>[2][@{}l@{\AgdaIndent{0}}]%
\>[4]\AgdaInductiveConstructor{GUARD}\AgdaSpace{}%
\AgdaSymbol{:}\AgdaSpace{}%
\AgdaSymbol{(}\AgdaBound{g}\AgdaSpace{}%
\AgdaSymbol{:}\AgdaSpace{}%
\AgdaFunction{Assertion}\AgdaSymbol{)}\AgdaSpace{}%
\AgdaSymbol{→}\AgdaSpace{}%
\AgdaOperator{\AgdaDatatype{[}}\AgdaSpace{}%
\AgdaSymbol{(}\AgdaBound{g}\AgdaSpace{}%
\AgdaSymbol{→}\AgdaSpace{}%
\AgdaGeneralizable{ϕ}\AgdaSymbol{)}\AgdaSpace{}%
\AgdaOperator{\AgdaDatatype{,}}\AgdaSpace{}%
\AgdaGeneralizable{ϕ}\AgdaSpace{}%
\AgdaOperator{\AgdaDatatype{]}}\<%
\end{code}
In \autoref{fig:rip}, state updates are modelled as merely 
functions from state to state, with an update axiom in \autoref{fig:hoare}
that expresses the pre-condition in terms of the post-condition.
In \textsc{Trip}, however, we use a more general type that allows the update function
to make use of knowledge from the statement's pre-condition. Here, state updates are a function 
from states that satisfy the pre-condition $\varphi$ 
to states that satisfy the post-condition $\psi$:
\begin{code}
\>[2][@{}l@{\AgdaIndent{0}}]%
\>[4]\AgdaInductiveConstructor{UPD}%
\>[10]\AgdaSymbol{:}\AgdaSpace{}%
\AgdaSymbol{(}\AgdaOperator{\AgdaFunction{Σ:}}\AgdaSpace{}%
\AgdaGeneralizable{ϕ}\AgdaSpace{}%
\AgdaSymbol{→}\AgdaSpace{}%
\AgdaOperator{\AgdaFunction{Σ:}}\AgdaSpace{}%
\AgdaGeneralizable{ψ}\AgdaSymbol{)}\AgdaSpace{}%
\AgdaSymbol{→}\AgdaSpace{}%
\AgdaOperator{\AgdaDatatype{[}}\AgdaSpace{}%
\AgdaGeneralizable{ϕ}\AgdaSpace{}%
\AgdaOperator{\AgdaDatatype{,}}\AgdaSpace{}%
\AgdaGeneralizable{ψ}%
\>[37]\AgdaOperator{\AgdaDatatype{]}}\<%
\end{code}
In a delightful confluence of notation, $\AgdaFunction{\ensuremath{\Sigma}:}\ \varphi$
is a dependent product of a state $\sigma$ and a proof that $\varphi$ holds for $\sigma$: 
\begin{code}
\>[2]\AgdaOperator{\AgdaFunction{Σ:\AgdaUnderscore{}}}\AgdaSpace{}%
\AgdaSymbol{:}\AgdaSpace{}%
\AgdaFunction{Assertion}\AgdaSpace{}%
\AgdaSymbol{→}\AgdaSpace{}%
\AgdaPrimitiveType{Set}\<%
\\
%
\>[2]\AgdaOperator{\AgdaFunction{Σ:}}\AgdaSpace{}%
\AgdaBound{ϕ}\AgdaSpace{}%
\AgdaSymbol{=}\AgdaSpace{}%
\AgdaRecord{Σ}\AgdaSpace{}%
\AgdaBound{State}\AgdaSpace{}%
\AgdaSymbol{(λ}\AgdaSpace{}%
\AgdaBound{σ}\AgdaSpace{}%
\AgdaSymbol{→}\AgdaSpace{}%
\AgdaBound{ϕ}\AgdaSpace{}%
\AgdaSymbol{⦃}\AgdaSpace{}%
\AgdaBound{σ}\AgdaSpace{}%
\AgdaSymbol{⦄)}\<%
\end{code}
Our state update is general enough to encode other traditional Hoare logic staples.
For example, the rule of consequence in \autoref{fig:hoare}, which allows us to move from one assertion 
to another by proving a logical implication, can be viewed as a state update that leaves 
the state unchanged:
\begin{code}
\>[2]\AgdaFunction{CONS}\AgdaSpace{}%
\AgdaSymbol{:}\AgdaSpace{}%
\AgdaOperator{\AgdaFunction{Π:}}\AgdaSpace{}%
\AgdaSymbol{(}\AgdaGeneralizable{ϕ}\AgdaSpace{}%
\AgdaSymbol{→}\AgdaSpace{}%
\AgdaGeneralizable{ψ}\AgdaSymbol{)}\AgdaSpace{}%
\AgdaSymbol{→}\AgdaSpace{}%
\AgdaOperator{\AgdaDatatype{[}}\AgdaSpace{}%
\AgdaGeneralizable{ϕ}\AgdaSpace{}%
\AgdaOperator{\AgdaDatatype{,}}\AgdaSpace{}%
\AgdaGeneralizable{ψ}\AgdaSpace{}%
\AgdaOperator{\AgdaDatatype{]}}\<%
\\
%
\>[2]\AgdaFunction{CONS}\AgdaSpace{}%
\AgdaBound{ϕ→ψ}\AgdaSpace{}%
\AgdaSymbol{=}\AgdaSpace{}%
\AgdaInductiveConstructor{UPD}\AgdaSpace{}%
\AgdaSymbol{λ}\AgdaSpace{}%
\AgdaSymbol{\{}\AgdaSpace{}%
\AgdaSymbol{(}\AgdaSpace{}%
\AgdaBound{σ}\AgdaSpace{}%
\AgdaOperator{\AgdaInductiveConstructor{,}}\AgdaSpace{}%
\AgdaBound{ϕ}\AgdaSpace{}%
\AgdaSymbol{)}\AgdaSpace{}%
\AgdaSymbol{→}\AgdaSpace{}%
\AgdaBound{σ}\AgdaSpace{}%
\AgdaOperator{\AgdaInductiveConstructor{,}}\AgdaSpace{}%
\AgdaSymbol{(}\AgdaBound{ϕ→ψ}\AgdaSpace{}%
\AgdaSymbol{⦃}\AgdaSpace{}%
\AgdaBound{σ}\AgdaSpace{}%
\AgdaSymbol{⦄}\AgdaSpace{}%
\AgdaBound{ϕ}\AgdaSymbol{)}\AgdaSpace{}%
\AgdaSymbol{\}}\<%
\end{code}
Here $\AgdaFunction{\ensuremath{\Pi}:}\ \varphi$ is the dual of our earlier $\AgdaFunction{\ensuremath{\Sigma}:}\ \varphi$.
The notation $\AgdaFunction{\ensuremath{\Pi}:}\ \varphi$ states that $\varphi$ holds under \emph{any} state:
\begin{code}
\>[2]\AgdaOperator{\AgdaFunction{Π:\AgdaUnderscore{}}}\AgdaSpace{}%
\AgdaSymbol{:}\AgdaSpace{}%
\AgdaFunction{Assertion}\AgdaSpace{}%
\AgdaSymbol{→}\AgdaSpace{}%
\AgdaPrimitiveType{Set}\<%
\\
%
\>[2]\AgdaOperator{\AgdaFunction{Π:}}\AgdaSpace{}%
\AgdaBound{ϕ}\AgdaSpace{}%
\AgdaSymbol{=}\AgdaSpace{}%
\AgdaSymbol{(∀}\AgdaSpace{}%
\AgdaSymbol{⦃}\AgdaSpace{}%
\AgdaBound{σ}\AgdaSpace{}%
\AgdaSymbol{:}\AgdaSpace{}%
\AgdaBound{State}\AgdaSpace{}%
\AgdaSymbol{⦄}\AgdaSpace{}%
\AgdaSymbol{→}\AgdaSpace{}%
\AgdaBound{ϕ}\AgdaSpace{}%
\AgdaSymbol{⦃}\AgdaSpace{}%
\AgdaBound{σ}\AgdaSpace{}%
\AgdaSymbol{⦄)}\<%
\end{code}
The venerable no-op \textbf{skip} statement is just \AgdaFunction{CONS} with a tautology:
\begin{code}
\>[2]\AgdaFunction{SKIP}\AgdaSpace{}%
\AgdaSymbol{:}\AgdaSpace{}%
\AgdaOperator{\AgdaDatatype{[}}\AgdaSpace{}%
\AgdaGeneralizable{ϕ}\AgdaSpace{}%
\AgdaOperator{\AgdaDatatype{,}}\AgdaSpace{}%
\AgdaGeneralizable{ϕ}\AgdaSpace{}%
\AgdaOperator{\AgdaDatatype{]}}\<%
\\
%
\>[2]\AgdaFunction{SKIP}\AgdaSpace{}%
\AgdaSymbol{=}\AgdaSpace{}%
\AgdaFunction{CONS}\AgdaSpace{}%
\AgdaSymbol{(λ}\AgdaSpace{}%
\AgdaBound{x}\AgdaSpace{}%
\AgdaSymbol{→}\AgdaSpace{}%
\AgdaBound{x}\AgdaSymbol{)}\<%
\end{code}

\subsection{Semantics}
Like the language on which it is based, we shall give the semantics of \textsc{Trip} 
in terms of binary relations on states:
\begin{code}
\>[2]\AgdaFunction{StateRel}\AgdaSpace{}%
\AgdaSymbol{=}\AgdaSpace{}%
\AgdaBound{State}\AgdaSpace{}%
\AgdaSymbol{→}\AgdaSpace{}%
\AgdaBound{State}\AgdaSpace{}%
\AgdaSymbol{→}\AgdaSpace{}%
\AgdaPrimitiveType{Set}\<%
\end{code}
We shall need some operations on state relations, specifically forward composition ($\AgdaFunction{\ensuremath\fatsemi}$) and 
union ($\AgdaFunction{\ensuremath{\cup}}$):
\begin{code}
\>[2]\AgdaOperator{\AgdaFunction{\AgdaUnderscore{}⦂\AgdaUnderscore{}}}\AgdaSpace{}%
\AgdaSymbol{:}\AgdaSpace{}%
\AgdaFunction{StateRel}\AgdaSpace{}%
\AgdaSymbol{→}\AgdaSpace{}%
\AgdaFunction{StateRel}\AgdaSpace{}%
\AgdaSymbol{→}\AgdaSpace{}%
\AgdaFunction{StateRel}\<%
\\
%
\>[2]\AgdaSymbol{(}\AgdaBound{R}\AgdaSpace{}%
\AgdaOperator{\AgdaFunction{⦂}}\AgdaSpace{}%
\AgdaBound{S}\AgdaSymbol{)}\AgdaSpace{}%
\AgdaBound{σ₁}\AgdaSpace{}%
\AgdaBound{σ₃}\AgdaSpace{}%
\AgdaSymbol{=}\AgdaSpace{}%
\AgdaFunction{∃[}\AgdaSpace{}%
\AgdaBound{σ₂}\AgdaSpace{}%
\AgdaFunction{]}\AgdaSpace{}%
\AgdaSymbol{(}\AgdaBound{R}\AgdaSpace{}%
\AgdaBound{σ₁}\AgdaSpace{}%
\AgdaBound{σ₂}\AgdaSpace{}%
\AgdaOperator{\AgdaFunction{×}}\AgdaSpace{}%
\AgdaBound{S}\AgdaSpace{}%
\AgdaBound{σ₂}\AgdaSpace{}%
\AgdaBound{σ₃}\AgdaSymbol{)}\<%
\\
%
\>[2]\AgdaOperator{\AgdaFunction{\AgdaUnderscore{}∪\AgdaUnderscore{}}}\AgdaSpace{}%
\AgdaSymbol{:}\AgdaSpace{}%
\AgdaFunction{StateRel}\AgdaSpace{}%
\AgdaSymbol{→}\AgdaSpace{}%
\AgdaFunction{StateRel}\AgdaSpace{}%
\AgdaSymbol{→}\AgdaSpace{}%
\AgdaFunction{StateRel}\<%
\\
%
\>[2]\AgdaSymbol{(}\AgdaBound{R}\AgdaSpace{}%
\AgdaOperator{\AgdaFunction{∪}}\AgdaSpace{}%
\AgdaBound{S}\AgdaSymbol{)}\AgdaSpace{}%
\AgdaBound{σ₁}\AgdaSpace{}%
\AgdaBound{σ₂}\AgdaSpace{}%
\AgdaSymbol{=}\AgdaSpace{}%
\AgdaBound{R}\AgdaSpace{}%
\AgdaBound{σ₁}\AgdaSpace{}%
\AgdaBound{σ₂}\AgdaSpace{}%
\AgdaOperator{\AgdaDatatype{⊎}}\AgdaSpace{}%
\AgdaBound{S}\AgdaSpace{}%
\AgdaBound{σ₁}\AgdaSpace{}%
\AgdaBound{σ₂}\<%
\end{code}
In addition, we also define the reflexive transitive closure $R\ \AgdaDatatype{\ensuremath{\star}}$ 
of a relation $R$ as an inductive datatype:
\begin{code}
\>[2]\AgdaKeyword{data}\AgdaSpace{}%
\AgdaOperator{\AgdaDatatype{\AgdaUnderscore{}⋆}}\AgdaSpace{}%
\AgdaSymbol{(}\AgdaBound{R}\AgdaSpace{}%
\AgdaSymbol{:}\AgdaSpace{}%
\AgdaFunction{StateRel}\AgdaSymbol{)}\AgdaSpace{}%
\AgdaSymbol{:}\AgdaSpace{}%
\AgdaFunction{StateRel}\AgdaSpace{}%
\AgdaKeyword{where}\<%
\\
\>[2][@{}l@{\AgdaIndent{0}}]%
\>[4]\AgdaInductiveConstructor{refl}\AgdaSpace{}%
\AgdaSymbol{:}\AgdaSpace{}%
\AgdaSymbol{(}\AgdaBound{R}\AgdaSpace{}%
\AgdaOperator{\AgdaDatatype{⋆}}\AgdaSymbol{)}\AgdaSpace{}%
\AgdaGeneralizable{σ}\AgdaSpace{}%
\AgdaGeneralizable{σ}\<%
\\
%
\>[4]\AgdaInductiveConstructor{step}\AgdaSpace{}%
\AgdaSymbol{:}\AgdaSpace{}%
\AgdaBound{R}\AgdaSpace{}%
\AgdaGeneralizable{σ₁}\AgdaSpace{}%
\AgdaGeneralizable{σ₂}\AgdaSpace{}%
\AgdaSymbol{→}\AgdaSpace{}%
\AgdaSymbol{(}\AgdaBound{R}\AgdaSpace{}%
\AgdaOperator{\AgdaDatatype{⋆}}\AgdaSymbol{)}\AgdaSpace{}%
\AgdaGeneralizable{σ₂}\AgdaSpace{}%
\AgdaGeneralizable{σ₃}\AgdaSpace{}%
\AgdaSymbol{→}\AgdaSpace{}%
\AgdaSymbol{(}\AgdaBound{R}\AgdaSpace{}%
\AgdaOperator{\AgdaDatatype{⋆}}\AgdaSymbol{)}\AgdaSpace{}%
\AgdaGeneralizable{σ₁}\AgdaSpace{}%
\AgdaGeneralizable{σ₃}\<%
\end{code}
These operators allow us to give semantics to sequential composition,
non-deterministic choice and Kleene star respectively:
\begin{code}
\>[2]\AgdaOperator{\AgdaFunction{⟦\AgdaUnderscore{}⟧}}\AgdaSpace{}%
\AgdaSymbol{:}\AgdaSpace{}%
\AgdaOperator{\AgdaDatatype{[}}\AgdaSpace{}%
\AgdaGeneralizable{ϕ}\AgdaSpace{}%
\AgdaOperator{\AgdaDatatype{,}}\AgdaSpace{}%
\AgdaGeneralizable{ψ}\AgdaSpace{}%
\AgdaOperator{\AgdaDatatype{]}}\AgdaSpace{}%
\AgdaSymbol{→}\AgdaSpace{}%
\AgdaFunction{StateRel}\<%
\\
%
\>[2]\AgdaOperator{\AgdaFunction{⟦}}\AgdaSpace{}%
\AgdaInductiveConstructor{SEQ}\AgdaSpace{}%
\AgdaBound{P}\AgdaSpace{}%
\AgdaBound{Q}\AgdaSpace{}%
\>[18]\AgdaOperator{\AgdaFunction{⟧}}\AgdaSpace{}%
\AgdaSymbol{=}\AgdaSpace{}%
\AgdaOperator{\AgdaFunction{⟦}}\AgdaSpace{}%
\AgdaBound{P}\AgdaSpace{}%
\AgdaOperator{\AgdaFunction{⟧}}\AgdaSpace{}%
\AgdaOperator{\AgdaSpace$\!$\AgdaFunction{⦂}}\AgdaSpace{}%
\>[22]\AgdaOperator{\AgdaFunction{⟦}}\AgdaSpace{}%
\AgdaBound{Q}\AgdaSpace{}%
\AgdaOperator{\AgdaFunction{⟧}}\<%
\\
%
\>[2]\AgdaOperator{\AgdaFunction{⟦}}\AgdaSpace{}%
\AgdaInductiveConstructor{CHO}\AgdaSpace{}%
\AgdaBound{P}\AgdaSpace{}%
\AgdaBound{Q}\AgdaSpace{}%
\>[18]\AgdaOperator{\AgdaFunction{⟧}}\AgdaSpace{}%
\AgdaSymbol{=}\AgdaSpace{}%
\AgdaOperator{\AgdaFunction{⟦}}\AgdaSpace{}%
\AgdaBound{P}\AgdaSpace{}%
\AgdaOperator{\AgdaFunction{⟧}}\AgdaSpace{}%
\AgdaOperator{\AgdaFunction{∪}}\AgdaSpace{}%
\>[22]\AgdaOperator{\AgdaFunction{⟦}}\AgdaSpace{}%
\AgdaBound{Q}\AgdaSpace{}%
\AgdaOperator{\AgdaFunction{⟧}}\<%
\\
%
\>[2]\AgdaOperator{\AgdaFunction{⟦}}\AgdaSpace{}%
\AgdaInductiveConstructor{STAR}\AgdaSpace{}%
\AgdaBound{P}\AgdaSpace{}%
\>[18]\AgdaOperator{\AgdaFunction{⟧}}\AgdaSpace{}%
\AgdaSymbol{=}\AgdaSpace{}%
\AgdaOperator{\AgdaFunction{⟦}}\AgdaSpace{}%
\AgdaBound{P}\AgdaSpace{}%
\AgdaOperator{\AgdaFunction{⟧}}\AgdaSpace{}%
\AgdaOperator{\AgdaDatatype{⋆}}\<%
\\
%
\>[2]\AgdaOperator{\AgdaFunction{⟦}}\AgdaSpace{}%
\AgdaInductiveConstructor{GUARD}\AgdaSpace{}%
\AgdaBound{g}\AgdaSpace{}%
\>[18]\AgdaOperator{\AgdaFunction{⟧}}\AgdaSpace{}%
\AgdaSymbol{=}\AgdaSpace{}%
\AgdaOperator{\AgdaFunction{⟦}}\AgdaSpace{}%
\AgdaBound{g}\AgdaSpace{}%
\AgdaOperator{\AgdaFunction{⟧g}}\<%
\\
%
\>[2]\AgdaOperator{\AgdaFunction{⟦}}\AgdaSpace{}%
\AgdaInductiveConstructor{UPD}\AgdaSpace{}%
\AgdaBound{u}\AgdaSpace{}%
\>[18]\AgdaOperator{\AgdaFunction{⟧}}\AgdaSpace{}%
\AgdaSymbol{=}\AgdaSpace{}%
\AgdaOperator{\AgdaFunction{⟦}}\AgdaSpace{}%
\AgdaBound{u}\AgdaSpace{}%
\AgdaOperator{\AgdaFunction{⟧u}}\<%
\end{code}
The semantics of guard statements
\AgdaOperator{\AgdaFunction{⟦}}\AgdaSpace{}%
\AgdaBound{g}\AgdaSpace{}%
\AgdaOperator{\AgdaFunction{⟧g}}\AgdaSpace{}%tement 
are the subset of the identity relation where the 
state satisfies $g$:\nopagebreak
\begin{code}
\>[2]\AgdaOperator{\AgdaFunction{⟦\AgdaUnderscore{}⟧g}}\AgdaSpace{}%
\AgdaSymbol{:}%
\>[10]\AgdaFunction{Assertion}\AgdaSpace{}%
\AgdaSymbol{→}\AgdaSpace{}%
\AgdaFunction{StateRel}\<%
\\
%
\>[2]\AgdaOperator{\AgdaFunction{⟦}}\AgdaSpace{}%
\AgdaBound{g}\AgdaSpace{}%
\AgdaOperator{\AgdaFunction{⟧g}}\AgdaSpace{}%
\AgdaBound{σ₁}\AgdaSpace{}%
\AgdaBound{σ₂}\AgdaSpace{}%
\AgdaSymbol{=}\AgdaSpace{}%
\AgdaBound{g}\AgdaSpace{}%
\AgdaSymbol{⦃}\AgdaSpace{}%
\AgdaBound{σ₁}\AgdaSpace{}%
\AgdaSymbol{⦄}\AgdaSpace{}%
\AgdaOperator{\AgdaFunction{×}}\AgdaSpace{}%
\AgdaBound{σ₁}\AgdaSpace{}%
\AgdaOperator{\AgdaDatatype{≡}}\AgdaSpace{}%
\AgdaBound{σ₂}\<%
\end{code}
State update statements are given semantics simply by 
interpreting the state update function as a relation:
\begin{code}
\>[2]\AgdaOperator{\AgdaFunction{⟦\AgdaUnderscore{}⟧u}}\AgdaSpace{}%
\AgdaSymbol{:}\AgdaSpace{}%
\AgdaSymbol{(}\AgdaOperator{\AgdaFunction{Σ:}}\AgdaSpace{}%
\AgdaGeneralizable{ϕ}\AgdaSpace{}%
\AgdaSymbol{→}\AgdaSpace{}%
\AgdaOperator{\AgdaFunction{Σ:}}\AgdaSpace{}%
\AgdaGeneralizable{ψ}\AgdaSymbol{)}\AgdaSpace{}%
\AgdaSymbol{→}\AgdaSpace{}%
\AgdaFunction{StateRel}\<%
\\
%
\>[2]\AgdaOperator{\AgdaFunction{⟦\AgdaSpace{}}\AgdaBound{u}\AgdaSpace{}\AgdaFunction{⟧u}}\AgdaSpace{}%
\AgdaBound{σ₁}\AgdaSpace{}%
\AgdaBound{σ₂}\AgdaSpace{}%
\AgdaSymbol{=}\AgdaSpace{}%
\AgdaSymbol{∀}\AgdaSpace{}%
\AgdaBound{prf-ϕ}\AgdaSpace{}%
\AgdaSymbol{→}%
\>[473I]\AgdaKeyword{let}\AgdaSpace{}%
\AgdaBound{σ₂$'$}\AgdaSpace{}%
\AgdaOperator{\AgdaInductiveConstructor{,}}\AgdaSpace{}%
\AgdaSymbol{\AgdaUnderscore{}}\AgdaSpace{}%
\AgdaSymbol{=}\AgdaSpace{}%
\AgdaBound{u}\AgdaSpace{}%
\AgdaSymbol{(}\AgdaBound{σ₁}\AgdaSpace{}%
\AgdaOperator{\AgdaInductiveConstructor{,}}\AgdaSpace{}%
\AgdaBound{prf-ϕ}\AgdaSymbol{)}\<%
\\
\>[473I][@{}l@{\AgdaIndent{0}}]%
\AgdaKeyword{in}%
\>[32]\!\!\AgdaBound{σ₂}\AgdaSpace{}%
\AgdaOperator{\AgdaDatatype{≡}}\AgdaSpace{}%
\AgdaBound{σ₂$'$}\<%
\end{code}
\subsection{Soundness}\label{subsec:soundness}
Our soundness result connects the specification of a program to its semantics.
It states that for any execution of a program 
\AgdaBound{P}\AgdaSpace{}%
\AgdaSymbol{:}\AgdaSpace{}%
\AgdaOperator{\AgdaDatatype{[}}\AgdaSpace{}%
\AgdaGeneralizable{ϕ}\AgdaSpace{}%
\AgdaOperator{\AgdaDatatype{,}}\AgdaSpace{}%
\AgdaGeneralizable{ψ}\AgdaSpace{}%
\AgdaOperator{\AgdaDatatype{]}}
from initial state $\sigma_1$ to 
final state $\sigma_2$, if the initial state $\sigma_1$ satisfies the pre-condition $\varphi$
then the final state $\sigma_2$ will satisfy the post-condition $\psi$.
\begin{code}
\>[2]\AgdaFunction{sound}%
\>[552I]\AgdaSymbol{:}\AgdaSpace{}%
\AgdaSymbol{(}\AgdaBound{P}\AgdaSpace{}%
\AgdaSymbol{:}\AgdaSpace{}%
\AgdaOperator{\AgdaDatatype{[}}\AgdaSpace{}%
\AgdaGeneralizable{ϕ}\AgdaSpace{}%
\AgdaOperator{\AgdaDatatype{,}}\AgdaSpace{}%
\AgdaGeneralizable{ψ}\AgdaSpace{}%
\AgdaOperator{\AgdaDatatype{]}}\AgdaSymbol{)}\<%
\>[8]\AgdaSymbol{→}\AgdaSpace{}%
\AgdaOperator{\AgdaFunction{⟦}}\AgdaSpace{}%
\AgdaBound{P}\AgdaSpace{}%
\AgdaOperator{\AgdaFunction{⟧}}\AgdaSpace{}%
\AgdaGeneralizable{σ₁}\AgdaSpace{}%
\AgdaGeneralizable{σ₂}\AgdaSpace{}%
\AgdaSymbol{→}\AgdaSpace{}%
\AgdaGeneralizable{ϕ}%
\>[27]\AgdaSymbol{⦃}\AgdaSpace{}\!%
\AgdaGeneralizable{σ₁}\AgdaSpace{}\!%
\AgdaSymbol{⦄}\AgdaSpace{}%
\AgdaSymbol{→}\AgdaSpace{}%
\AgdaGeneralizable{ψ}\AgdaSpace{}%
\AgdaSymbol{⦃}\AgdaSpace{}\!%
\AgdaGeneralizable{σ₂}\AgdaSpace{}\!%
\AgdaSymbol{⦄}\<%
\end{code}
Recall that the typing rule for a sequential composition $P\ \AgdaFunction{;}\ Q$ requires an intermediate assertion $\alpha$ 
as both the post-condition for $P$ and the pre-condition for $Q$. Therefore, soundness for the sequential composition is established 
simply by the composition of inductive hypotheses for $P$ and $Q$ by transitivity of implication (i.e.\ function composition): 
\begin{code}
\>[2]\AgdaFunction{sound}\AgdaSpace{}%
\AgdaSymbol{(}\AgdaInductiveConstructor{SEQ}\AgdaSpace{}%
\AgdaBound{P}\AgdaSpace{}%
\AgdaBound{Q}\AgdaSymbol{)}\AgdaSpace{}%
\AgdaSymbol{(\AgdaUnderscore{}}\AgdaSpace{}%
\AgdaOperator{\AgdaInductiveConstructor{,}}\AgdaSpace{}%
\AgdaBound{p}\AgdaSpace{}%
\AgdaOperator{\AgdaInductiveConstructor{,}}\AgdaSpace{}%
\AgdaBound{q}\AgdaSymbol{)}\AgdaSpace{}%
\>[30]\AgdaSymbol{=}\AgdaSpace{}%
\AgdaFunction{sound}\AgdaSpace{}%
\AgdaBound{Q}\AgdaSpace{}%
\AgdaBound{q}\AgdaSpace{}%
\AgdaOperator{\AgdaFunction{∘}}\AgdaSpace{}%
\AgdaFunction{sound}\AgdaSpace{}%
\AgdaBound{P}\AgdaSpace{}%
\AgdaBound{p}\<%
\end{code}
For a non-deterministic choice $P\ \AgdaFunction{\ensuremath{+}}\ Q$, 
soundness follows straightforwardly from the inductive hypotheses,
depending on which of $P$ or $Q$ was the origin of the execution in question:
\begin{code}
\>[2]\AgdaFunction{sound}\AgdaSpace{}%
\AgdaSymbol{(}\AgdaInductiveConstructor{CHO}\AgdaSpace{}%
\AgdaBound{P}\AgdaSpace{}%
\AgdaBound{Q}\AgdaSymbol{)}\AgdaSpace{}%
\AgdaSymbol{(}\AgdaInductiveConstructor{inj₁}\AgdaSpace{}%
\AgdaBound{p}\AgdaSymbol{)}%
\>[30]\AgdaSymbol{=}\AgdaSpace{}%
\AgdaFunction{sound}\AgdaSpace{}%
\AgdaBound{P}\AgdaSpace{}%
\AgdaBound{p}\<%
\\
%
\>[2]\AgdaFunction{sound}\AgdaSpace{}%
\AgdaSymbol{(}\AgdaInductiveConstructor{CHO}\AgdaSpace{}%
\AgdaBound{P}\AgdaSpace{}%
\AgdaBound{Q}\AgdaSymbol{)}\AgdaSpace{}%
\AgdaSymbol{(}\AgdaInductiveConstructor{inj₂}\AgdaSpace{}%
\AgdaBound{q}\AgdaSymbol{)}%
\>[30]\AgdaSymbol{=}\AgdaSpace{}%
\AgdaFunction{sound}\AgdaSpace{}%
\AgdaBound{Q}\AgdaSpace{}%
\AgdaBound{q}\<%
\end{code}
Soundness for Kleene star is established by a structural induction on 
the reflexive transitive closure, effectively unrolling the inductive step of $\mathcal{P}^\star$ to 
the equivalent $(\mathcal{P};\mathcal{P}^\star)$ and the base case to $\mathbf{skip}$:
\begin{code}
\>[2]\AgdaFunction{sound}\AgdaSpace{}%
\AgdaSymbol{(}\AgdaInductiveConstructor{STAR}\AgdaSpace{}%
\AgdaBound{P}\AgdaSymbol{)}\AgdaSpace{}%
\AgdaInductiveConstructor{refl}%
\>[30]\AgdaSymbol{=}\AgdaSpace{}%
\AgdaFunction{id}\<%
\\
%
\>[2]\AgdaFunction{sound}\AgdaSpace{}%
\AgdaSymbol{(}\AgdaInductiveConstructor{STAR}\AgdaSpace{}%
\AgdaBound{P}\AgdaSymbol{)}\AgdaSpace{}%
\AgdaSymbol{(}\AgdaInductiveConstructor{step}\AgdaSpace{}%
\AgdaBound{p}\AgdaSpace{}%
\AgdaBound{ps}\AgdaSymbol{)}%
\>[30]\AgdaSymbol{=}\AgdaSpace{}%
\AgdaFunction{sound}\AgdaSpace{}\!%
\AgdaSymbol{(}\AgdaInductiveConstructor{STAR}\AgdaSpace{}%
\AgdaBound{P}\AgdaSymbol{)}\AgdaSpace{}%
\AgdaBound{ps}\AgdaSpace{}%
\AgdaOperator{\AgdaFunction{∘}}\AgdaSpace{}%
\AgdaFunction{sound}\AgdaSpace{}\!%
\AgdaBound{P}\AgdaSpace{}%
\AgdaBound{p}\<%
\end{code}
A guard $g$ has a pre-condition of $g \rightarrow \varphi$ and a post-condition of 
$\varphi$. Seeing as the semantics of guards require all executions to satisfy $g$, 
we can establish the post-condition by modus ponens (i.e.\ function application):
\begin{code}
\>[2]\AgdaFunction{sound}\AgdaSpace{}%
\AgdaSymbol{(}\AgdaInductiveConstructor{GUARD}\AgdaSpace{}%
\AgdaBound{g}\AgdaSymbol{)}\AgdaSpace{}%
\AgdaSymbol{(}\AgdaBound{p}\AgdaSpace{}%
\AgdaOperator{\AgdaInductiveConstructor{,}}\AgdaSpace{}%
\AgdaInductiveConstructor{refl}\AgdaSymbol{)}%
\>[30]\AgdaSymbol{=}\AgdaSpace{}%
\AgdaOperator{\AgdaFunction{\AgdaUnderscore{}\$}}\AgdaSpace{}%
\AgdaBound{p}\<%
\end{code}
For state updates, the state update function itself carries the proof 
of its soundness. Therefore, the overall soundness proof must merely 
observe the determinism of the update function when interpreted as a relation:
\begin{code}
\>[2]\AgdaFunction{sound}\AgdaSpace{}%
\AgdaSymbol{(}\AgdaInductiveConstructor{UPD}\AgdaSpace{}%
\AgdaBound{x}\AgdaSymbol{)}\AgdaSpace{}%
\AgdaSymbol{(}\AgdaBound{p}\AgdaSymbol{)}%
\>[30]\AgdaSymbol{=}\AgdaSpace{}%
\AgdaFunction{snd-upd}\AgdaSpace{}%
\AgdaBound{x}\AgdaSpace{}%
\AgdaBound{p}\<%
\\
\>[2][@{}l@{\AgdaIndent{0}}]%
\>[4]\AgdaKeyword{where}\<%
\\
\>[4][@{}l@{\AgdaIndent{0}}]%
\>[6]\AgdaFunction{snd-upd}%
\>[636I]\AgdaSymbol{:}\AgdaSpace{}%
\AgdaSymbol{(}\AgdaBound{u}\AgdaSpace{}%
\AgdaSymbol{:}\AgdaSpace{}%
\AgdaOperator{\AgdaFunction{Σ:}}\AgdaSpace{}%
\AgdaGeneralizable{ϕ}\AgdaSpace{}%
\AgdaSymbol{→}\AgdaSpace{}%
\AgdaOperator{\AgdaFunction{Σ:}}\AgdaSpace{}%
\AgdaGeneralizable{ψ}\AgdaSymbol{)}\<%
\\
\>[.][@{}l@{}]\<[636I]%
\>[16]\AgdaSymbol{→}\AgdaSpace{}%
\AgdaOperator{\AgdaFunction{⟦}}\AgdaSpace{}%
\AgdaBound{u}\AgdaSpace{}%
\AgdaOperator{\AgdaFunction{⟧u}}\AgdaSpace{}%
\AgdaGeneralizable{σ₁}\AgdaSpace{}%
\AgdaGeneralizable{σ₂}\AgdaSpace{}%
\AgdaSymbol{→}\AgdaSpace{}%
\AgdaGeneralizable{ϕ}\AgdaSpace{}%
\AgdaSymbol{⦃}\AgdaSpace{}%
\AgdaGeneralizable{σ₁}\AgdaSpace{}%
\AgdaSymbol{⦄}\AgdaSpace{}%
\AgdaSymbol{→}\AgdaSpace{}%
\AgdaGeneralizable{ψ}\AgdaSpace{}%
\AgdaSymbol{⦃}\AgdaSpace{}%
\AgdaGeneralizable{σ₂}\AgdaSpace{}%
\AgdaSymbol{⦄}\<%
\\
%
\>[6]\AgdaFunction{snd-upd}\AgdaSpace{}%
\AgdaBound{u}\AgdaSpace{}%
\AgdaBound{sem}\AgdaSpace{}%
\AgdaBound{prf-ϕ}\<%
\>[7]\AgdaKeyword{with}\AgdaSpace{}%
\AgdaBound{u}\AgdaSpace{}%
\AgdaSymbol{(\AgdaUnderscore{}}\AgdaSpace{}%
\AgdaOperator{\AgdaInductiveConstructor{,}}\AgdaSpace{}%
\AgdaBound{prf-ϕ}\AgdaSymbol{)}\AgdaSpace{}%
\AgdaSymbol{|}\AgdaSpace{}%
\AgdaBound{sem}\AgdaSpace{}%
\AgdaBound{prf-ϕ}\<%
\\
\>[6]\AgdaSymbol{...}\AgdaSpace{}%
\AgdaSymbol{|}\AgdaSpace{}%
\AgdaSymbol{\AgdaUnderscore{}}\AgdaSpace{}%
\AgdaOperator{\AgdaInductiveConstructor{,}}\AgdaSpace{}%
\AgdaBound{prf-ψ}%
\>[26]\AgdaSymbol{|}\AgdaSpace{}%
\AgdaInductiveConstructor{refl}%
\>[39]\AgdaSymbol{=}\AgdaSpace{}%
\AgdaBound{prf-ψ}\<%
\end{code}
\subsection{Deterministic constructs}
Now that we have defined our core language and established the soundness of its Hoare logic
types, any derivable language construct from that sound core 
is necessarily also sound. For example, the deterministic conditional $\textbf{if}$ 
statement we derived earlier can be given a type resembling the typical Hoare logic 
rule for $\textbf{if}$:
\begin{code}
\>[2]\AgdaFunction{IFTHENELSE}%
\>[994I]\AgdaSymbol{:}\AgdaSpace{}%
\AgdaSymbol{(}\AgdaBound{g}\AgdaSpace{}%
\AgdaSymbol{:}\AgdaSpace{}%
\AgdaFunction{Assertion}\AgdaSymbol{)}\<%
\\
\>[.][@{}l@{}]\<[994I]%
\>[13]\AgdaSymbol{→}\AgdaSpace{}%
\AgdaOperator{\AgdaDatatype{[}}\AgdaSpace{}%
\AgdaGeneralizable{ϕ}\AgdaSpace{}%
\AgdaOperator{\AgdaFunction{×}}\AgdaSpace{}%
\AgdaBound{g}\AgdaSpace{}%
\AgdaOperator{\AgdaDatatype{,}}\AgdaSpace{}%
\AgdaGeneralizable{ψ}\AgdaSpace{}%
\AgdaOperator{\AgdaDatatype{]}}\<%
\\
%
\>[13]\AgdaSymbol{→}\AgdaSpace{}%
\AgdaOperator{\AgdaDatatype{[}}\AgdaSpace{}%
\AgdaGeneralizable{ϕ}\AgdaSpace{}%
\AgdaOperator{\AgdaFunction{×}}\AgdaSpace{}%
\AgdaOperator{\AgdaFunction{¬}}\AgdaSpace{}%
\AgdaBound{g}\AgdaSpace{}%
\AgdaOperator{\AgdaDatatype{,}}\AgdaSpace{}%
\AgdaGeneralizable{ψ}%
\>[30]\AgdaOperator{\AgdaDatatype{]}}\<%
\\
%
\>[13]\AgdaSymbol{→}\AgdaSpace{}%
\AgdaOperator{\AgdaDatatype{[}}\AgdaSpace{}%
\AgdaGeneralizable{ϕ}\AgdaSpace{}%
\AgdaOperator{\AgdaDatatype{,}}\AgdaSpace{}%
\AgdaGeneralizable{ψ}\AgdaSpace{}%
\AgdaOperator{\AgdaDatatype{]}}\<%
\end{code}
The implementation is comprised solely of already-defined 
constructs. The translation is essentially the same as the one in \autoref{subsec:rip}, 
with the addition of the rule of consequence, used to make the assertions fit together:
\begin{code}
\>[2]\AgdaFunction{IFTHENELSE}\AgdaSpace{}%
\AgdaBound{g}\AgdaSpace{}%
\AgdaBound{P}\AgdaSpace{}%
\AgdaBound{Q}\<%
\\
\>[2][@{}l@{\AgdaIndent{0}}]%
\>[4]\AgdaSymbol{=}\AgdaSpace{}%
\AgdaInductiveConstructor{CHO}%
\>[1021I]\AgdaSymbol{(}\AgdaInductiveConstructor{SEQ}\AgdaSpace{}%
\AgdaSymbol{(}\AgdaInductiveConstructor{SEQ}\AgdaSpace{}%
\AgdaSymbol{(}\AgdaFunction{CONS}\AgdaSpace{}%
\AgdaOperator{\AgdaInductiveConstructor{\AgdaUnderscore{},\AgdaUnderscore{}}}\AgdaSymbol{)}\AgdaSpace{}%
\AgdaSymbol{(}\AgdaInductiveConstructor{GUARD}\AgdaSpace{}%
\AgdaBound{g}\AgdaSymbol{))}\AgdaSpace{}%
\AgdaBound{P}\AgdaSymbol{)}\<%
\\
\>[.][@{}l@{}]\<[1021I]%
\>[10]\AgdaSymbol{(}\AgdaInductiveConstructor{SEQ}\AgdaSpace{}%
\AgdaSymbol{(}\AgdaInductiveConstructor{SEQ}\AgdaSpace{}%
\AgdaSymbol{(}\AgdaFunction{CONS}\AgdaSpace{}%
\AgdaOperator{\AgdaInductiveConstructor{\AgdaUnderscore{},\AgdaUnderscore{}}}\AgdaSymbol{)}\AgdaSpace{}%
\AgdaSymbol{(}\AgdaInductiveConstructor{GUARD}\AgdaSpace{}%
\AgdaSymbol{(}\AgdaOperator{\AgdaFunction{¬}}\AgdaSpace{}%
\AgdaBound{g}\AgdaSymbol{)))}\AgdaSpace{}%
\AgdaBound{Q}\AgdaSymbol{)}\<%
\end{code}
The rule for \textbf{while} loops is also similar to \autoref{subsec:rip}, save 
for the addition of the rule of consequence:
\begin{code}
\>[2]\AgdaFunction{WHILE}\AgdaSpace{}%
\AgdaSymbol{:}\AgdaSpace{}%
\AgdaSymbol{(}\AgdaBound{g}\AgdaSpace{}%
\AgdaSymbol{:}\AgdaSpace{}%
\AgdaFunction{Assertion}\AgdaSymbol{)}\AgdaSpace{}%
\AgdaSymbol{→}\AgdaSpace{}%
\AgdaOperator{\AgdaDatatype{[}}\AgdaSpace{}%
\AgdaBound{g}\AgdaSpace{}%
\AgdaOperator{\AgdaFunction{×}}\AgdaSpace{}%
\AgdaGeneralizable{ϕ}\AgdaSpace{}%
\AgdaOperator{\AgdaDatatype{,}}\AgdaSpace{}%
\AgdaGeneralizable{ϕ}\AgdaSpace{}%
\AgdaOperator{\AgdaDatatype{]}}\AgdaSpace{}%
\AgdaSymbol{→}\AgdaSpace{}%
\AgdaOperator{\AgdaDatatype{[}}\AgdaSpace{}%
\AgdaGeneralizable{ϕ}\AgdaSpace{}%
\AgdaOperator{\AgdaDatatype{,}}\AgdaSpace{}%
\AgdaOperator{\AgdaFunction{¬}}\AgdaSpace{}%
\AgdaBound{g}\AgdaSpace{}%
\AgdaOperator{\AgdaFunction{×}}\AgdaSpace{}%
\AgdaGeneralizable{ϕ}\AgdaSpace{}%
\AgdaOperator{\AgdaDatatype{]}}\<%
\\
%
\>[2]\AgdaFunction{WHILE}\AgdaSpace{}%
\AgdaBound{g}\AgdaSpace{}%
\AgdaBound{P}\AgdaSpace{}%
\AgdaSymbol{=}\AgdaSpace{}%
\AgdaInductiveConstructor{SEQ}%
\>[1011I]\AgdaSymbol{(}\AgdaInductiveConstructor{STAR}\AgdaSpace{}%
\AgdaSymbol{(}\AgdaInductiveConstructor{SEQ}%
\>[1013I]\AgdaSymbol{(}\AgdaInductiveConstructor{SEQ}%
\>[1014I]\AgdaSymbol{(}\AgdaFunction{CONS}\AgdaSpace{}%
\AgdaSymbol{(}\AgdaFunction{flip}\AgdaSpace{}%
\AgdaOperator{\AgdaInductiveConstructor{\AgdaUnderscore{},\AgdaUnderscore{}}}\AgdaSymbol{))}\<%
\\
\>[.][@{}l@{}]\<[1014I]%
\>[34]\AgdaSymbol{(}\AgdaInductiveConstructor{GUARD}\AgdaSpace{}%
\AgdaBound{g}\AgdaSymbol{))}\<%
\\
\>[.][@{}l@{}]\<[1013I]%
\>[29]\AgdaBound{P}\AgdaSymbol{))}\<%
\\
\>[.][@{}l@{}]\<[1011I]%
\>[18]\AgdaSymbol{(}\AgdaInductiveConstructor{SEQ}%
\>[1018I]\AgdaSymbol{(}\AgdaFunction{CONS}\AgdaSpace{}%
\AgdaSymbol{(}\AgdaFunction{flip}\AgdaSpace{}%
\AgdaOperator{\AgdaInductiveConstructor{\AgdaUnderscore{},\AgdaUnderscore{}}}\AgdaSymbol{))}\<%
\\
\>[.][@{}l@{}]\<[1018I]%
\>[23]\AgdaSymbol{(}\AgdaInductiveConstructor{GUARD}\AgdaSpace{}%
\AgdaSymbol{(}\AgdaOperator{\AgdaFunction{¬}}\AgdaSpace{}%
\AgdaBound{g}\AgdaSymbol{)))}\<%
\end{code}

\section{The Surface of \textsc{Trip}}\label{sec:surface}

As can be glimpsed in our core language translations for \textbf{if} 
and \textbf{while}, nesting proof terms directly inside the program
whenever the rule of consequence is used produces significantly less 
readable results than a typical pen-and-paper Hoare logic 
derivation. This is for two main reasons:
\begin{itemize}
\item The proof terms do not indicate what the intermediate assertion being established is,
      but merely how to prove it.
\item The proof terms clutter the program with a potentially large number of 
      terms that are not computationally relevant.
\end{itemize}
In pen-and-paper Hoare logic derivations, the use of the rule of consequence 
is usually left implicit. Instead, the program is annotated with an assertion, 
and the proof of the implication required to establish that assertion is 
presented afterwards.

For the surface language of \textsc{Trip}, we can achieve a similar effect 
with our \AgdaDatatype{Delay} applicative. In our surface language, use of 
the consequence rule is indicated by an \AgdaFunction{assert}
statement that uses the \AgdaDatatype{Delay} applicative to defer proving 
the implication until \AgdaFunction{later}:
\begin{code}
\>[2]\AgdaFunction{assert}\AgdaSpace{}%
\AgdaSymbol{:}\AgdaSpace{}%
\AgdaSymbol{(}\AgdaBound{ϕ}\AgdaSpace{}%
\AgdaSymbol{:}\AgdaSpace{}%
\AgdaFunction{Assertion}\AgdaSymbol{)}\AgdaSpace{}%
\AgdaSymbol{→}\AgdaSpace{}%
\AgdaRecord{Delay}\AgdaSpace{}%
\AgdaOperator{\AgdaDatatype{[}}\AgdaSpace{}%
\AgdaBound{ϕ}\AgdaSpace{}%
\AgdaOperator{\AgdaDatatype{,}}\AgdaSpace{}%
\AgdaGeneralizable{ψ}\AgdaSpace{}%
\AgdaOperator{\AgdaDatatype{]}}\<%
\\
%
\>[2]\AgdaFunction{assert}\AgdaSpace{}%
\AgdaBound{ϕ}\AgdaSpace{}%
\AgdaSymbol{=}\AgdaSpace{}%
\AgdaSymbol{⦇}\AgdaSpace{}%
\AgdaFunction{CONS}\AgdaSpace{}%
\AgdaFunction{later}\AgdaSpace{}%
\AgdaSymbol{⦈}\<%
\end{code}
The user provides the \emph{pre-condition} $\varphi$ explicitly but the post-condition 
is left implicit. This is because of the structure of many of the Hoare logic rules given 
in \autoref{subsec:trip} are such that the pre-condition is inferrable from 
the post-condition, in the spirit of the weakest pre-condition calculi of \citet{dijkstra}. Thus it is more likely 
that Agda will be able to infer the post-condition $\psi$ from the subsequent parts of the program, and 
require specification of the pre-condition $\varphi$.

We also define wrappers for sequential composition, \textbf{if}, \textbf{while}, and state update in our \AgdaFunction{Delay} applicative:
\\[-0.5em]

\noindent \AgdaBound{P}\AgdaSpace{}%
\AgdaOperator{\AgdaFunction{﹕}}\AgdaSpace{}%
\AgdaBound{Q}\AgdaSpace{}%
\AgdaSymbol{=}\AgdaSpace{}%
\AgdaSymbol{⦇}\AgdaSpace{}%
\AgdaInductiveConstructor{SEQ}\AgdaSpace{}%
\AgdaBound{P}\AgdaSpace{}%
\AgdaBound{Q}\AgdaSpace{}%
\AgdaSymbol{⦈}
\\[-0.5em]

\noindent \AgdaOperator{\AgdaFunction{if}}\AgdaSpace{}%
\AgdaBound{g}\AgdaSpace{}%
\AgdaOperator{\AgdaFunction{then}}\AgdaSpace{}%
\AgdaBound{p}\AgdaSpace{}%
\AgdaOperator{\AgdaFunction{else}}\AgdaSpace{}%
\AgdaBound{q}\AgdaSpace{}%
\AgdaOperator{\AgdaFunction{fi}}\AgdaSpace{}%
\AgdaSymbol{=}\AgdaSpace{}%
\AgdaSymbol{⦇}\AgdaSpace{}%
\AgdaSymbol{(}\AgdaFunction{IFTHENELSE}\AgdaSpace{}%
\AgdaBound{g}\AgdaSymbol{)}\AgdaSpace{}%
\AgdaBound{p}\AgdaSpace{}%
\AgdaBound{q}\AgdaSpace{}%
\AgdaSymbol{⦈}%
\begin{code}
\>[2]\AgdaOperator{\AgdaFunction{while}}\AgdaSpace{}%
\AgdaBound{g}\AgdaSpace{}%
\AgdaOperator{\AgdaFunction{begin}}\AgdaSpace{}%
\AgdaBound{P}\AgdaSpace{}%
\AgdaOperator{\AgdaFunction{end}}\AgdaSpace{}%
\AgdaSymbol{=}\AgdaSpace{}%
\AgdaSymbol{⦇}\AgdaSpace{}%
\AgdaSymbol{(}\AgdaFunction{WHILE}\AgdaSpace{}%
\AgdaBound{g}\AgdaSymbol{)}\AgdaSpace{}%
\AgdaBound{P}\AgdaSpace{}%
\AgdaSymbol{⦈}\<%
\end{code}

\noindent \AgdaFunction{upd}\AgdaSpace{}%
\AgdaBound{u}\AgdaSpace{}%
\AgdaSymbol{=}\AgdaSpace{}%
\AgdaFunction{pure}\AgdaSpace{}%
\AgdaSymbol{(}\AgdaInductiveConstructor{UPD}\AgdaSpace{}%
\AgdaBound{u}\AgdaSymbol{)}

\subsection{Initial attempt at swap}

With the basic constructs of our surface language defined, we can now 
attempt to write and verify some basic programs. One of the simplest
is the program that swaps two variables using a temporary storage variable.

For this program we shall define our state parameter to be the following record type:

\begin{code}
\>[2]\AgdaKeyword{record}\AgdaSpace{}%
\AgdaRecord{SwapState}\AgdaSpace{}%
\AgdaSymbol{:}\AgdaSpace{}%
\AgdaPrimitiveType{Set}\AgdaSpace{}%
\AgdaKeyword{where}\<%
\\
\>[2][@{}l@{\AgdaIndent{0}}]%
\>[4]\AgdaKeyword{field}\<%
\\
\>[4][@{}l@{\AgdaIndent{0}}]%
\>[6]\AgdaField{i}\AgdaSpace{}%
\AgdaSymbol{:}\AgdaSpace{}%
\AgdaDatatype{ℕ}\<%
\\
%
\>[6]\AgdaField{j}\AgdaSpace{}%
\AgdaSymbol{:}\AgdaSpace{}%
\AgdaDatatype{ℕ}\<%
\\
%
\>[6]\AgdaField{temp}\AgdaSpace{}%
\AgdaSymbol{:}\AgdaSpace{}%
\AgdaDatatype{ℕ}\<%
\end{code}
The specification of a swap procedure is interesting because it requires the 
use of logical constants or \emph{freeze variables} to refer to the values that the 
variables $\AgdaField{i}$ and $\AgdaField{j}$ had at the beginning of the 
program. In \textsc{Trip}, ordinary Agda variables, acting as metavariables for 
\textsc{Trip}, can fulfil this purpose:
\begin{code}
\>[2]\AgdaFunction{swp}\AgdaSpace{}%
\AgdaSymbol{:}\AgdaSpace{}%
\AgdaSymbol{∀\{}\AgdaBound{I}\AgdaSpace{}%
\AgdaBound{J}\AgdaSpace{}%
\AgdaSymbol{:}\AgdaSpace{}%
\AgdaDatatype{ℕ}\AgdaSymbol{\}}\AgdaSpace{}%
\AgdaSymbol{→}\AgdaSpace{}%
\AgdaOperator{\AgdaDatatype{[}}\AgdaSpace{}%
\AgdaField{i}\AgdaSpace{}%
\AgdaOperator{\AgdaDatatype{≡}}\AgdaSpace{}%
\AgdaBound{I}\AgdaSpace{}%
\AgdaOperator{\AgdaFunction{×}}\AgdaSpace{}%
\AgdaField{j}\AgdaSpace{}%
\AgdaOperator{\AgdaDatatype{≡}}\AgdaSpace{}%
\AgdaBound{J}\AgdaSpace{}%
\AgdaOperator{\AgdaDatatype{,}}\AgdaSpace{}%
\AgdaField{j}\AgdaSpace{}%
\AgdaOperator{\AgdaDatatype{≡}}\AgdaSpace{}%
\AgdaBound{I}\AgdaSpace{}%
\AgdaOperator{\AgdaFunction{×}}\AgdaSpace{}%
\AgdaField{i}\AgdaSpace{}%
\AgdaOperator{\AgdaDatatype{≡}}\AgdaSpace{}%
\AgdaBound{J}\AgdaSpace{}%
\AgdaOperator{\AgdaDatatype{]}}\<%
\end{code}
Unfortunately, the syntax for state updates in our implementation still leaves much to be 
desired.
 Also, the 
proof term $p$ can be seen in the \AgdaFunction{structure} section, despite 
the fact that we do not want to nest proofs within the program structure:
\begin{code}
\>[2]\AgdaFunction{swp}\AgdaSpace{}%
\AgdaSymbol{=}%
\>[1956I]\AgdaOperator{\AgdaFunction{structure:}}\<%
\\
\>[13]\AgdaFunction{upd}\AgdaSpace{}%
\AgdaSymbol{(λ}\AgdaSpace{}%
\AgdaSymbol{\{}\AgdaSpace{}%
\AgdaSymbol{(}\AgdaBound{σ}\AgdaSpace{}%
\AgdaOperator{\AgdaInductiveConstructor{,}}\AgdaSpace{}%
\AgdaBound{p}\AgdaSymbol{)}\AgdaSpace{}%
\AgdaSymbol{→}\AgdaSpace{}%
\AgdaKeyword{record}\AgdaSpace{}%
\AgdaBound{σ}\AgdaSpace{}%
\AgdaSymbol{\{}\AgdaSpace{}%
\AgdaField{temp}\AgdaSpace{}%
\AgdaSymbol{=}\AgdaSpace{}%
\AgdaField{i}%
\>[58]\AgdaSymbol{⦃}\AgdaSpace{}%
\AgdaBound{σ}\AgdaSpace{}%
\AgdaSymbol{⦄}\AgdaSpace{}%
\AgdaSymbol{\}}\AgdaSpace{}%
\AgdaOperator{\AgdaInductiveConstructor{,}}\AgdaSpace{}%
\AgdaBound{p}\AgdaSpace{}%
\AgdaSymbol{\})}\AgdaSpace{}%
\AgdaOperator{\AgdaFunction{﹕}}\<%
\\
%
\>[13]\AgdaFunction{upd}\AgdaSpace{}%
\AgdaSymbol{(λ}\AgdaSpace{}%
\AgdaSymbol{\{}\AgdaSpace{}%
\AgdaSymbol{(}\AgdaBound{σ}\AgdaSpace{}%
\AgdaOperator{\AgdaInductiveConstructor{,}}\AgdaSpace{}%
\AgdaBound{p}\AgdaSymbol{)}\AgdaSpace{}%
\AgdaSymbol{→}\AgdaSpace{}%
\AgdaKeyword{record}\AgdaSpace{}%
\AgdaBound{σ}\AgdaSpace{}%
\AgdaSymbol{\{}\AgdaSpace{}%
\AgdaField{i}%
\>[51]\AgdaSymbol{=}\AgdaSpace{}%
\AgdaField{j}%
\>[58]\AgdaSymbol{⦃}\AgdaSpace{}%
\AgdaBound{σ}\AgdaSpace{}%
\AgdaSymbol{⦄}\AgdaSpace{}%
\AgdaSymbol{\}}\AgdaSpace{}%
\AgdaOperator{\AgdaInductiveConstructor{,}}\AgdaSpace{}%
\AgdaBound{p}\AgdaSpace{}%
\AgdaSymbol{\})}\AgdaSpace{}%
\AgdaOperator{\AgdaFunction{﹕}}\<%
\\
%
\>[13]\AgdaFunction{upd}\AgdaSpace{}%
\AgdaSymbol{(λ}\AgdaSpace{}%
\AgdaSymbol{\{}\AgdaSpace{}%
\AgdaSymbol{(}\AgdaBound{σ}\AgdaSpace{}%
\AgdaOperator{\AgdaInductiveConstructor{,}}\AgdaSpace{}%
\AgdaBound{p}\AgdaSymbol{)}\AgdaSpace{}%
\AgdaSymbol{→}\AgdaSpace{}%
\AgdaKeyword{record}\AgdaSpace{}%
\AgdaBound{σ}\AgdaSpace{}%
\AgdaSymbol{\{}\AgdaSpace{}%
\AgdaField{j}%
\>[51]\AgdaSymbol{=}\AgdaSpace{}%
\AgdaField{temp}\AgdaSpace{}%
\AgdaSymbol{⦃}\AgdaSpace{}%
\AgdaBound{σ}\AgdaSpace{}%
\AgdaSymbol{⦄}\AgdaSpace{}%
\AgdaSymbol{\}}\AgdaSpace{}%
\AgdaOperator{\AgdaInductiveConstructor{,}}\AgdaSpace{}%
\AgdaBound{p}\AgdaSpace{}%
\AgdaSymbol{\})}\<%
\\
\>[.][@{}l@{}]\<[1956I]%
\>[13]\AgdaOperator{\AgdaFunction{proofs:}}\AgdaSpace{}%
\AgdaInductiveConstructor{[]}\<%
\\
%
\>[13]\AgdaOperator{\AgdaFunction{done}}\<%
\end{code}
\subsection{Record update macros}

Ideally, we would like to write simple assignment statements 
such as $\AgdaField{temp}\ \AgdaMacro{\ensuremath{:=}}\ \AgdaField{i}$ rather than 
the syntactically noisy record update syntax built in to Agda. Because Agda's 
record system does not support bidirectional first class accessors, however, 
we cannot define such a statement as an ordinary Agda definition. Instead, 
we must turn to meta-programming.

Agda's meta-programming facilities directly expose the implementation of
parts of the Agda type checker and elaborator behind an interface in a 
monad called $\AgdaDatatype{TC}$. This feature is similar to 
elaborator reflection in Idris~\citep{elab1} or tactic metaprogramming in Lean~\citep{leantac}.
We define a meta-program that, given a field name, generates a setter function 
for the state record which updates that field:\footnote{We omit implementation here 
in the interest of brevity, as it is over 50 lines long.}
\begin{code}
  \>[3]\AgdaFunction{fieldSetter}\AgdaSpace{}%
  \AgdaSymbol{:}\AgdaSpace{}%
  \AgdaSymbol{\{}\AgdaBound{X}\AgdaSpace{}%
  \AgdaSymbol{:}\AgdaSpace{}%
  \AgdaPrimitiveType{Set}\AgdaSymbol{\}}\AgdaSpace{}%
  \AgdaSymbol{→}\AgdaSpace{}%
  \AgdaPostulate{Name}\AgdaSpace{}%
  \AgdaSymbol{→}\AgdaSpace{}%
  \AgdaPostulate{TC}\AgdaSpace{}%
  \AgdaSymbol{(}\AgdaBound{X}\AgdaSpace{}%
  \AgdaSymbol{→}\AgdaSpace{}%
  \AgdaBound{State}\AgdaSpace{}%
  \AgdaSymbol{→}\AgdaSpace{}%
  \AgdaBound{State}\AgdaSymbol{)}\<%
\end{code}
We also define a version of the Hoare logic update
axiom from \autoref{fig:hoare} as a special case of our 
general rule for state updates. This assignment principle is 
parameterised by our setter function and a (possibly state-dependent)
value with which to update our variable:
\begin{code}
\>[3]\AgdaFunction{assn}%
\>[1840I]\AgdaSymbol{:}%
\>[1841I]\AgdaSymbol{\{}\AgdaBound{ϕ}\AgdaSpace{}%
\AgdaSymbol{:}\AgdaSpace{}%
\AgdaFunction{Assertion}\AgdaSymbol{\}}\<%
\\
\>[.][@{}l@{}]\<[1841I]%
\>[10]\AgdaSymbol{\{}\AgdaBound{X}\AgdaSpace{}%
\AgdaSymbol{:}\AgdaSpace{}%
\AgdaPrimitiveType{Set}\AgdaSymbol{\}}\<%
\\
%
\>[10]\AgdaSymbol{(}\AgdaBound{e}\AgdaSpace{}%
\AgdaSymbol{:}\AgdaSpace{}%
\AgdaSymbol{⦃}\AgdaSpace{}%
\AgdaBound{σ}\AgdaSpace{}%
\AgdaSymbol{:}\AgdaSpace{}%
\AgdaBound{State}\AgdaSpace{}%
\AgdaSymbol{⦄}\AgdaSpace{}%
\AgdaSymbol{→}\AgdaSpace{}%
\AgdaBound{X}\AgdaSymbol{)}\<%
\\
%
\>[10]\AgdaSymbol{(}\AgdaBound{set}\AgdaSpace{}%
\AgdaSymbol{:}\AgdaSpace{}%
\AgdaBound{X}\AgdaSpace{}%
\AgdaSymbol{→}\AgdaSpace{}%
\AgdaBound{State}\AgdaSpace{}%
\AgdaSymbol{→}\AgdaSpace{}%
\AgdaBound{State}\AgdaSymbol{)}\<%
\\
\>[.][@{}l@{}]\<[1840I]%
\>[8]\AgdaSymbol{→}\AgdaSpace{}%
\AgdaRecord{Delay}%
\>[1861I]\AgdaOperator{\AgdaDatatype{[}}\AgdaSpace{}%
\AgdaSymbol{(λ}\AgdaSpace{}%
\AgdaSymbol{⦃}\AgdaSpace{}%
\AgdaBound{σ}\AgdaSpace{}%
\AgdaSymbol{⦄}\AgdaSpace{}%
\AgdaSymbol{→}\AgdaSpace{}%
\AgdaBound{ϕ}\AgdaSpace{}%
\AgdaSymbol{⦃}\AgdaSpace{}%
\AgdaBound{set}\AgdaSpace{}%
\AgdaBound{e}\AgdaSpace{}%
\AgdaBound{σ}\AgdaSpace{}%
\AgdaSymbol{⦄)}\<%
\\
\>[.][@{}l@{}]\<[1861I]%
\>[19]\AgdaOperator{\AgdaDatatype{,}}\AgdaSpace{}%
\AgdaSymbol{(λ}\AgdaSpace{}%
\AgdaSymbol{⦃}\AgdaSpace{}%
\AgdaBound{σ}\AgdaSpace{}%
\AgdaSymbol{⦄}\AgdaSpace{}%
\AgdaSymbol{→}\AgdaSpace{}%
\AgdaBound{ϕ}\AgdaSpace{}%
\AgdaSymbol{⦃}\AgdaSpace{}%
\AgdaBound{σ}\AgdaSpace{}%
\AgdaSymbol{⦄)}\AgdaSpace{}%
\AgdaOperator{\AgdaDatatype{]}}\<%
\\
%
\>[3]\AgdaFunction{assn}\AgdaSpace{}%
\AgdaBound{e}\AgdaSpace{}%
\AgdaBound{set}\AgdaSpace{}%
\AgdaSymbol{=}\AgdaSpace{}%
\AgdaFunction{upd}\AgdaSpace{}%
\AgdaSymbol{λ}\AgdaSpace{}%
\AgdaSymbol{\{}\AgdaSpace{}%
\AgdaSymbol{(}\AgdaBound{σ}\AgdaSpace{}%
\AgdaOperator{\AgdaInductiveConstructor{,}}\AgdaSpace{}%
\AgdaBound{p}\AgdaSymbol{)}\AgdaSpace{}%
\AgdaSymbol{→}\AgdaSpace{}%
\AgdaSymbol{(}\AgdaBound{set}\AgdaSpace{}%
\AgdaSymbol{(}\AgdaBound{e}\AgdaSpace{}%
\AgdaSymbol{⦃}\AgdaSpace{}%
\AgdaBound{σ}\AgdaSpace{}%
\AgdaSymbol{⦄)}\AgdaSpace{}%
\AgdaBound{σ}\AgdaSpace{}%
\AgdaOperator{\AgdaInductiveConstructor{,}}\AgdaSpace{}%
\AgdaBound{p}\AgdaSymbol{)}\AgdaSpace{}%
\AgdaSymbol{\}}\<%
\end{code}
Lastly, we define an Agda macro to generate the appropriate code 
for the syntax $a\ \AgdaMacro{\ensuremath{:=}}\ b$. An Agda macro 
is a \AgdaDatatype{TC} procedure that generates a term to unify with 
a hole, which is always the last argument given to the macro. Some 
of the macro's arguments can also be \emph{quoted}, i.e.\ represented
as an explicit syntax tree. In our macro, the only quoted term is the
field name, used to generate the field setter which is in turn used 
as the parameter to our assignment principle:
\begin{code}
\>[3]\AgdaKeyword{macro}\<%
\\
\>[3][@{}l@{\AgdaIndent{0}}]%
\>[4]\AgdaFunction{assnM}%
\>[1904I]\AgdaSymbol{:}%
\>[1905I]\AgdaSymbol{\{}\AgdaBound{ϕ}\AgdaSpace{}%
\AgdaSymbol{:}\AgdaSpace{}%
\AgdaFunction{Assertion}\AgdaSymbol{\}\{}\AgdaBound{X}\AgdaSpace{}%
\AgdaSymbol{:}\AgdaSpace{}%
\AgdaPrimitiveType{Set}\AgdaSymbol{\}}\<%
\\
\>[.][@{}l@{}]\<[1905I]%
\>[12]\AgdaSymbol{(}\AgdaBound{e}\AgdaSpace{}%
\AgdaSymbol{:}%
\>[18]\AgdaSymbol{⦃}\AgdaSpace{}%
\AgdaBound{σ}\AgdaSpace{}%
\AgdaSymbol{:}\AgdaSpace{}%
\AgdaBound{State}\AgdaSpace{}%
\AgdaSymbol{⦄}\AgdaSpace{}%
\AgdaSymbol{→}\AgdaSpace{}%
\AgdaBound{X}\AgdaSymbol{)}\<%
\\
\>[.][@{}l@{}]\<[1904I]%
\>[10]\AgdaSymbol{→}\AgdaSpace{}%
\AgdaPostulate{Name}\<%
\\
%
\>[10]\AgdaSymbol{→}\AgdaSpace{}%
\AgdaDatatype{Term}\AgdaSpace{}%
\AgdaSymbol{→}\AgdaSpace{}%
\AgdaPostulate{TC}\AgdaSpace{}%
\AgdaRecord{⊤}\<%
\\
%
\>[4]\AgdaFunction{assnM}\AgdaSpace{}%
\AgdaSymbol{\{}\AgdaBound{ϕ}\AgdaSymbol{\}}\AgdaSpace{}%
\AgdaBound{e}\AgdaSpace{}%
\AgdaBound{fld}\AgdaSpace{}%
\AgdaBound{hole}\AgdaSpace{}%
\AgdaSymbol{=}\AgdaSpace{}%
\AgdaKeyword{do}\<%
\\
\>[4][@{}l@{\AgdaIndent{0}}]%
\>[8]\AgdaBound{setter}\AgdaSpace{}%
\AgdaOperator{\AgdaFunction{←}}\AgdaSpace{}%
\AgdaFunction{fieldSetter}\AgdaSpace{}%
\AgdaBound{fld}\<%
\\
%
\>[8]\AgdaBound{trm}\AgdaSpace{}%
\AgdaOperator{\AgdaFunction{←}}\AgdaSpace{}%
\AgdaPostulate{quoteTC}\AgdaSpace{}%
\AgdaSymbol{(}\AgdaFunction{assn}\AgdaSpace{}%
\AgdaSymbol{\{}\AgdaBound{ϕ}\AgdaSymbol{\}}\AgdaSpace{}%
\AgdaBound{e}\AgdaSpace{}%
\AgdaBound{setter}\AgdaSymbol{)}\<%
\\
%
\>[8]\AgdaPostulate{unify}\AgdaSpace{}%
\AgdaBound{hole}\AgdaSpace{}%
\AgdaBound{trm}\<%
\\
%
\>[4]\AgdaKeyword{syntax}\AgdaSpace{}%
\AgdaMacro{assnM}\AgdaSpace{}%
\AgdaBound{b}\AgdaSpace{}%
\AgdaBound{a}\AgdaSpace{}%
\AgdaSymbol{=}\AgdaSpace{}%
\AgdaBound{a}\AgdaSpace{}%
\AgdaMacro{≔}\AgdaSpace{}%
\AgdaBound{b}\<%
\end{code}
\subsection{Swap, redux}
With our new assignment macro, our swap procedure is much more 
palatable:
\begin{code}
\>[2]\AgdaFunction{swp$'$}\AgdaSpace{}%
\AgdaSymbol{:}\AgdaSpace{}%
\AgdaSymbol{∀\{}\AgdaBound{I}\AgdaSpace{}%
\AgdaBound{J}\AgdaSpace{}%
\AgdaSymbol{:}\AgdaSpace{}%
\AgdaDatatype{ℕ}\AgdaSymbol{\}}\AgdaSpace{}%
\AgdaSymbol{→}\AgdaSpace{}%
\AgdaOperator{\AgdaDatatype{[}}\AgdaSpace{}%
\AgdaField{i}\AgdaSpace{}%
\AgdaOperator{\AgdaDatatype{≡}}\AgdaSpace{}%
\AgdaBound{I}\AgdaSpace{}%
\AgdaOperator{\AgdaFunction{×}}\AgdaSpace{}%
\AgdaField{j}\AgdaSpace{}%
\AgdaOperator{\AgdaDatatype{≡}}\AgdaSpace{}%
\AgdaBound{J}\AgdaSpace{}%
\AgdaOperator{\AgdaDatatype{,}}\AgdaSpace{}%
\AgdaField{j}\AgdaSpace{}%
\AgdaOperator{\AgdaDatatype{≡}}\AgdaSpace{}%
\AgdaBound{I}\AgdaSpace{}%
\AgdaOperator{\AgdaFunction{×}}\AgdaSpace{}%
\AgdaField{i}\AgdaSpace{}%
\AgdaOperator{\AgdaDatatype{≡}}\AgdaSpace{}%
\AgdaBound{J}\AgdaSpace{}%
\AgdaOperator{\AgdaDatatype{]}}\<%
\\
%
\>[2]\AgdaFunction{swp$'$}\AgdaSpace{}%
\AgdaSymbol{=}%
\>[2090I]\AgdaOperator{\AgdaFunction{structure:}}\<%
\\
\>[2090I][@{}l@{\AgdaIndent{0}}]%
\>[16]\AgdaField{temp}\AgdaSpace{}%
\AgdaMacro{≔}\AgdaSpace{}%
\AgdaField{i}\AgdaSpace{}%
\AgdaOperator{\AgdaFunction{﹕}}\AgdaSpace{}%
\AgdaField{i}\AgdaSpace{}%
\AgdaMacro{≔}\AgdaSpace{}%
\AgdaField{j}\AgdaSpace{}%
\AgdaOperator{\AgdaFunction{﹕}}\AgdaSpace{}%
\AgdaField{j}\AgdaSpace{}%
\AgdaMacro{≔}\AgdaSpace{}%
\AgdaField{temp}\<%
\\
\>[.][@{}l@{}]\<[2090I]%
\>[12]\AgdaOperator{\AgdaFunction{proofs:}}\AgdaSpace{}%
\AgdaInductiveConstructor{[]}\<%
\\
%
\>[12]\AgdaOperator{\AgdaFunction{done}}\<%
\end{code}
Because our \AgdaDatatype{assn} principle once again allows 
(weakest) pre-conditions to be mechanically derived from the 
post-condition, no in-program assertions or proofs are necessary
here.
\subsection{Guarded assignments}
\begin{figure*}
\begin{code}
\>[2]\AgdaFunction{lsum}\AgdaSpace{}%
\AgdaSymbol{:}\AgdaSpace{}%
\AgdaOperator{\AgdaDatatype{[}}\AgdaSpace{}%
\AgdaRecord{⊤}\AgdaSpace{}%
\AgdaOperator{\AgdaDatatype{,}}\AgdaSpace{}%
\AgdaField{result}\AgdaSpace{}%
\AgdaOperator{\AgdaDatatype{≡}}\AgdaSpace{}%
\AgdaFunction{sum}\AgdaSpace{}%
\AgdaField{arr}\AgdaSpace{}%
\AgdaOperator{\AgdaDatatype{]}}\<%
\\
%
\>[2]\AgdaFunction{lsum}\AgdaSpace{}%
\AgdaSymbol{=}\<%
\end{code}\\[-0.9em]
\begin{tikzpicture}
  \node[anchor=north west, text width=\textwidth] at (0,0) {
\begin{code}
\>[2][@{}l@{\AgdaIndent{0}}]%
\>[7]\AgdaOperator{\AgdaFunction{structure:}}\<%
\\
\>[7][@{}l@{\AgdaIndent{0}}]%
\>[9]\AgdaFunction{assert}\AgdaSpace{}%
\AgdaRecord{⊤}\AgdaSpace{}%
\AgdaOperator{\AgdaFunction{﹕}}\<%
\\
%
\>[9]\AgdaField{i}\AgdaSpace{}%
\AgdaMacro{≔}\AgdaSpace{}%
\AgdaNumber{0}\AgdaSpace{}%
\AgdaOperator{\AgdaFunction{﹕}}\<%
\\
%
\>[9]\AgdaField{total}\AgdaSpace{}%
\AgdaMacro{≔}\AgdaSpace{}%
\AgdaNumber{0}\AgdaSpace{}%
\AgdaOperator{\AgdaFunction{﹕}}\<%
\\
%
\>[9]\AgdaFunction{assert}\AgdaSpace{}%
\AgdaSymbol{(}\AgdaField{i}\AgdaSpace{}%
\AgdaOperator{\AgdaDatatype{≡}}\AgdaSpace{}%
\AgdaNumber{0}\AgdaSpace{}%
\AgdaOperator{\AgdaFunction{×}}\AgdaSpace{}%
\AgdaField{total}\AgdaSpace{}%
\AgdaOperator{\AgdaDatatype{≡}}\AgdaSpace{}%
\AgdaNumber{0}\AgdaSymbol{)}\AgdaSpace{}%
\AgdaOperator{\AgdaFunction{﹕}}\<%
\\
%
\>[9]\AgdaOperator{\AgdaFunction{while}}\AgdaSpace{}%
\AgdaSymbol{(}\AgdaField{i}\AgdaSpace{}%
\AgdaOperator{\AgdaFunction{<}}\AgdaSpace{}%
\AgdaFunction{length}\AgdaSpace{}%
\AgdaField{arr}\AgdaSymbol{)}\AgdaSpace{}%
\AgdaOperator{\AgdaFunction{begin}}\<%
\\
\>[9][@{}l@{\AgdaIndent{0}}]%
\>[11]\AgdaField{total}\AgdaSpace{}%
\AgdaMacro{≔}\AgdaSpace{}%
\AgdaField{total}\AgdaSpace{}%
\AgdaOperator{\AgdaPrimitive{+}}\AgdaSpace{}%
\AgdaSymbol{(}\AgdaField{arr}\AgdaSpace{}%
\AgdaOperator{\AgdaFunction{!}}\AgdaSpace{}%
\AgdaBound{i<len}\AgdaSymbol{)}\<%
\\
\>[11]$\quad$\AgdaMacro{given}\AgdaSpace{}%
\AgdaBound{i<len}\AgdaSpace{}%
\AgdaMacro{∶}\AgdaSpace{}%
\AgdaSymbol{(}\AgdaField{i}\AgdaSpace{}%
\AgdaOperator{\AgdaFunction{<}}\AgdaSpace{}%
\AgdaFunction{length}\AgdaSpace{}%
\AgdaField{arr}\AgdaSymbol{)}\AgdaSpace{}%
\AgdaOperator{\AgdaFunction{﹕}}\<%
\\
%
\>[11]\AgdaField{i}\AgdaSpace{}%
\AgdaMacro{≔}\AgdaSpace{}%
\AgdaSymbol{(}\AgdaNumber{1}\AgdaSpace{}%
\AgdaOperator{\AgdaPrimitive{+}}\AgdaSpace{}%
\AgdaField{i}\AgdaSymbol{)}\<%
\\
%
\>[9]\AgdaOperator{\AgdaFunction{end}}\AgdaSpace{}%
\AgdaOperator{\AgdaFunction{﹕}}\<%
\\
%
\>[9]\AgdaFunction{assert}\AgdaSpace{}%
\AgdaSymbol{(}\AgdaOperator{\AgdaFunction{¬}}\AgdaSpace{}%
\AgdaField{i}\AgdaSpace{}%
\AgdaOperator{\AgdaFunction{<}}\AgdaSpace{}%
\AgdaFunction{length}\AgdaSpace{}%
\AgdaField{arr}\<%
\\
\>[11]$\quad$\AgdaOperator{\AgdaFunction{×}}\AgdaSpace{}%
\AgdaField{total}\AgdaSpace{}%
\AgdaOperator{\AgdaDatatype{≡}}\AgdaSpace{}%
\AgdaFunction{sum}\AgdaSpace{}%
\AgdaSymbol{(}\AgdaFunction{take}\AgdaSpace{}%
\AgdaField{i}\AgdaSpace{}%
\AgdaField{arr}\AgdaSymbol{)}\AgdaSpace{}\<%
\\
\>[11]$\quad$\AgdaOperator{\AgdaFunction{×}}\AgdaSpace{}%
\AgdaField{i}\AgdaSpace{}%
\AgdaOperator{\AgdaDatatype{≤}}\AgdaSpace{}%
\AgdaFunction{length}\AgdaSpace{}%
\AgdaField{arr}\AgdaSpace{}%
\AgdaSymbol{)}\<%
\end{code}
  };
\node[text width=0.5\textwidth, anchor=north west] at (9,0) { 
\begin{code}
\>[7]\AgdaOperator{\AgdaFunction{proofs:}}\<%
\\
\>[7][@{}l@{\AgdaIndent{0}}]%
\>[11]\AgdaSymbol{(λ}\AgdaSpace{}%
\AgdaBound{x}\AgdaSpace{}%
\AgdaSymbol{→}\AgdaSpace{}%
\AgdaInductiveConstructor{refl}\AgdaSpace{}%
\AgdaOperator{\AgdaInductiveConstructor{,}}\AgdaSpace{}%
\AgdaInductiveConstructor{refl}\AgdaSymbol{)}\<%
\\
\>[7][@{}l@{\AgdaIndent{0}}]%
\>[9]\AgdaOperator{\AgdaInductiveConstructor{∷}}\AgdaSpace{}%
\AgdaSymbol{(λ}\AgdaSpace{}%
\AgdaSymbol{\{}\AgdaSpace{}%
\AgdaSymbol{(}\AgdaInductiveConstructor{refl}\AgdaSpace{}%
\AgdaOperator{\AgdaInductiveConstructor{,}}\AgdaSpace{}%
\AgdaInductiveConstructor{refl}\AgdaSymbol{)}\AgdaSpace{}%
\AgdaSymbol{→}\AgdaSpace{}%
\AgdaInductiveConstructor{refl}\AgdaSpace{}%
\AgdaOperator{\AgdaInductiveConstructor{,}}\AgdaSpace{}%
\AgdaInductiveConstructor{z≤n}\AgdaSpace{}%
\AgdaSymbol{\})}\<%
\\
%
\>[9]\AgdaOperator{\AgdaInductiveConstructor{∷}}\AgdaSpace{}%
\AgdaSymbol{(λ}\AgdaSpace{}%
\AgdaSymbol{\{}\AgdaSpace{}%
\AgdaSymbol{(}\AgdaBound{i<len}\AgdaSpace{}%
\AgdaOperator{\AgdaInductiveConstructor{,}}\AgdaSpace{}%
\AgdaBound{r≡sumᵢ}\AgdaSpace{}%
\AgdaOperator{\AgdaInductiveConstructor{,}}\AgdaSpace{}%
\AgdaBound{i≤len}\AgdaSymbol{)}\AgdaSpace{}%
\AgdaSymbol{→}\AgdaSpace{}%
\AgdaBound{i<len}\AgdaSpace{}%
\AgdaSymbol{\})}\<%
\\
%
\>[9]\AgdaOperator{\AgdaInductiveConstructor{∷}}\AgdaSpace{}%
\AgdaSymbol{(λ}%
\>[2355I]\AgdaSymbol{\{}\AgdaSpace{}%
\AgdaSymbol{(\AgdaUnderscore{}}\AgdaSpace{}%
\AgdaOperator{\AgdaInductiveConstructor{,}}\AgdaSpace{}%
\AgdaBound{r≡sumᵢ}\AgdaSpace{}%
\AgdaOperator{\AgdaInductiveConstructor{,}}\AgdaSpace{}%
\AgdaBound{i≤len}\AgdaSymbol{)}\AgdaSpace{}%
\AgdaBound{i<len}\AgdaSpace{}%
\AgdaSymbol{→}\<%
\\
\>[.][@{}l@{}]\<[2355I]%
\>[14]\AgdaSymbol{(}\AgdaSpace{}%
\AgdaOperator{\AgdaFunction{begin}}%
\>[2364I]\AgdaField{total}\AgdaSpace{}%
\AgdaOperator{\AgdaPrimitive{+}}\AgdaSpace{}%
\AgdaSymbol{(}\AgdaField{arr}\AgdaSpace{}%
\AgdaOperator{\AgdaFunction{!}}\AgdaSpace{}%
\AgdaBound{i<len}\AgdaSymbol{)}%
\>[55]\AgdaOperator{\AgdaFunction{≡⟨}}\AgdaSpace{}%
$\cdots$
\AgdaOperator{\AgdaFunction{⟩}}\<%
\\
\>[.][@{}l@{}]\<[2364I]%
\>[22]\AgdaFunction{sum}\AgdaSpace{}%
\AgdaSymbol{(}\AgdaFunction{take}\AgdaSpace{}%
\AgdaField{i}\AgdaSpace{}%
\AgdaField{arr}\AgdaSymbol{)}\AgdaSpace{}%
\AgdaOperator{\AgdaPrimitive{+}}\AgdaSpace{}%
\AgdaSymbol{(}\AgdaField{arr}\AgdaSpace{}%
\AgdaOperator{\AgdaFunction{!}}\AgdaSpace{}%
\AgdaBound{i<len}\AgdaSymbol{)}\AgdaSpace{}%
\>[55]\AgdaOperator{\AgdaFunction{≡⟨}}\AgdaSpace{}%
$\cdots$
\AgdaOperator{\AgdaFunction{⟩}}\<%
\\
%
\>[22]\AgdaFunction{sum}\AgdaSpace{}%
\AgdaSymbol{(}\AgdaFunction{take}\AgdaSpace{}%
\AgdaSymbol{(}\AgdaNumber{1}\AgdaSpace{}%
\AgdaOperator{\AgdaPrimitive{+}}\AgdaSpace{}%
\AgdaField{i}\AgdaSymbol{)}\AgdaSpace{}%
\AgdaField{arr}\AgdaSymbol{)}%
\>[55]\AgdaOperator{\AgdaFunction{∎}}\AgdaSymbol{)}\<%
\\
%
\>[14]\AgdaOperator{\AgdaInductiveConstructor{,}}\AgdaSpace{}%
\AgdaBound{i<len}
\AgdaSymbol{\}}\AgdaSpace{}%
\AgdaSymbol{)}\<%
\\
%
\>[9]\AgdaOperator{\AgdaInductiveConstructor{∷}}\AgdaSpace{}%
\AgdaSymbol{(λ}%
\>[2393I]\AgdaSymbol{\{}\AgdaSpace{}%
\AgdaSymbol{(}\AgdaBound{¬i<len}\AgdaSpace{}%
\AgdaOperator{\AgdaInductiveConstructor{,}}\AgdaSpace{}%
\AgdaBound{r≡sumᵢ}\AgdaSpace{}%
\AgdaOperator{\AgdaInductiveConstructor{,}}\AgdaSpace{}%
\AgdaBound{i≤len}\AgdaSymbol{)}\<%
\\
\>[.][@{}l@{}]\<[2393I]%
\>[14]\AgdaSymbol{→}\AgdaSpace{}%
\AgdaOperator{\AgdaFunction{begin}}%
\>[2400I]\AgdaField{total}%
\>[50]\AgdaOperator{\AgdaFunction{≡⟨}}\AgdaSpace{}%
$\cdots$
\AgdaOperator{\AgdaFunction{⟩}}\<%
\\
\>[.][@{}l@{}]\<[2400I]%
\>[22]\AgdaFunction{sum}\AgdaSpace{}%
\AgdaSymbol{(}\AgdaFunction{take}\AgdaSpace{}%
\AgdaField{i}\AgdaSpace{}%
\AgdaField{arr}\AgdaSymbol{)}%
\>[50]\AgdaOperator{\AgdaFunction{≡⟨}}\AgdaSpace{}%
$\cdots$
\AgdaOperator{\AgdaFunction{⟩}}\<%
\\
%
\>[22]\AgdaFunction{sum}\AgdaSpace{}%
\AgdaSymbol{(}\AgdaFunction{take}\AgdaSpace{}%
\AgdaSymbol{(}\AgdaFunction{length}\AgdaSpace{}%
\AgdaField{arr}\AgdaSymbol{)}\AgdaSpace{}%
\AgdaField{arr}\AgdaSymbol{)}\AgdaSpace{}%
\>[50]\AgdaOperator{\AgdaFunction{≡⟨}}\AgdaSpace{}%
$\cdots$
\AgdaOperator{\AgdaFunction{⟩}}\<%
\\
%
\>[22]\AgdaFunction{sum}\AgdaSpace{}%
\AgdaField{arr}%
\>[50]\AgdaOperator{\AgdaFunction{∎}}%
\AgdaSymbol{\}}\AgdaSpace{}%
\AgdaSymbol{)}\AgdaSpace{}\AgdaOperator{\AgdaInductiveConstructor{∷}}\AgdaSpace{}%
\AgdaInductiveConstructor{[]}\<%
\\
\end{code}
};
\draw[->, thick] (2.3,-1.05) -- (9.2,-1.05);
\draw[->, thick, rounded corners = 0.5em] (4.9,-2.45) -- (6,-2.45) |- (9.2,-1.5);
\draw[->, thick, rounded corners = 0.5em] (5.9,-3.85) -- (8,-3.85) |- (9.2,-1.95);
\draw[->, thick, rounded corners = 0.5em] (5.9,-3.85) -- (8,-3.85) |- (9.2,-2.45);
\draw[->, thick, rounded corners = 0.5em] (5.6,-5.65) -- (7,-5.65) |- (9.2,-4.7);
\end{tikzpicture}
\caption{Verified list sum. Our verification condition generator in action.}
\label{fig:vcg}
\end{figure*}
Suppose we now wished to write and verify program that would compute the sum of
all the elements in a list. Our state would include the list itself, the current 
\AgdaField{total}, and a loop counter \AgdaField{i}:
\begin{code}
\>[2]\AgdaKeyword{record}\AgdaSpace{}%
\AgdaRecord{SumState}\AgdaSpace{}%
\AgdaSymbol{:}\AgdaSpace{}%
\AgdaPrimitiveType{Set}\AgdaSpace{}%
\AgdaKeyword{where}\<%
\\
\>[2][@{}l@{\AgdaIndent{0}}]%
\>[4]\AgdaKeyword{field}\<%
\\
\>[4][@{}l@{\AgdaIndent{0}}]%
\>[6]\AgdaField{i}\AgdaSpace{}%
\AgdaSymbol{:}\AgdaSpace{}%
\AgdaDatatype{ℕ}\<%
\\
%
\>[6]\AgdaField{arr}\AgdaSpace{}%
\AgdaSymbol{:}\AgdaSpace{}%
\AgdaDatatype{List}\AgdaSpace{}%
\AgdaDatatype{ℕ}\<%
\\
%
\>[6]\AgdaField{total}\AgdaSpace{}%
\AgdaSymbol{:}\AgdaSpace{}%
\AgdaDatatype{ℕ}\<%
\end{code}
Our imagined program would resemble the following, assuming some 
appropriate list indexing operation $\AgdaFunction{!!}$:
\begin{code}
\>[.][@{}l@{}]\<[2226I]%
\>[9]\AgdaField{i}\AgdaSpace{}%
\AgdaMacro{≔}\AgdaSpace{}%
\AgdaNumber{0}\AgdaSpace{}%
\AgdaOperator{\AgdaFunction{﹕}}\<%
\\
%
\>[9]\AgdaField{total}\AgdaSpace{}%
\AgdaMacro{≔}\AgdaSpace{}%
\AgdaNumber{0}\AgdaSpace{}%
\AgdaOperator{\AgdaFunction{﹕}}\<%
\\
%
\>[9]\AgdaOperator{\AgdaFunction{while}}\AgdaSpace{}%
\AgdaSymbol{(}\AgdaField{i}\AgdaSpace{}%
\AgdaOperator{\AgdaFunction{<}}\AgdaSpace{}%
\AgdaFunction{length}\AgdaSpace{}%
\AgdaField{arr}\AgdaSymbol{)}\AgdaSpace{}%
\AgdaOperator{\AgdaFunction{begin}}\<%
\\
\>[9][@{}l@{\AgdaIndent{0}}]%
%
\>[11]\AgdaField{total}\AgdaSpace{}%
\AgdaMacro{≔}\AgdaSpace{}%
\AgdaSymbol{(}\AgdaField{total}\AgdaSpace{}%
\AgdaOperator{\AgdaPrimitive{+}}\AgdaSpace{}%
\AgdaField{arr}\AgdaSpace{}%
\AgdaOperator{\AgdaPostulate{!!}}\AgdaSpace{}%
\AgdaField{i}\AgdaSymbol{)}\AgdaSpace{}%
\AgdaOperator{\AgdaFunction{﹕}}\<%
\\
%
\>[11]\AgdaField{i}\AgdaSpace{}%
\AgdaMacro{≔}\AgdaSpace{}%
\AgdaSymbol{(}\AgdaNumber{1}\AgdaSpace{}%
\AgdaOperator{\AgdaPrimitive{+}}\AgdaSpace{}%
\AgdaField{i}\AgdaSymbol{)}\<%
\\
%
\>[9]\AgdaOperator{\AgdaFunction{end}}\AgdaSpace{}%
\end{code}
Unfortunately, such a list indexing operation would need a type like:\\[0.5em]
\AgdaOperator{\AgdaPostulate{\AgdaUnderscore{}!!\AgdaUnderscore{}}}\AgdaSpace{}%
\AgdaSymbol{:}\AgdaSpace{}%
\AgdaDatatype{List}\AgdaSpace{}%
\AgdaGeneralizable{A}\AgdaSpace{}%
\AgdaSymbol{→}\AgdaSpace{}%
\AgdaDatatype{ℕ}\AgdaSpace{}%
\AgdaSymbol{→}\AgdaSpace{}%
\AgdaGeneralizable{A}\\[0.5em]
This type is not inhabited in Agda, where function types refer to \emph{total} functions, 
as there is no valid behaviour if the given list index is out of range. Even though 
we know from our assertions that every indexing is in range, the type of the 
function does not reflect this knowledge. Therefore, we shall use a more precise type 
instead, which requires evidence that the index is valid for the list:
\begin{code}
\>[2]\AgdaOperator{\AgdaFunction{\AgdaUnderscore{}!\AgdaUnderscore{}}}\AgdaSpace{}%
\AgdaSymbol{:}\AgdaSpace{}%
\AgdaSymbol{∀}\AgdaSpace{}%:
\AgdaSymbol{\{}\AgdaBound{n}\AgdaSymbol{\}}\AgdaSpace{}%
\AgdaSymbol{→}\AgdaSpace{}%
\AgdaSymbol{(}\AgdaBound{ls}\AgdaSpace{}%
\AgdaSymbol{:}\AgdaSpace{}%
\AgdaDatatype{List}\AgdaSpace{}%
\AgdaGeneralizable{A}\AgdaSymbol{)}\AgdaSpace{}%
\AgdaSymbol{→}\AgdaSpace{}%
\AgdaBound{n}\AgdaSpace{}%
\AgdaOperator{\AgdaFunction{<}}\AgdaSpace{}%
\AgdaFunction{length}\AgdaSpace{}%
\AgdaBound{ls}\AgdaSpace{}%
\AgdaSymbol{→}\AgdaSpace{}%
\AgdaGeneralizable{A}\<%
\\
%
\>[2]\AgdaOperator{\AgdaFunction{\AgdaUnderscore{}!\AgdaUnderscore{}}}\AgdaSpace{}%
\AgdaSymbol{\{}\AgdaArgument{n}\AgdaSpace{}%
\AgdaSymbol{=}\AgdaSpace{}%
\AgdaInductiveConstructor{suc}\AgdaSpace{}%
\AgdaSymbol{\AgdaUnderscore{}\}}\AgdaSpace{}%
\AgdaSymbol{(}\AgdaBound{x}\AgdaSpace{}%
\AgdaOperator{\AgdaInductiveConstructor{∷}}\AgdaSpace{}%
\AgdaBound{ls}\AgdaSymbol{)}\AgdaSpace{}%
\AgdaSymbol{(}\AgdaInductiveConstructor{s≤s}\AgdaSpace{}%
\AgdaBound{n}\AgdaSymbol{)}\AgdaSpace{}%
\AgdaSymbol{=}\AgdaSpace{}%
\AgdaBound{ls}\AgdaSpace{}%
\AgdaOperator{\AgdaFunction{!}}\AgdaSpace{}%
\AgdaBound{n}\<%
\\
%
\>[2]\AgdaOperator{\AgdaFunction{\AgdaUnderscore{}!\AgdaUnderscore{}}}\AgdaSpace{}%
\AgdaSymbol{\{}\AgdaArgument{n}\AgdaSpace{}%
\AgdaSymbol{=}\AgdaSpace{}%
\AgdaInductiveConstructor{zero}\AgdaSpace{}%
\AgdaSymbol{\}}\AgdaSpace{}%
\AgdaSymbol{(}\AgdaBound{x}\AgdaSpace{}%
\AgdaOperator{\AgdaInductiveConstructor{∷}}\AgdaSpace{}%
\AgdaBound{ls}\AgdaSymbol{)}\AgdaSpace{}%
\AgdaSymbol{(}\AgdaInductiveConstructor{s≤s}\AgdaSpace{}%
\AgdaBound{n}\AgdaSymbol{)}\AgdaSpace{}%
\AgdaSymbol{=}\AgdaSpace{}%
\AgdaBound{x}\<%
\end{code}
This corrected indexing operation presents us with a problem, however, 
because it means that the expression used to update the \AgdaField{total}
in our example now needs access to a \emph{proof} that the given 
index is less than the length of the list. We know that the index is in 
bounds from our assertions, but there is no means in our current syntax 
to get a proof of that fact into a variable assignment.

Therefore, we introduce a more general form of assignment syntax, called 
a \emph{guarded assignment}. With it, we could write our \AgdaField{total}
update as:
\begin{code}
\>[9]\AgdaField{total}\AgdaSpace{}%
\AgdaMacro{≔}\AgdaSpace{}%
\AgdaField{total}\AgdaSpace{}%
\AgdaOperator{\AgdaPrimitive{+}}\AgdaSpace{}%
\AgdaSymbol{(}\AgdaField{arr}\AgdaSpace{}%
\AgdaOperator{\AgdaFunction{!}}\AgdaSpace{}%
\AgdaBound{i<len}\AgdaSymbol{)}\AgdaSpace{}%
\AgdaMacro{given}\AgdaSpace{}%
\AgdaBound{i<len}\AgdaSpace{}%
\AgdaMacro{∶}\AgdaSpace{}%
\AgdaSymbol{(}\AgdaField{i}\AgdaSpace{}%
\AgdaOperator{\AgdaFunction{<}}\AgdaSpace{}%
\AgdaFunction{length}\AgdaSpace{}%
\AgdaField{arr}\AgdaSymbol{)}\AgdaSpace{}%
\end{code}
This additional assertion  
\AgdaSymbol{(}\AgdaField{i}\AgdaSpace{}%
\AgdaOperator{\AgdaFunction{<}}\AgdaSpace{}%
\AgdaFunction{length}\AgdaSpace{}%
\AgdaField{arr}\AgdaSymbol{)}\AgdaSpace{}%
, called the guard, is proven to follow from the statement's pre-condition as a
deferred obligation, separately from the program.


\begin{figure*}
\begin{tikzpicture}
  \node[anchor=north west, text width=\textwidth] at (0,0) {
\begin{code}
\>[2]\AgdaFunction{csum}%
\>[3167I]\AgdaSymbol{:}\AgdaSpace{}%
\AgdaSymbol{∀\{}\AgdaBound{arr₀}\AgdaSymbol{\}}\<%
\\
\>[.][@{}l@{}]\<[3167I]%
\>[10]\AgdaSymbol{→}%
\>[3169I]\AgdaOperator{\AgdaDatatype{[}}\AgdaSpace{}%
\AgdaField{arr}\AgdaSpace{}%
\AgdaOperator{\AgdaDatatype{≡}}\AgdaSpace{}%
\AgdaBound{arr₀}\AgdaSpace{}%
\AgdaOperator{\AgdaFunction{×}}\AgdaSpace{}%
\AgdaFunction{length}\AgdaSpace{}%
\AgdaField{arr}\AgdaSpace{}%
\AgdaOperator{\AgdaFunction{>}}\AgdaSpace{}%
\AgdaNumber{0}\<%
\\
\>[.][@{}l@{}]\<[3169I]%
\>[12]\AgdaOperator{\AgdaDatatype{,}}\AgdaSpace{}%
\AgdaFunction{length}\AgdaSpace{}%
\AgdaField{arr}\AgdaSpace{}%
\AgdaOperator{\AgdaDatatype{≡}}\AgdaSpace{}%
\AgdaFunction{length}\AgdaSpace{}%
\AgdaBound{arr₀}\<%
\\
%
\>[12]\AgdaOperator{\AgdaFunction{×}}\AgdaSpace{}%
\AgdaSymbol{(∀}\AgdaSpace{}%
\AgdaBound{i}\AgdaSpace{}%
\AgdaSymbol{(}\AgdaBound{i}\AgdaSpace{}%
\AgdaOperator{\AgdaFunction{<}}\AgdaSpace{}%
\AgdaFunction{length}\AgdaSpace{}%
\AgdaField{arr}\AgdaSymbol{)}\AgdaSpace{}%
\\
\>[12]$\quad$\AgdaSymbol{→}\AgdaSpace{}%
\AgdaField{arr}\AgdaSpace{}%
\AgdaOperator{\AgdaDatatype{[}}\AgdaSpace{}%
\AgdaBound{i}\AgdaSpace{}%
\AgdaOperator{\AgdaDatatype{]=}}\AgdaSpace{}%
\AgdaFunction{sum}\AgdaSpace{}%
\AgdaSymbol{(}\AgdaFunction{take}\AgdaSpace{}%
\AgdaSymbol{(}\AgdaInductiveConstructor{suc}\AgdaSpace{}%
\AgdaBound{i}\AgdaSymbol{)}\AgdaSpace{}%
\AgdaBound{arr₀}\AgdaSymbol{)}\AgdaSpace{}%
\AgdaSymbol{)}\<%
\\
%
\>[12]\AgdaOperator{\AgdaDatatype{]}}\<%
\\
%
\>[2]\AgdaFunction{csum}\AgdaSpace{}%
\AgdaSymbol{\{}\AgdaBound{arr₀}\AgdaSymbol{\}}\AgdaSpace{}%
\AgdaSymbol{=}\AgdaSpace{}%
\AgdaKeyword{let}\<%
\\
\>[2][@{}l@{\AgdaIndent{0}}]%
\>[7]\AgdaBound{Inv}\AgdaSpace{}%
\AgdaSymbol{:}\AgdaSpace{}%
\AgdaDatatype{ℕ}\AgdaSpace{}%
\AgdaSymbol{→}\AgdaSpace{}%
\AgdaDatatype{ℕ}\AgdaSpace{}%
\AgdaSymbol{→}\AgdaSpace{}%
\AgdaFunction{Assertion}\<%
\\
%
\>[7]\AgdaBound{Inv}\AgdaSpace{}%
\AgdaBound{i}\AgdaSpace{}%
\AgdaBound{i$'$}\AgdaSpace{}%
\AgdaSymbol{⦃}\AgdaSpace{}%
\AgdaBound{σ}\AgdaSpace{}%
\AgdaSymbol{⦄}%
\>[3215I]\AgdaSymbol{=}\AgdaSpace{}%
\AgdaSymbol{(}\AgdaFunction{length}\AgdaSpace{}%
\AgdaField{arr}\AgdaSpace{}%
\AgdaOperator{\AgdaDatatype{≡}}\AgdaSpace{}%
\AgdaFunction{length}\AgdaSpace{}%
\AgdaBound{arr₀}\AgdaSymbol{)}\<%
\\
\>[.][@{}l@{}]\<[3215I]%
\>[22]\AgdaOperator{\AgdaFunction{×}}\AgdaSpace{}%
\AgdaSymbol{(}\AgdaBound{i}\AgdaSpace{}%
\AgdaOperator{\AgdaDatatype{≤}}\AgdaSpace{}%
\AgdaFunction{length}\AgdaSpace{}%
\AgdaField{arr}\AgdaSymbol{)}\<%
\\
%
\>[22]\AgdaOperator{\AgdaFunction{×}}\AgdaSpace{}%
\AgdaSymbol{(∀}\AgdaSpace{}%
\AgdaBound{j}\AgdaSpace{}%
\AgdaSymbol{(}\AgdaBound{j<i}\AgdaSpace{}%
\AgdaSymbol{:}\AgdaSpace{}%
\AgdaBound{j}\AgdaSpace{}%
\AgdaOperator{\AgdaFunction{<}}\AgdaSpace{}%
\AgdaBound{i}\AgdaSymbol{)}\AgdaSpace{}\\%
\>[22]$\quad$\AgdaSymbol{→}\AgdaSpace{}%
\AgdaField{arr}\AgdaSpace{}%
\AgdaOperator{\AgdaDatatype{[}}\AgdaSpace{}%
\AgdaBound{j}\AgdaSpace{}%
\AgdaOperator{\AgdaDatatype{]=}}\AgdaSpace{}%
\AgdaFunction{sum}\AgdaSpace{}%
\AgdaSymbol{(}\AgdaFunction{take}\AgdaSpace{}%
\AgdaSymbol{(}\AgdaInductiveConstructor{suc}\AgdaSpace{}%
\AgdaBound{j}\AgdaSymbol{)}\AgdaSpace{}%
\AgdaBound{arr₀}\AgdaSymbol{))}\<%
\\
%
\>[22]\AgdaOperator{\AgdaFunction{×}}%
\>[25]\AgdaFunction{drop}\AgdaSpace{}%
\AgdaBound{i}\AgdaSpace{}%
\AgdaField{arr}\AgdaSpace{}%
\AgdaOperator{\AgdaDatatype{≡}}\AgdaSpace{}%
\AgdaFunction{drop}\AgdaSpace{}%
\AgdaBound{i}\AgdaSpace{}%
\AgdaBound{arr₀}\<%
\\
%
\>[22]\AgdaOperator{\AgdaFunction{×}}\AgdaSpace{}%
\AgdaField{total}\AgdaSpace{}%
\AgdaOperator{\AgdaDatatype{≡}}\AgdaSpace{}%
\AgdaFunction{sum}\AgdaSpace{}%
\AgdaSymbol{(}\AgdaFunction{take}\AgdaSpace{}%
\AgdaBound{i′}\AgdaSpace{}%
\AgdaBound{arr₀}\AgdaSymbol{)}\<%
\end{code}
  };
\node[text width=0.5\textwidth, anchor=north west] at (8.5,0) { 
\begin{code}
\>[4]\AgdaKeyword{in}%
\>[3254I]\AgdaOperator{\AgdaFunction{structure:}}\<%
\\
\>[.][@{}l@{}]\<[3254I]%
\>[7]\AgdaFunction{assert}\AgdaSpace{}%
\AgdaSymbol{(}\AgdaField{arr}\AgdaSpace{}%
\AgdaOperator{\AgdaDatatype{≡}}\AgdaSpace{}%
\AgdaBound{arr₀}\AgdaSpace{}%
\AgdaOperator{\AgdaFunction{×}}\AgdaSpace{}%
\AgdaFunction{length}\AgdaSpace{}%
\AgdaField{arr}\AgdaSpace{}%
\AgdaOperator{\AgdaFunction{>}}\AgdaSpace{}%
\AgdaNumber{0}\AgdaSymbol{)}\AgdaSpace{}%
\AgdaOperator{\AgdaFunction{﹕}}\<%
\\
%
\>[7]\AgdaField{i}\AgdaSpace{}%
\AgdaMacro{≔}\AgdaSpace{}%
\AgdaNumber{0}\AgdaSpace{}%
\AgdaOperator{\AgdaFunction{﹕}}\<%
\\
%
\>[7]\AgdaField{total}\AgdaSpace{}%
\AgdaMacro{≔}\AgdaSpace{}%
\AgdaNumber{0}\AgdaSpace{}%
\AgdaOperator{\AgdaFunction{﹕}}\<%
\\

\>[7]\AgdaFunction{assert}\AgdaSpace{}%
\AgdaSymbol{(}\AgdaField{i}\AgdaSpace{}%
\AgdaOperator{\AgdaDatatype{≡}}\AgdaSpace{}%
\AgdaNumber{0}\AgdaSpace{}%
\AgdaOperator{\AgdaFunction{×}}\AgdaSpace{}%
\AgdaField{total}\AgdaSpace{}%
\AgdaOperator{\AgdaDatatype{≡}}\AgdaSpace{}%
\AgdaNumber{0}\AgdaSpace{}%
\AgdaOperator{\AgdaFunction{×}}\AgdaSpace{}%
\AgdaField{arr}\AgdaSpace{}%
\AgdaOperator{\AgdaDatatype{≡}}\AgdaSpace{}%
\AgdaBound{arr₀}\AgdaSpace{}%
\AgdaOperator{\AgdaFunction{×}}\AgdaSpace{}%
\AgdaField{i}\AgdaSpace{}%
\AgdaOperator{\AgdaFunction{<}}\AgdaSpace{}%
\AgdaFunction{length}\AgdaSpace{}%
\AgdaField{arr}\AgdaSymbol{)}\AgdaSpace{}%
\AgdaOperator{\AgdaFunction{﹕}}\<%
\\
%
\>[7]\AgdaOperator{\AgdaFunction{while}}\AgdaSpace{}%
\AgdaSymbol{(}\AgdaField{i}\AgdaSpace{}%
\AgdaOperator{\AgdaFunction{<}}\AgdaSpace{}%
\AgdaFunction{length}\AgdaSpace{}%
\AgdaField{arr}\AgdaSymbol{)}\AgdaSpace{}%
\AgdaOperator{\AgdaFunction{begin}}\<%
\\
\>[7][@{}l@{\AgdaIndent{0}}]%
\>[9]\AgdaField{total}\AgdaSpace{}%
\AgdaMacro{≔}\AgdaSpace{}%
\AgdaField{total}\AgdaSpace{}%
\AgdaOperator{\AgdaPrimitive{+}}\AgdaSpace{}%
\AgdaSymbol{(}\AgdaField{arr}\AgdaSpace{}%
\AgdaOperator{\AgdaFunction{!}}\AgdaSpace{}%
\AgdaBound{i<len}\AgdaSymbol{)}\AgdaSpace{}\\%
\>[9]$\quad$\AgdaMacro{given}\AgdaSpace{}%
\AgdaBound{i<len}\AgdaSpace{}%
\AgdaMacro{∶}\AgdaSpace{}%
\AgdaSymbol{(}\AgdaField{i}\AgdaSpace{}%
\AgdaOperator{\AgdaFunction{<}}\AgdaSpace{}%
\AgdaFunction{length}\AgdaSpace{}%
\AgdaField{arr}\AgdaSymbol{)}\AgdaSpace{}%
\AgdaOperator{\AgdaFunction{﹕}}\<%
\\
%
\>[9]\AgdaFunction{assert}\AgdaSpace{}%
\AgdaSymbol{(}\AgdaSpace{}%
\AgdaSymbol{(}\AgdaField{i}\AgdaSpace{}%
\AgdaOperator{\AgdaFunction{<}}\AgdaSpace{}%
\AgdaFunction{length}\AgdaSpace{}%
\AgdaField{arr}\AgdaSymbol{)}\AgdaSpace{}%
\AgdaOperator{\AgdaFunction{×}}%
\>[38]\AgdaBound{Inv}\AgdaSpace{}%
\AgdaField{i}\AgdaSpace{}%
\AgdaSymbol{(}\AgdaInductiveConstructor{suc}\AgdaSpace{}%
\AgdaField{i}\AgdaSymbol{))}\AgdaSpace{}%
\AgdaOperator{\AgdaFunction{﹕}}\<%
\\
%
\>[9]\AgdaField{arr}\AgdaSpace{}%
\AgdaMacro{≔}\AgdaSpace{}%
\AgdaSymbol{(}\AgdaField{arr}\AgdaSpace{}%
\AgdaOperator{\AgdaFunction{[}}\AgdaSpace{}%
\AgdaBound{i<len}\AgdaSpace{}%
\AgdaOperator{\AgdaFunction{]≔}}\AgdaSpace{}%
\AgdaField{total}\AgdaSymbol{)}\AgdaSpace{}\\%
\>[9]$\quad$\AgdaMacro{given}\AgdaSpace{}%
\AgdaBound{i<len}\AgdaSpace{}%
\AgdaMacro{∶}\AgdaSpace{}%
\AgdaSymbol{(}\AgdaField{i}\AgdaSpace{}%
\AgdaOperator{\AgdaFunction{<}}\AgdaSpace{}%
\AgdaFunction{length}\AgdaSpace{}%
\AgdaField{arr}\AgdaSymbol{)}\AgdaSpace{}%
\AgdaOperator{\AgdaFunction{﹕}}\<%
\\
%
\>[9]\AgdaFunction{assert}\AgdaSpace{}%
\AgdaSymbol{(}\AgdaBound{Inv}\AgdaSpace{}%
\AgdaSymbol{(}\AgdaInductiveConstructor{suc}\AgdaSpace{}%
\AgdaField{i}\AgdaSymbol{)}\AgdaSpace{}%
\AgdaSymbol{(}\AgdaInductiveConstructor{suc}\AgdaSpace{}%
\AgdaField{i}\AgdaSymbol{))}\AgdaSpace{}%
\AgdaOperator{\AgdaFunction{﹕}}\<%
\\
%
\>[9]\AgdaField{i}\AgdaSpace{}%
\AgdaMacro{≔}\AgdaSpace{}%
\AgdaSymbol{(}\AgdaNumber{1}\AgdaSpace{}%
\AgdaOperator{\AgdaPrimitive{+}}\AgdaSpace{}%
\AgdaField{i}\AgdaSymbol{)}\<%
\\
%
\>[7]\AgdaOperator{\AgdaFunction{end}}\AgdaSpace{}%
\AgdaOperator{\AgdaFunction{﹕}}\<%
\\
%
\>[7]\AgdaFunction{assert}\AgdaSpace{}%
\AgdaSymbol{(}\AgdaOperator{\AgdaFunction{¬}}\AgdaSpace{}%
\AgdaSymbol{(}\AgdaField{i}\AgdaSpace{}%
\AgdaOperator{\AgdaFunction{<}}\AgdaSpace{}%
\AgdaFunction{length}\AgdaSpace{}%
\AgdaField{arr}\AgdaSymbol{)}\AgdaSpace{}%
\AgdaOperator{\AgdaFunction{×}}\AgdaSpace{}%
\AgdaBound{Inv}\AgdaSpace{}%
\AgdaField{i}\AgdaSpace{}%
\AgdaField{i}\AgdaSpace{}%
\AgdaSymbol{)}\<%
\\
%
\>[7]\AgdaOperator{\AgdaFunction{proofs:}} \ensuremath{\langle\ \textit{100 lines of proof}\ \rangle} \AgdaFunction{done}\<%
\end{code}
};
\end{tikzpicture}
\caption{Cumulative sum}
\label{fig:csum}
\end{figure*}
The assignment principle for this syntax is significantly more general 
than our principle for simple assignments (\AgdaFunction{assn}).
Because the update expression 
now depends upon the pre-condition, the pre-condition can no longer 
be merely the post-condition composed with the update. Instead,
we require a proof that the pre-condition $\varphi$ implies the guard $\alpha$,
and a proof that the post-condition $\psi$ holds after the state has been updated:
\begin{code}
\>[3]\AgdaFunction{grd}%
\>[1515I]\AgdaSymbol{:}%
\>[1516I]\AgdaSymbol{\{}\AgdaBound{ϕ}\AgdaSpace{}%
\AgdaBound{ψ}\AgdaSpace{}%
\AgdaBound{α}\AgdaSpace{}%
\AgdaSymbol{:}\AgdaSpace{}%
\AgdaFunction{Assertion}\AgdaSymbol{\}\{}\AgdaBound{X}\AgdaSpace{}%
\AgdaSymbol{:}\AgdaSpace{}%
\AgdaPrimitiveType{Set}\AgdaSymbol{\}}\<%
\\
\>[11]\AgdaSymbol{(}\AgdaBound{e}\AgdaSpace{}%
\AgdaSymbol{:}\AgdaSpace{}%
\AgdaSymbol{⦃}\AgdaSpace{}%
\AgdaBound{σ}\AgdaSpace{}%
\AgdaSymbol{:}\AgdaSpace{}%
\AgdaBound{State}\AgdaSpace{}%
\AgdaSymbol{⦄}\AgdaSpace{}%
\AgdaSymbol{→}\AgdaSpace{}%
\AgdaBound{α}\AgdaSpace{}%
\AgdaSymbol{⦃}\AgdaSpace{}%
\AgdaBound{σ}\AgdaSpace{}%
\AgdaSymbol{⦄}\AgdaSpace{}%
\AgdaSymbol{→}\AgdaSpace{}%
\AgdaBound{X}\AgdaSymbol{)}\<%
\\
%
\>[11]\AgdaSymbol{(}\AgdaBound{set}\AgdaSpace{}%
\AgdaSymbol{:}\AgdaSpace{}%
\AgdaBound{X}\AgdaSpace{}%
\AgdaSymbol{→}\AgdaSpace{}%
\AgdaBound{State}\AgdaSpace{}%
\AgdaSymbol{→}\AgdaSpace{}%
\AgdaBound{State}\AgdaSymbol{)}\<%
\\
\>[.][@{}l@{}]\<[1515I]%
\>[11]\AgdaSymbol{→}\AgdaSpace{}%
\AgdaRecord{Delay}\AgdaSpace{}%
\AgdaSymbol{(}\AgdaOperator{\AgdaFunction{Π:}}\AgdaSpace{}%
\AgdaSymbol{(}\AgdaBound{ϕ}\AgdaSpace{}%
\AgdaSymbol{→}\AgdaSpace{}%
\AgdaBound{α}\AgdaSymbol{))}\<%
\\
%
\>[11]\AgdaSymbol{→}\AgdaSpace{}%
\AgdaRecord{Delay}\AgdaSpace{}%
\AgdaSymbol{(}\AgdaOperator{\AgdaFunction{Π:}}\AgdaSpace{}%
\AgdaSymbol{λ}\AgdaSpace{}%
\AgdaSymbol{⦃}\AgdaSpace{}\!%
\AgdaBound{σ}\AgdaSpace{}\!%
\AgdaSymbol{⦄}%
\>[1556I]\AgdaSymbol{→}\AgdaSpace{}%
\AgdaBound{ϕ}\AgdaSpace{}%
\AgdaSymbol{→}\AgdaSpace{}%
\AgdaSymbol{(}\AgdaBound{g}\AgdaSpace{}%
\AgdaSymbol{:}\AgdaSpace{}%
\AgdaBound{α}\AgdaSymbol{)}\AgdaSpace{}%
\AgdaSymbol{→}\AgdaSpace{}%
\AgdaBound{ψ}\AgdaSpace{}%
\AgdaSymbol{⦃}\AgdaSpace{}\!%
\AgdaBound{set}\AgdaSpace{}%
\AgdaSymbol{(}\AgdaBound{e}\AgdaSpace{}%
\AgdaBound{g}\AgdaSymbol{)}\AgdaSpace{}%
\AgdaBound{σ}\AgdaSpace{}\!%
\AgdaSymbol{⦄)}\<%
\\
%
\>[11]\AgdaSymbol{→}\AgdaSpace{}%
\AgdaRecord{Delay}\AgdaSpace{}%
\AgdaOperator{\AgdaDatatype{[}}\AgdaSpace{}%
\AgdaBound{ϕ}\AgdaSpace{}%
\AgdaOperator{\AgdaDatatype{,}}\AgdaSpace{}%
\AgdaBound{ψ}\AgdaSpace{}%
\AgdaOperator{\AgdaDatatype{]}}\<%
\\
%
\>[3]\AgdaFunction{grd}\AgdaSpace{}%
\AgdaBound{e}\AgdaSpace{}%
\AgdaBound{set}\AgdaSpace{}%
\AgdaBound{p₁}\AgdaSpace{}%
\AgdaBound{p₂}\AgdaSpace{}%
\AgdaSymbol{=}\AgdaSpace{}%
\AgdaSymbol{⦇}\AgdaSpace{}%
\AgdaFunction{update}\AgdaSpace{}%
\AgdaBound{p₁}\AgdaSpace{}%
\AgdaBound{p₂}\AgdaSpace{}%
\AgdaSymbol{⦈}\<%
\\
\>[3][@{}l@{\AgdaIndent{0}}]%
\>[5]\AgdaKeyword{where}%
\>[1585I]\AgdaFunction{update}\AgdaSpace{}%
\AgdaSymbol{=}\AgdaSpace{}%
\AgdaSymbol{λ}\AgdaSpace{}%
\AgdaBound{ϕ→α}\AgdaSpace{}%
\AgdaBound{ϕ→α→ψ}\AgdaSpace{}%
\AgdaSymbol{→}\AgdaSpace{}%
\AgdaInductiveConstructor{UPD}\AgdaSpace{}%
\AgdaSymbol{λ}\AgdaSpace{}%
\AgdaKeyword{where}\<%
\\
\>[1585I][@{}l@{\AgdaIndent{0}}]%
\>[13]\AgdaSymbol{(}\AgdaBound{σ}\AgdaSpace{}%
\AgdaOperator{\AgdaInductiveConstructor{,}}%
\>[1595I]\AgdaBound{ϕ}\AgdaSymbol{)}\AgdaSpace{}%
\AgdaSymbol{→}\AgdaSpace{}%
\AgdaKeyword{let}\<%
\\
\>[.][@{}l@{}]\<[1595I]%
\>[18]\AgdaBound{α}\AgdaSpace{}%
\AgdaSymbol{=}\AgdaSpace{}%
\AgdaBound{ϕ→α}\AgdaSpace{}%
\AgdaSymbol{⦃}\AgdaSpace{}%
\AgdaBound{σ}\AgdaSpace{}%
\AgdaSymbol{⦄}\AgdaSpace{}%
\AgdaBound{ϕ}\<%
\\
%
\>[18]\AgdaBound{ψ}\AgdaSpace{}%
\AgdaSymbol{=}\AgdaSpace{}%
\AgdaBound{ϕ→α→ψ}\AgdaSpace{}%
\AgdaSymbol{⦃}\AgdaSpace{}%
\AgdaBound{σ}\AgdaSpace{}%
\AgdaSymbol{⦄}\AgdaSpace{}%
\AgdaBound{ϕ}\AgdaSpace{}%
\AgdaBound{α}\<%
\\
\>[13][@{}l@{\AgdaIndent{0}}]%
\>[15]\AgdaKeyword{in}\AgdaSpace{}%
\AgdaBound{set}\AgdaSpace{}%
\AgdaSymbol{(}\AgdaBound{e}\AgdaSpace{}%
\AgdaSymbol{⦃}\AgdaSpace{}%
\AgdaBound{σ}\AgdaSpace{}%
\AgdaSymbol{⦄}\AgdaSpace{}%
\AgdaBound{α}\AgdaSymbol{)}\AgdaSpace{}%
\AgdaBound{σ}\AgdaSpace{}%
\AgdaOperator{\AgdaInductiveConstructor{,}}\AgdaSpace{}%
\AgdaBound{ψ}\<%
\end{code}
Our macro for the syntax is broadly similar to the macro for \AgdaFunction{assn},
except that it uses the proof delay applicative to defer proving both of the implications 
until \AgdaFunction{later}:
\begin{code}
\>[4]\AgdaFunction{guardedM}%
\>[1620I]\AgdaSymbol{:}%
\>[1621I]\AgdaSymbol{\{}\AgdaBound{ϕ}\AgdaSpace{}%
\AgdaBound{ψ}\AgdaSpace{}%
\AgdaSymbol{:}\AgdaSpace{}%
\AgdaFunction{Assertion}\AgdaSymbol{\}\{}\AgdaBound{X}\AgdaSpace{}%
\AgdaSymbol{:}\AgdaSpace{}%
\AgdaPrimitiveType{Set}\AgdaSymbol{\}(}\AgdaBound{α}\AgdaSpace{}%
\AgdaSymbol{:}\AgdaSpace{}%
\AgdaFunction{Assertion}\AgdaSymbol{)}\<%
\\
\>[.][@{}l@{}]\<[1621I]%
\>[15]\AgdaSymbol{(}\AgdaBound{e}\AgdaSpace{}%
\AgdaSymbol{:}\AgdaSpace{}%
\AgdaSymbol{⦃}\AgdaSpace{}%
\AgdaBound{σ}\AgdaSpace{}%
\AgdaSymbol{:}\AgdaSpace{}%
\AgdaBound{State}\AgdaSpace{}%
\AgdaSymbol{⦄}\AgdaSpace{}%
\AgdaSymbol{→}\AgdaSpace{}%
\AgdaBound{α}\AgdaSpace{}%
\AgdaSymbol{⦃}\AgdaSpace{}%
\AgdaBound{σ}\AgdaSpace{}%
\AgdaSymbol{⦄}\AgdaSpace{}%
\AgdaSymbol{→}\AgdaSpace{}%
\AgdaBound{X}\AgdaSymbol{)}\<%
\\
\>[.][@{}l@{}]\<[1620I]%
\>[13]\AgdaSymbol{→}\AgdaSpace{}%
\AgdaPostulate{Name}\<%
\\
%
\>[13]\AgdaSymbol{→}\AgdaSpace{}%
\AgdaDatatype{Term}\AgdaSpace{}%
\AgdaSymbol{→}\AgdaSpace{}%
\AgdaPostulate{TC}\AgdaSpace{}%
\AgdaRecord{⊤}\<%
\\
%
\>[4]\AgdaFunction{guardedM}\AgdaSpace{}%
\AgdaSymbol{\{}\AgdaBound{ϕ}\AgdaSymbol{\}}\AgdaSpace{}%
\AgdaSymbol{\{}\AgdaBound{ψ}\AgdaSymbol{\}}\AgdaSpace{}%
\AgdaBound{α}\AgdaSpace{}%
\AgdaBound{e}\AgdaSpace{}%
\AgdaBound{fld}\AgdaSpace{}%
\AgdaBound{hole}\AgdaSpace{}%
\AgdaSymbol{=}\AgdaSpace{}%
\AgdaKeyword{do}\<%
\\
\>[4][@{}l@{\AgdaIndent{0}}]%
\>[8]\AgdaBound{setter}\AgdaSpace{}%
\AgdaOperator{\AgdaFunction{←}}\AgdaSpace{}%
\AgdaFunction{fieldSetter}\AgdaSpace{}%
\AgdaBound{fld}\<%
\\
%
\>[8]\AgdaBound{trm}\AgdaSpace{}%
\AgdaOperator{\AgdaFunction{←}}\AgdaSpace{}%
\AgdaPostulate{quoteTC}\AgdaSpace{}%
\AgdaSymbol{(}\AgdaFunction{guarded}\AgdaSpace{}%
\AgdaSymbol{\{}\AgdaBound{ϕ}\AgdaSymbol{\}}\AgdaSpace{}%
\AgdaSymbol{\{}\AgdaBound{ψ}\AgdaSymbol{\}}\AgdaSpace{}%
\AgdaBound{e}\AgdaSpace{}%
\AgdaBound{setter}\AgdaSpace{}%
\AgdaFunction{later}\AgdaSpace{}%
\AgdaFunction{later}\AgdaSymbol{)}\<%
\\
%
\>[8]\AgdaPostulate{unify}\AgdaSpace{}%
\AgdaBound{hole}\AgdaSpace{}%
\AgdaBound{trm}\<%
\\
%
\>[4]\AgdaKeyword{syntax}\AgdaSpace{}%
\AgdaMacro{guardedM}\AgdaSpace{}%
\AgdaBound{τ}\AgdaSpace{}%
\AgdaSymbol{(λ}\AgdaSpace{}%
\AgdaBound{g}\AgdaSpace{}%
\AgdaSymbol{→}\AgdaSpace{}%
\AgdaBound{v}\AgdaSymbol{)}\AgdaSpace{}%
\AgdaBound{fld}\AgdaSpace{}%
\AgdaSymbol{=}\AgdaSpace{}%
\AgdaBound{fld}\AgdaSpace{}%
\AgdaMacro{≔}\AgdaSpace{}%
\AgdaBound{v}\AgdaSpace{}%
\AgdaMacro{given}\AgdaSpace{}%
\AgdaBound{g}\AgdaSpace{}%
\AgdaMacro{∶}\AgdaSpace{}%
\AgdaBound{τ}\<%
\end{code}
This syntax is general enough to encompass any assignment we could need, but it 
does generate two deferred obligations. Therefore, the more specialised simple 
assignment syntax is still preferable wherever possible.

\subsection{Verified list sum}
Having now completed our definitions for the surface syntax of \textsc{Trip}, we 
can now return to our list sum problem. \autoref{fig:vcg} shows a verified 
list sum implementation in \textsc{Trip}, with the proofs slightly expurgated 
for presentation purposes. The arrows in the figure show how each proof 
obligation is generated, either from an \AgdaFunction{assert} statement or 
a guarded assignment. 

We use an explicit \AgdaDatatype{assert} at the end of the loop to fix the 
loop invariant, which expresses that the \AgdaField{total} will be the sum 
of the first $\AgdaField{i}$ elements of the list, as well as the book-keeping 
invariant that $\AgdaField{i}$ is at most the length of the list.

\subsection{A more involved example}
\autoref{fig:csum} shows a more involved example of a verified \textsc{Trip} program.
This program uses the same \AgdaDatatype{SumState} type as the previous list sum example,
but this time the list is mutated to contain the running total up to that point in the original list.
For example, the list
$ \AgdaNumber{3}\ \AgdaInductiveConstructor{::}\ \AgdaNumber{1}\ \AgdaInductiveConstructor{::}\ \AgdaNumber{7}\ \AgdaInductiveConstructor{::}\ \AgdaNumber{9}\ \AgdaInductiveConstructor{::}\ \AgdaInductiveConstructor{[]} $
would become
$ \AgdaNumber{3}\ \AgdaInductiveConstructor{::}\ \AgdaNumber{4}\ \AgdaInductiveConstructor{::}\ \AgdaNumber{11}\ \AgdaInductiveConstructor{::}\ \AgdaNumber{20}\ \AgdaInductiveConstructor{::}\ \AgdaInductiveConstructor{[]} $
after executing $\AgdaFunction{csum}$.

This example necessitates a few more definitions for dealing with lists. Firstly, we need 
a way to update a list at a particular index. Like our previous list getter, this will require 
a proof that the desired index is in range:

\begin{code}
\>[2]\AgdaOperator{\AgdaFunction{\AgdaUnderscore{}[\AgdaUnderscore{}]≔\AgdaUnderscore{}}}\AgdaSpace{}%
\AgdaSymbol{:}\AgdaSpace{}%
\AgdaSymbol{(}\AgdaBound{xs}\AgdaSpace{}%
\AgdaSymbol{:}\AgdaSpace{}%
\AgdaDatatype{List}\AgdaSpace{}%
\AgdaGeneralizable{A}\AgdaSymbol{)\{}\AgdaBound{n}\AgdaSpace{}%
\AgdaSymbol{:}\AgdaSpace{}%
\AgdaDatatype{ℕ}\AgdaSymbol{\}}\AgdaSpace{}%
\AgdaSymbol{→}\AgdaSpace{}%
\AgdaBound{n}\AgdaSpace{}%
\AgdaOperator{\AgdaFunction{<}}\AgdaSpace{}%
\AgdaFunction{length}\AgdaSpace{}%
\AgdaBound{xs}\AgdaSpace{}%
\AgdaSymbol{→}\AgdaSpace{}%
\AgdaGeneralizable{A}\AgdaSpace{}%
\AgdaSymbol{→}\AgdaSpace{}%
\AgdaDatatype{List}\AgdaSpace{}%
\AgdaGeneralizable{A}\<%
\\
%
\>[2]\AgdaOperator{\AgdaFunction{\AgdaUnderscore{}[\AgdaUnderscore{}]≔\AgdaUnderscore{}}}\AgdaSpace{}%
\AgdaSymbol{(}\AgdaBound{x}\AgdaSpace{}%
\AgdaOperator{\AgdaInductiveConstructor{∷}}\AgdaSpace{}%
\AgdaBound{xs}\AgdaSymbol{)}\AgdaSpace{}%
\AgdaSymbol{\{}\AgdaInductiveConstructor{zero}\AgdaSymbol{\}}\AgdaSpace{}%
\AgdaBound{n<len}\AgdaSpace{}%
\AgdaBound{a}\AgdaSpace{}%
\AgdaSymbol{=}\AgdaSpace{}%
\AgdaBound{a}\AgdaSpace{}%
\AgdaOperator{\AgdaInductiveConstructor{∷}}\AgdaSpace{}%
\AgdaBound{xs}\<%
\\
%
\>[2]\AgdaOperator{\AgdaFunction{\AgdaUnderscore{}[\AgdaUnderscore{}]≔\AgdaUnderscore{}}}\AgdaSpace{}%
\AgdaSymbol{(}\AgdaBound{x}\AgdaSpace{}%
\AgdaOperator{\AgdaInductiveConstructor{∷}}\AgdaSpace{}%
\AgdaBound{xs}\AgdaSymbol{)}\AgdaSpace{}%
\AgdaSymbol{\{}\AgdaInductiveConstructor{suc}\AgdaSpace{}%
\AgdaBound{n}\AgdaSymbol{\}}\AgdaSpace{}%
\AgdaSymbol{(}\AgdaInductiveConstructor{s≤s}\AgdaSpace{}%
\AgdaBound{n<len}\AgdaSymbol{)}\AgdaSpace{}%
\AgdaBound{a}\AgdaSpace{}%
\AgdaSymbol{=}\AgdaSpace{}%
\AgdaBound{x}\AgdaSpace{}%
\AgdaOperator{\AgdaInductiveConstructor{∷}}\AgdaSpace{}%
\AgdaSymbol{(}\AgdaBound{xs}\AgdaSpace{}%
\AgdaOperator{\AgdaFunction{[}}\AgdaSpace{}%
\AgdaBound{n<len}\AgdaSpace{}%
\AgdaOperator{\AgdaFunction{]≔}}\AgdaSpace{}%
\AgdaBound{a}\AgdaSymbol{)}\<%
\end{code}
We must also define a predicate that holds iff the given element 
is in the given list at a given index, for use in our 
specifications and assertions. This is just a dependent pair 
of a proof that the index is in range and a proof that the element 
is found at that index:
\begin{code}
\>[2]\AgdaOperator{\AgdaFunction{\AgdaUnderscore{}[\AgdaUnderscore{}]=\AgdaUnderscore{}}}\AgdaSpace{}%
\AgdaSymbol{:}\AgdaSpace{}%
\AgdaDatatype{List}\AgdaSpace{}%
\AgdaBound{A}\AgdaSpace{}%
\AgdaSymbol{→}\AgdaSpace{}%
\AgdaDatatype{ℕ}\AgdaSpace{}%
\AgdaSymbol{→}\AgdaSpace{}%
\AgdaBound{A}\AgdaSpace{}%
\AgdaSymbol{→}\AgdaSpace{}%
\AgdaPrimitiveType{Set}\<%
\\
%
\>[2]\AgdaBound{xs}\AgdaSpace{}%
\AgdaOperator{\AgdaFunction{[}}\AgdaSpace{}%
\AgdaBound{n}\AgdaSpace{}%
\AgdaOperator{\AgdaFunction{]=}}\AgdaSpace{}%
\AgdaBound{v}\AgdaSpace{}%
\AgdaSymbol{=}\AgdaSpace{}%
\AgdaRecord{Σ}\AgdaSpace{}%
\AgdaSymbol{(}\AgdaBound{n}\AgdaSpace{}%
\AgdaOperator{\AgdaFunction{<}}\AgdaSpace{}%
\AgdaFunction{length}\AgdaSpace{}%
\AgdaBound{xs}\AgdaSymbol{)}\AgdaSpace{}%
\AgdaSymbol{λ}\AgdaSpace{}%
\AgdaBound{n<len}\AgdaSpace{}%
\AgdaSymbol{→}\AgdaSpace{}%
\AgdaSymbol{(}\AgdaBound{xs}\AgdaSpace{}%
\AgdaOperator{\AgdaFunction{!}}\AgdaSpace{}%
\AgdaBound{n<len}\AgdaSymbol{)}\AgdaSpace{}%
\AgdaOperator{\AgdaDatatype{≡}}\AgdaSpace{}%
\AgdaBound{v}\<%
\end{code}
The left hand column of \autoref{fig:csum} provides 
the specification and the loop invariant for our program.
The specification makes use of a logical freeze variable 
to refer to the original list before any changes are made. 
The post-condition states that the length of the list is unchanged,
and that the $i$th element of the list will be the sum of the first $i + 1$
elements of the original list. As can be seen by the use of Agda quantifiers
inside the specification, the full power of Agda as a logic is available when 
specifying \textsc{Trip} programs.

The loop invariant for \AgdaFunction{csum} is somewhat involved, consisting 
of five conjuncts, stating:
\begin{enumerate}
  \item that the length of the list is unchanged, 
  \item the book-keeping invariant that the loop counter $\AgdaField{i}$ is at most 
        the length of the list, 
  \item that the list up to $\AgdaField{i}$ consists of our desired running totals, 
  \item that the list from $\AgdaField{i}$ onward is still identical to the original list, and 
  \item that the variable \AgdaField{total} is the sum of all elements so far.
\end{enumerate}
The local definition of this loop invariant ($\mathit{Inv}$) is parameterised by the 
value of the variable $\AgdaField{i}$. This allows us to assert the invariant modulo substitutions
for $\AgdaField{i}$ throughout the loop (for example, after the assignment to \AgdaField{total}). 

When the loop finishes and $\AgdaField{i}$ is equal to the length of the list, 
the first and fourth conjuncts easily imply our post-condition. Nonetheless, 
establishing that the invariant is maintained and that the post-condition is met 
required approximately one hundred lines of Agda proof. As part of future work, 
we hope to integrate proof automation into our framework, hopefully reducing the amount 
of tedious proof obligations (see \autoref{subsec:futurework}). 

\section{Discussion}\label{sec:discuss}

\begin{figure*}

  \begin{tikzpicture}
\node[text width=5.5cm, anchor=south] (start) at (-12,4) {
\AgdaHole{\texttt{\{ \}0}}
};
\draw (start.south west) -- (start.south east);
\node[text width=5.5cm, anchor=north] at (start.south) { 
\texttt{?0:}\AgdaOperator{\AgdaDatatype{[}}\AgdaSpace{}%
\AgdaField{i}\AgdaSpace{}%
\AgdaOperator{\AgdaDatatype{≡}}\AgdaSpace{}%
\AgdaBound{I}\AgdaSpace{}%
\AgdaOperator{\AgdaFunction{×}}\AgdaSpace{}%
\AgdaField{j}\AgdaSpace{}%
\AgdaOperator{\AgdaDatatype{≡}}\AgdaSpace{}%
\AgdaBound{J}\AgdaSpace{}%
\AgdaOperator{\AgdaDatatype{,}}\AgdaSpace{}%
\AgdaField{j}\AgdaSpace{}%
\AgdaOperator{\AgdaDatatype{≡}}\AgdaSpace{}%
\AgdaBound{I}\AgdaSpace{}%
\AgdaOperator{\AgdaFunction{×}}\AgdaSpace{}%
\AgdaField{i}\AgdaSpace{}%
\AgdaOperator{\AgdaDatatype{≡}}\AgdaSpace{}%
\AgdaBound{J}\AgdaSpace{}%
\AgdaOperator{\AgdaDatatype{]}}%
};
\node[text width=5.5cm, anchor=south] (step0) at (-6,4) {
\AgdaHole{\texttt{\{ \}0}}
\AgdaOperator{\AgdaFunction{﹕}}\AgdaSpace{}\\
\AgdaHole{\texttt{\{ \}1}}
};
\draw (step0.south west) -- (step0.south east);
\node[text width=5.5cm, anchor=north] at (step0.south) { 
\texttt{?0:}\AgdaOperator{\AgdaDatatype{[}}\AgdaSpace{}%
\AgdaField{i}\AgdaSpace{}%
\AgdaOperator{\AgdaDatatype{≡}}\AgdaSpace{}%
\AgdaBound{I}\AgdaSpace{}%
\AgdaOperator{\AgdaFunction{×}}\AgdaSpace{}%
\AgdaField{j}\AgdaSpace{}%
\AgdaOperator{\AgdaDatatype{≡}}\AgdaSpace{}%
\AgdaBound{J}\AgdaSpace{}%
\AgdaOperator{\AgdaDatatype{,}}\AgdaSpace{}%
\textcolor{olive}{?$\alpha_1$}
\AgdaOperator{\AgdaDatatype{]}}%
\\
\texttt{?1:}\AgdaOperator{\AgdaDatatype{[}}\AgdaSpace{}%
\textcolor{olive}{?$\alpha_1$}
\AgdaOperator{\AgdaDatatype{,}}\AgdaSpace{}%
\AgdaField{j}\AgdaSpace{}%
\AgdaOperator{\AgdaDatatype{≡}}\AgdaSpace{}%
\AgdaBound{I}\AgdaSpace{}%
\AgdaOperator{\AgdaFunction{×}}\AgdaSpace{}%
\AgdaField{i}\AgdaSpace{}%
\AgdaOperator{\AgdaDatatype{≡}}\AgdaSpace{}%
\AgdaBound{J}\AgdaSpace{}%
\AgdaOperator{\AgdaDatatype{]}}%
};
\node[text width=5.5cm, anchor=south] (step1) at (0,4) {
\AgdaHole{\texttt{\{ \}0}}
\AgdaOperator{\AgdaFunction{﹕}}\AgdaSpace{}\\
\AgdaField{j}\AgdaSpace{}%
\AgdaMacro{≔}\AgdaSpace{}%
\AgdaField{temp}\AgdaSpace{}%
};
\draw (step1.south west) -- (step1.south east);
\node[text width=5.6cm, anchor=north] at (step1.south) { 
\texttt{?0:}\AgdaOperator{\AgdaDatatype{[}}\AgdaSpace{}%
\AgdaField{i}\AgdaSpace{}%
\AgdaOperator{\AgdaDatatype{≡}}\AgdaSpace{}%
\AgdaBound{I}\AgdaSpace{}%
\AgdaOperator{\AgdaFunction{×}}\AgdaSpace{}%
\AgdaField{j}\AgdaSpace{}%
\AgdaOperator{\AgdaDatatype{≡}}\AgdaSpace{}%
\AgdaBound{J}\AgdaSpace{}%
\AgdaOperator{\AgdaDatatype{,}}\AgdaSpace{}%
\AgdaField{temp}\AgdaSpace{}%
\AgdaOperator{\AgdaDatatype{≡}}\AgdaSpace{}%
\AgdaBound{I}\AgdaSpace{}%
\AgdaOperator{\AgdaFunction{×}}\AgdaSpace{}%
\AgdaField{i}\AgdaSpace{}%
\AgdaOperator{\AgdaDatatype{≡}}\AgdaSpace{}%
\AgdaBound{J}\AgdaSpace{}%
\AgdaOperator{\AgdaDatatype{]}}%
};
\node[text width=5.5cm, anchor=south] (step2) at (-12,0) {
\AgdaHole{\texttt{\{ \}0}}
\AgdaOperator{\AgdaFunction{﹕}}\AgdaSpace{}\\
\AgdaHole{\texttt{\{ \}1}}
\AgdaOperator{\AgdaFunction{﹕}}\AgdaSpace{}\\
\AgdaField{j}\AgdaSpace{}%
\AgdaMacro{≔}\AgdaSpace{}%
\AgdaField{temp}\AgdaSpace{}%
};
\draw (step2.south west) -- (step2.south east);
\node[text width=5.5cm, anchor=north] at (step2.south) { 
\texttt{?0:}\AgdaOperator{\AgdaDatatype{[}}\AgdaSpace{}%
\AgdaField{i}\AgdaSpace{}%
\AgdaOperator{\AgdaDatatype{≡}}\AgdaSpace{}%
\AgdaBound{I}\AgdaSpace{}%
\AgdaOperator{\AgdaFunction{×}}\AgdaSpace{}%
\AgdaField{j}\AgdaSpace{}%
\AgdaOperator{\AgdaDatatype{≡}}\AgdaSpace{}%
\AgdaBound{J}\AgdaSpace{}%
\AgdaOperator{\AgdaDatatype{,}}\AgdaSpace{}%
\textcolor{olive}{?$\alpha_2$}
\AgdaOperator{\AgdaDatatype{]}}\\
\texttt{?1:}\AgdaOperator{\AgdaDatatype{[}}\AgdaSpace{}%
\textcolor{olive}{?$\alpha_2$}
\AgdaOperator{\AgdaDatatype{,}}\AgdaSpace{}%
\AgdaField{temp}\AgdaSpace{}%
\AgdaOperator{\AgdaDatatype{≡}}\AgdaSpace{}%
\AgdaBound{I}\AgdaSpace{}%
\AgdaOperator{\AgdaFunction{×}}\AgdaSpace{}%
\AgdaField{i}\AgdaSpace{}%
\AgdaOperator{\AgdaDatatype{≡}}\AgdaSpace{}%
\AgdaBound{J}\AgdaSpace{}%
\AgdaOperator{\AgdaDatatype{]}}%
};
\node[text width=5.5cm, anchor=south] (step3) at (-6,0) {
\AgdaHole{\texttt{\{ \}0}}
\AgdaOperator{\AgdaFunction{﹕}}\AgdaSpace{}\\
\AgdaField{i}\AgdaSpace{}%
\AgdaMacro{≔}\AgdaSpace{}%
\AgdaField{j}\AgdaSpace{}%
\AgdaOperator{\AgdaFunction{﹕}}\AgdaSpace{}\\
\AgdaField{j}\AgdaSpace{}%
\AgdaMacro{≔}\AgdaSpace{}%
\AgdaField{temp}\AgdaSpace{}%
};
\draw (step3.south west) -- (step3.south east);
\node[text width=5.6cm, anchor=north] at (step3.south) { 
\texttt{?0:}\AgdaOperator{\AgdaDatatype{[}}\AgdaSpace{}%
\AgdaField{i}\AgdaSpace{}%
\AgdaOperator{\AgdaDatatype{≡}}\AgdaSpace{}%
\AgdaBound{I}\AgdaSpace{}%
\AgdaOperator{\AgdaFunction{×}}\AgdaSpace{}%
\AgdaField{j}\AgdaSpace{}%
\AgdaOperator{\AgdaDatatype{≡}}\AgdaSpace{}%
\AgdaBound{J}\AgdaSpace{}%
\AgdaOperator{\AgdaDatatype{,}}\AgdaSpace{}%
\AgdaField{temp}\AgdaSpace{}%
\AgdaOperator{\AgdaDatatype{≡}}\AgdaSpace{}%
\AgdaBound{I}\AgdaSpace{}%
\AgdaOperator{\AgdaFunction{×}}\AgdaSpace{}%
\AgdaField{j}\AgdaSpace{}%
\AgdaOperator{\AgdaDatatype{≡}}\AgdaSpace{}%
\AgdaBound{J}\AgdaSpace{}%
\AgdaOperator{\AgdaDatatype{]}}%
};
\node[text width=5.5cm, anchor=south] (done) at (0,0) {
\AgdaField{temp}\AgdaSpace{}%
\AgdaMacro{≔}\AgdaSpace{}%
\AgdaField{i}\AgdaSpace{}%
\AgdaOperator{\AgdaFunction{﹕}}\AgdaSpace{}\\
\AgdaField{i}\AgdaSpace{}%
\AgdaMacro{≔}\AgdaSpace{}%
\AgdaField{j}\AgdaSpace{}%
\AgdaOperator{\AgdaFunction{﹕}}\AgdaSpace{}\\
\AgdaField{j}\AgdaSpace{}%
\AgdaMacro{≔}\AgdaSpace{}%
\AgdaField{temp}
};
\draw (done.south west) -- (done.south east);
\node[text width=5.5cm, anchor=north] at (done.south) { 
  \texttt{Checked}
};
\node[circle,draw] at ([yshift=0.5cm]done.north) {6};
\node[circle,draw] at ([yshift=0.5cm,xshift=-6cm]done.north) {5};
\node[circle,draw] at ([yshift=0.5cm,xshift=-12cm]done.north) {4};
\node[circle,draw] at ([yshift=0.5cm,xshift=6cm]step0.north) {3};
\node[circle,draw] at ([yshift=0.5cm]step0.north) {2};
\node[circle,draw] at ([yshift=0.5cm,xshift=-6cm]step0.north) {1};
  \end{tikzpicture}
  \caption{Typed holes as a refinement calculus}
  \label{fig:holes}
\end{figure*}
\subsection{Typed holes as a refinement calculus}
One of the main motivations for encoding the specifications of 
\textsc{Trip} programs in their types is the ability to apply ``hole-driven''
development to derive the program from its specification.
At any point while writing the program, we can leave a hole in a location 
in the program and Agda will present us with a specification, derived from 
the surrounding statements and assertions. \autoref{fig:holes} provides 
an illustration of this process for our swap example.

\citet{Morgan} defines a minimal imperative language with \emph{specification
statements}, analogous to holes, which can be incrementally refined into
an implementation through specialised axioms in a \emph{refinement calculus}. 
The experience of deriving \textsc{Trip} programs from their specifications 
using typed holes strongly resembles a tool-assisted version of Morgan's 
refinement calculus.

One difference between Morgan's refinement calculus and our specification-typed holes
is that Morgan's specification statements include a \emph{frame}, denoting 
which variables may be modified by that part of the program. This is to enable 
compositional reasoning and to avoid having to explicitly freeze all unchanged state.
Our framework is generic on the state, however, and does not explicitly model 
variables, which makes it difficult to impose frame restrictions on 
specific parts of the state. In future, we hope to add procedures 
and function calls into \textsc{Trip}, which would necessitate 
a more fine-grained handling of state. With that in place, we could 
also annotate our specifications with frames.

\subsection{Related Work}
\subsubsection{Separating proofs from outlines}
While our proof delay applicative functor is novel, there are some other approaches 
also aimed at the separation of proofs and outlines in other proof assistants:

\paragraph{The Coq \texttt{Program} tactic} The Coq proof assistant~\citep{coq}
includes a feature called \texttt{Program}~\citep{coqprogram} where the algorithmic structure 
of a program may be written as normal, but the type of the program can form 
a rich specification. Proving that the program meets the specification is 
deferred until later, where the user will write a Coq proof script to discharge 
these deferred obligations. 

\paragraph{Predicate subtyping in PVS} The proof assistant PVS~\citep{pvs} has 
a similar feature to Coq's \texttt{Program} feature, where types are annotated 
with refinements, and the type checker generates proof obligations for the user 
to discharge~\citep{pvsprogram}.\medskip

\noindent Both of these approaches are designed as a built-in feature of 
the proof assistant, requiring extension to the tactic language or 
the type checking algorithm to enable 
interactive proof to supplement automatic typechecking for programs with
 predicate subtypes. By contrast, our applicative functor does not rely on 
 language extensions or meta-programming, but rather encodes the delineation 
 between structure and proof directly in Agda itself.

\paragraph{Ornaments} The ornaments idea, introduced originally by \citet{ornaments1}
and subsequently developed by many others~\citep{ornaments2,ornaments3,ornaments4,ornaments5},
is aimed at a similar fusion of internal and external approaches to proof in 
dependently-typed languages. Using ornaments, one can take a basic type and ornament 
it with additional type indices or proof obligations, such as turning a binary tree 
into a binary search tree. This gives rise to an isomorphism between the ornamented type 
and the product of the plain type and a descriptive structure that carries all the additional 
proof obligations. For our binary search tree example, it would establish an isomorphism 
between a binary search tree and the product of a plain binary tree with a proof that it 
satisfies the binary search tree invariants. Such an isomorphism is therefore 
performing much the same task as our \AgdaDatatype{Delay} functor, allowing proofs 
to be deferred until after the basic structure is defined.

One of the main advantages of the ornaments approach is that by starting with a basic type and 
enriching it with further details, operations can be defined generically across many types that 
share the same structure. With our \AgdaDatatype{Delay} functor approach, however, the user  
starts from a richly-typed data structure, and defers obligations as-needed. This is more flexible,
in that the user can freely choose to defer some obligations and not others, but also precludes 
the kind of generic programming possible with ornaments.

From a practical standpoint, ornaments do not integrate with Agda as cleanly:  they typically 
require encoding of datatypes in a closed universe, and such encodings can be syntactically
cumbersome. Our approach, on the other hand, is a standard applicative functor, for which Agda 
has built-in syntactic conveniences. 


\subsubsection{Imperative software verification}
There have been countless other approaches to verifying imperative software, far
too many to list here. We shall narrow our focus to those that are
based on program logics and verification conditions.

\paragraph{Hoare logic languages and VCGs}
The widely-used verification framework Why3~\citep{why3} contains a stand-alone language for both 
programming and specification. It is not embedded 
in any particular proof assistant, but is capable of generating verification 
conditions in a variety of proof assistants, or deferring to automated 
theorem proving tools such as SMT solvers. By contrast, \textsc{Trip}'s 
partially shallow embedding within Agda means that it is easily extensible 
just by writing Agda definitions, and specifications and proofs are written 
using standard Agda code.

The languages Whiley~\citep{whiley} and Dafny~\citep{dafny}, as well as the VCC~\citep{vcc}
tool for C programs, all rely on automated theorem provers exclusively.
When these tools succeed, this is very convenient as no manual proving must be done.
When they fail, however, it can be extremely difficult to coax the verification 
through the theorem prover, even when the user is capable of proving the
verification conditions by hand. 


\paragraph{VCGs in Proof Assistants}
The \textsc{simpl} language defined by \citet{simpl} strongly resembles \textsc{Trip}. Like 
\textsc{Trip}, it is embedded in a proof assistant (specifically Isabelle/HOL~\citep{isabelle}).
It is also generic on the type used for state, and shallowly 
embeds expressions and state updates. Unlike \textsc{Trip}, however, it has a 
deterministic semantics, and includes features like exceptions and procedures.
It includes a verification condition generator, but it is defined as an Isabelle
tactic, and is intended for use in \emph{post-hoc} verification of software. Unlike 
\textsc{Trip}, it does not facilitate the derivation of a program's code from its specification.

\textsc{Simpl} has been used to give a semantics for C~\citep{csem} using a 
typed memory model~\citep{cmem}. This C semantics was used for the verification 
of the seL4 microkernel~\citep{sel4} as well forming the basis of the automatic 
abstraction tool AutoCorres~\citep{autocorres1, autocorres2}.  This suggests that the 
expressivity of \textsc{Simpl} is sufficient for verification of software at scale,
which bodes well for \textsc{Trip}.

The Bedrock framework~\citep{bedrock} is another framework that is embedded 
inside a proof assistant, this time Coq. Bedrock targets low-level systems, 
and supports significant amounts of proof automation. Like \textsc{Simpl}, it
is intended for \emph{post-hoc} verification --- using a verification condition
generator tactic on a completed program.

\paragraph{Hoare-state indexed monads}
\citet{Swierstra} presents a variant of the state monad where monadic 
computations are typed by their specifications. It uses the aforementioned 
\texttt{Program} feature in Coq as a miniature verification condition generator.
Unlike \textsc{Trip}, the Hoare-state monad is entirely shallowly-embedded inside
Coq, and does not support non-determinism.


\subsection{Future Work}\label{subsec:futurework}

\subsubsection{Other uses for the proof delay applicative}
The proof delay applicative is very general, and there are likely many use 
cases we have not yet considered. In the area of software verification,  
verifying concurrent programs~\citep{owickigries} often produces large numbers of proof obligations for 
non-interference, which would be better off delayed. 

Other uses of the \texttt{Program} feature in Coq might also be good uses for 
the proof delay applicative. For example, an encoding of refinement types
in Agda which delays refinement obligations until later is a promising 
avenue for further research. The wealth of previous work on 
ornaments~\citep{ornaments1,ornaments2,ornaments3,ornaments4,ornaments5} may also provide 
some interesting examples.

\subsubsection{Proof automation}

Often we wish to delay proofs until later because they are not interesting proofs.
Certainly, when verifying \textsc{Trip} programs, many of the obligations 
generated are trivial. By integrating existing Agda proof automation techniques~\citep{agdaauto1,ringsolver,agdaauto3} 
with \textsc{Trip}, or with the proof delay applicative more generally, it may be 
possible to automatically discharge many of the deferred obligations, only 
requiring the user to prove those obligations that could not be proved automatically.

\subsubsection{Language importer}

As previously mentioned, the similar language \textsc{Simpl} has been used to 
give a semantics to C programs. Designing a similar language importer around 
\textsc{Trip} would need to generate \textsc{Trip} terms programmatically. Seeing 
as Agda's meta-programming interface does not (yet) support file I/O, this importer would 
likely have to be written outside Agda, generating and pretty-printing Agda text.

\subsubsection{State management and procedures}

For large-scale verifications to be feasible in \textsc{Trip}, we must extend 
the language to support procedures and function calls. Seeing as procedures 
also have local state, this would necessitate a more fine-grained handling of 
state than merely treating it as a monolithic abstract parameter, particularly 
if we wish to support recursion. As previously mentioned, the refinement 
calculus of \citet{Morgan} provides a possible method to handle procedures, 
local constants and local variables by integrating \emph{frames} into the
specification.


\section{Conclusion}\label{sec:conc}

We have presented \textsc{Trip}, a proof-carrying imperative embedded
language inside Agda which is typed by its specifications, to allow 
Hoare logic style derivation and verification of imperative programs.
We presented a soundness proof of these specification-types against 
a relational semantics for the language. To avoid messy proofs and 
clutter in the \textsc{Trip} derivations, we introduced the proof delay 
applicative, which allows us to cleanly separate structure from proof 
not only in \textsc{Trip} derivations, but in general. 
The entire development outlined in this paper, including all proofs 
and auxiliary definitions, can be found at this address: \nopagebreak
\begin{center}
\texttt{http://www.github.com/liamoc/dddp}
\end{center}

\begin{acks}                            %% acks environment is optional
I would like to thank Kai Engelhardt, who
indirectly inspired this paper. Thanks also to Willem-Paul de Roever and Rob van Glabbeek
who assisted me with research archaeology, and to anonymous reviewers who, among other suggestions, pointed out that 
ornaments are highly related.
                                        %% contents suppressed with 'anonymous'
  %% Commands \grantsponsor{<sponsorID>}{<name>}{<url>} and
  %% \grantnum[<url>]{<sponsorID>}{<number>} should be used to
  %% acknowledge financial support and will be used by metadata
  %% extraction tools.
%  This material is based upon work supported by the
%  \grantsponsor{GS100000001}{National Science
%    Foundation}{http://dx.doi.org/10.13039/100000001} under Grant
%  No.~\grantnum{GS100000001}{nnnnnnn} and Grant
%  No.~\grantnum{GS100000001}{mmmmmmm}.  Any opinions, findings, and
%  conclusions or recommendations expressed in this material are those
%  of the author and do not necessarily reflect the views of the
%  National Science Foundation.
\end{acks}


% Bibliography
\bibliography{cites}


%% Appendix
%\appendix
%\section{Appendix}
%
%Text of appendix \ldots

\end{document}
